\documentclass[11pt]{article}
\usepackage{amssymb}
\usepackage{amsmath}
\usepackage{mathtools}
\usepackage{enumerate}
\usepackage{esint}
\usepackage{siunitx}
\usepackage{fullpage}
\usepackage{graphicx}
\usepackage{caption}
\usepackage{subcaption}
\usepackage{wrapfig}
\usepackage{epstopdf}
\usepackage{float}
\usepackage{natbib}
\newcommand{\conj}[1]{\overline{#1}}
\newcommand{\newpar}{\vspace{5mm}\par}
\newcommand{\vnorm}[1]{\left\|#1\right\|}
\usepackage{amsthm}
\usepackage{units}
\usepackage{tikz}


\newtheorem{theorem}{Theorem}[section]
\newtheorem{lemma}{Lemma}[section]
\newtheorem{remark}{Remark}
\newtheorem{definition}{Definition}

\begin{document}
	
	\begin{center}
{\sc PhD Student Research Prize Essay -- Brian G. Kodalen}
\end{center}

Many of the terms used in this summary will be defined below. My research at WPI focuses on association schemes in their various forms. One example is to find links between certain 4-class cometric association schemes known as linked systems of symmetric designs (LSDs) and mutually unbiased bases (MUBs). I have shown that we can use the idempotents of an association scheme built from an LSD to build MUBs as long as our design has Menon parameters. This results in an upper bound on the number of fibers of any LSD with Menon parameters based on the theorem that, while as man as $v+1$ MUBs can exist in $\mathbb{C}^v$, the upper bound for $\mathbb{R}^v$ is $\frac{v}{2}+1$. It is also known that any set of $w$ MUBs give us a set of $w-1$ pairwise unbiased Hadamard matrices. Using this equivalence, I found that any set of $w$ MUBs give us an LSD on $w$ fibers whenever the $w-1$ Hadamards are mutually unbiased. % meaning that $H_i^TH_j$ scaled appropriately is also a Hadamard matrix.
This allows us to use the construction of Beth and Wocjan for MUBs in higher dimensions to construct non-trivial LSDs with larger parameter sets. We now find that, given any Menon parameter symmetric design with $v$ points, we can construct an LSD on $v^2$ points with $N$ fibers where $N$ is the maximum number of columns of an orthogonal array with entries in $[v]$. \par

Another project was to extend a result for metric association schemes to the general case. In metric schemes, any connected basis relation is $k$-connected where $k$ is the valency. A long standing conjecture of Brouwer is that this holds for any connected basis relation in any symmetric association scheme. As a step towards proving this conjecture, in joint work I have proven that any proper subset of the neighborhood of any vertex will not serve as a disconnecting set. More generally, I have shown that deleting any vertex and its neighborhood leaves behind at most one non-trivial component. The remaining trivial components only occur in certain imprimitive association schemes. This work then explores the various implications of this result. The results of this project have been submitted for publication to the Electronic Journal of Combinatorics and will be presented at the Ninth Discrete Geometry and Algebraic Combinatorics Conference in Texas this May.\par

My most recent project has been to return to the problem of LSDs and classify the examples on 16 points. These parameters correspond to Kerdock codes and are the first to admit non-trivial examples. Rudi Mathon has previously shown the number of LSDs on 16 points with 2, 3, 4 and 5-8 fibers are 3, 3, 12, and 1, respectively, up to isomorphism. His result is via a computer searche and does not illuminate the structure of the different designs. My approach uses a geometric coordinatization, group theory, and a simple transformation technique on ovals to perturb each design into the others, thereby allowing us to unify Mathon's description of the 2-fiber and 3-fiber examples. Further, we can then view the 12 distinct 4-fiber designs in terms of 4 base designs with these perturbations applied. Examining this structure carefully will help us understand the distinct ways to build LSDs and the modifications we can make to produce more non-isomorphic designs. Currently, I am investigating larger parameter sets in an attempt to build larger nontrivial examples which do not correspond to Kerdock parameters.\par

The significance of the research outlined above lies in our understanding of certain association schemes which serve as a framework for the study of problems in other areas. Mutually unbiased bases, for example, are desired by quantum physicists in order to help study quantum states. These can be used both for efficient quantum measurements as well as constructing an unbreakable cryptographic scheme. While this application only requires our vectors to be complex, understanding real MUBs further can assist our search for their complex counterparts. Beyond this, the real analogue of the MUB problem is interesting in its own right; these give us special types of 4-class association schemes. This further motivates us to understand and build non-trivial examples of LSDs in order to arrive at more examples of MUBs. Currently, building a single example of an LSD with more than two fibers with other than Menon parameters is an open problem.\par
In each problem, we always seek to understand the basic examples as well as the general structure of the object in question. While the known examples tend to have structure imposed on them, we often are able to derive some information which can be generalized to other cases. An example of such an approach is when we study LSDs on 16 points (which are well known) in an attempt to build LSDs on 36 or 45 points, the existence of which, remarkably, remains an open problem. In the case of 16 points, the natural point set is $\mathbb{Z}_2^4$. With this coordinatization, the blocks of the main design correspond to cosets of the symmetric difference of two planes. However, the only difference set with $v=45$ occurs inside $\mathbb{Z}_{15}\times\mathbb{Z}_3$; it is no longer possible to describe geometric objects such as ``planes" with such ease but the concept of an ``oval" can be abstracted to this less structured case.\par
While the search for examples is important, it is also important to continue building a theory that sheds light on these examples. For example, both LSDs and MUBs are cometric association schemes. We therefore seek to understand the different types of cometric association schemes further. Using only the fact that intersection numbers must be integral, we show that there are only 11 (non-trivial) feasible parameter sets for LSDs with less than 200 points. We can also use the non-negativity of the Krein parameters to find bounds on the number of fibers for each parameter set. In this way our project not only involves a search for examples but also the development of a theory which proves non-existence in some cases and reveals structure in others.\par
A \textit{(symmetric) association scheme} is an edge-coloring of the complete graph (with loops all colored the same) with the added property that the vector space spanned by the corresponding adjacency matrices is closed under standard matrix multiplication and entry-wise product. By diagonalizing the basis matrices, we form a second basis of primitive idempotents, one for each eigenspace. A metric (co-metric) association scheme is one in which the adjacency (idempotent) matrices can be ordered in such a way that the $i^\text{th}$ matrix is representable as a polynomial of the first matrix with degree $i$ under standard (entry-wise) multiplication. A \textit{symmetric design} is a point-block incidence structure with $v$ points and $v$ blocks where every block contains $k$ points and every pair of points occurs in $\lambda$ blocks. For example any finite projective geometry is a symmetric design. A \textit{linked system of symmetric designs} is a graph with $w$ fibers where the adjacency structure between any two fibers is the incidence structure of a symmetric design and these pairwise incidence relations are ``consistent", in a sense, when looking at triples of fibers. A set of $k$ \textit{mutually unbiased bases} is a collection of $k$ unitary bases for $\mathbb{C}^d$ where the inner product of any pair of vectors from distinct bases has modulus $\frac{1}{\sqrt{d}}$ In summary, I use tools from areas such as graph theory, group theory, and linear algebra in order to further applications in coding theory, experimental designs, and quantum information theory.\par
While carrying out this research, I have also served as a TA each term with assignments including managing the Stratton Tutoring Center, helping organize teaching teams, working with and observing PLAs who are struggling with conferences, organizing my own conferences, and assisting professors with online courses. I also taught my own section of Discrete Mathematics over the summer. I have presented in the Discrete Seminar and traveled to Rose-Hulman Institute of Technology to give a research presentation as a recruitment event for WPI. As mentioned, I am presenting at a conference in Texas this May and have been invited to present at a conference in August provided this does not conflict with my summer research program with the NSA.


\end{document}