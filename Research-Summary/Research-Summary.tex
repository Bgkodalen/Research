\documentclass[11pt]{article}
\usepackage{amssymb}
\usepackage{amsmath}
\usepackage{mathtools}
\usepackage{enumerate}
\usepackage{esint}
\usepackage{siunitx}
\usepackage{fullpage}
\usepackage{graphicx}
\usepackage{caption}
\usepackage{subcaption}
\usepackage{wrapfig}
\usepackage{epstopdf}
\usepackage{float}
\usepackage{natbib}
\newcommand{\conj}[1]{\overline{#1}}
\newcommand{\newpar}{\vspace{5mm}\par}
\newcommand{\vnorm}[1]{\left\|#1\right\|}
\usepackage{amsthm}
\usepackage{units}
\usepackage{tikz}


\newtheorem{theorem}{Theorem}[section]
\newtheorem{lemma}{Lemma}[section]
\newtheorem{remark}{Remark}
\newtheorem{definition}{Definition}

\begin{document}
	
	\begin{center}
{\sc PhD Student Research Prize Essay -- Brian G. Kodalen}
\end{center}

My research at WPI focuses on association schemes, mainly those that are $Q$-polynomial. For instance, one focus is on 3-class $Q$-antipodal association schemes known as linked systems of symmetric designs (LSSDs). Previously, it was known that these objects were equivalent to certain Mutually Unbiased Bases (MUBs) and examples were built whenever a specific parameter was a power of two. In a paper recently accepted for publication by Algebraic Combinatorics, I have classified exactly when this connection can be drawn between LSSDs and MUBs as well as introduced new geometric objects called ``Linked simplices" which are equivalent to LSSDs in general. I then use this connection to build an infinite families of new examples which correspond to MUBs without requiring the power of two restriction which was seen previously. These constructions use not only original work, but also ties in various other constructions such as MacNeish's construction for orthogonal arrays, Muzychuk and Xiang's construction for Hadamard matrices, and Wocjan and Beth's construction for MUBs. The paper finishes by classifying those families of symmetric designs which could produce non-trivial LSSDs. Out of the 21 known families of symmetric designs, I have shown that ten of these families will never produce large ($w>2$) LSSDs and another 6 families can only produce large LSSDs in very restricted cases. I have presented on this work both at the CMS winter meeting in Canada as well as the AMS sectional meeting in Ohio this year. One reviewer of the paper insisted that ``This paper is an outstanding piece of work and is clearly worthy of publication".\par

Another project was to extend a result for metric association schemes to the general case. In metric schemes, any connected basis relation is $k$-connected where $k$ is the valency. A long standing conjecture of Brouwer is that this holds for any connected basis relation in any symmetric association scheme. As a step towards proving this conjecture, in joint work I have proven that any proper subset of the neighborhood of any vertex will not serve as a disconnecting set. More generally, I have shown that deleting any vertex and its neighborhood leaves behind at most one non-trivial component. The remaining trivial components only occur in certain imprimitive association schemes. This work then explores the various implications of this result including showing that 2-connected basis relations must have $k=2$. The results of this project were published in the Electronic Journal of Combinatorics and presented at the Ninth Discrete Geometry and Algebraic Combinatorics Conference in Texas last May.\par

My current work focuses on $4$-class $Q$-bipartite schemes, where we recently were able to show that this family of schemes can be generated by exactly 3 integral parameters. These necessary parameters are the spectrum of the underlying strongly regular graph (srg) of which the association scheme covers. Knowing this allows us to examine these parameters more closely to determine which srgs will permit $Q$-bipartite double covers and, without any information about whether the SRG exists or not, I have shown that certain parameter values violate a spherical code bound from 1977 using spherical harmonic polynomials. This extra tool, motivated by work by Dr. Wei-Hsuan Yu, has ruled out eleven 4-class association schemes by showing that functions which are necessarily positive semi-definite (psd) on the entire sphere, would not be psd on our scheme due to their projections onto the eigenspaces of our scheme. These parameter sets were previously open cases and therefore this bound is independent of previous techniques used to rule out association schemes. Further, this same technique rules out nine 3-class primitive $Q$-polynomial schemes which were previously open. Thus, this bound applies in more areas than just bipartite schemes and I will generalize the requirements to apply to $Q$-polynomial association schemes of various other class numbers.\par

The significance of the research outlined above lies in the applicability of various $Q$-polynomial schemes of low class. In the $Q$-polynomial case, the number of classes refers to the number of angles that show up in the spherical code generated by the first eigenspace. Thus $Q$-polynomial schemes have a close relation with many important geometric objects. For instance, $Q$-bipartite schemes correspond to lines in real space with few angles between them. Specifically, $3$-class permits a single angle (``equiangular lines") while two possible angles give us $4$ and $5$-class schemes (depending on if the second angle is $90^\circ$ or not). Equiangular lines have been studied for over 70 years now with a resurgence in interest with the onset of quantum computing where the complex analogues of these line systems allow us to make efficient non-von Neumann measurements. The $2$-angle analogues give rise to designs with very high strength (as high as 7 in some cases) which allow us to simplify integration of polynomials on the sphere whenever the degree of the polynomial is less than the strength of our design. The $4$-class case also corresponds to MUBs when the scheme is also $Q$-antipodal which permit a unbreakable quantum algorithm for communication as well as giving von-Neumann measurements if the example is large enough. If we just consider $Q$-antipodal schemes, the $3$-class case discussed before is also equivalent to sets of ``bent functions" which have been applied in coding theory and design theory. Specific sets of these functions are known as Kerdock sets which are nonlinear binary codes with more codewords than any known linear code with comparable parameters. These sets are additionally linked to unique mathematical objects of great interest such as the $E_8$ and Leech lattices. For instance, Conway and Sloane wrote in \text{Sphere packings, Lattices and groups} that the Kerdock codes permit ``the simplest construction of the Leech lattice known to date". As every design discussed is at least a 2-design (3-design for $Q$-bipartite), more applications arise in both classical cryptography and statistical designs.\par
A \textit{(symmetric) association scheme} is an edge-coloring of the complete graph (with loops all colored the same) with the added property that the vector space spanned by the corresponding adjacency matrices is closed under standard matrix multiplication and entry-wise product. By diagonalizing the basis matrices, we form a second basis of primitive idempotents, one for each eigenspace. A metric (co-metric) association scheme is one in which the adjacency (idempotent) matrices can be ordered in such a way that the $i^\text{th}$ matrix is representable as a polynomial of the first matrix with degree $i$ under standard (entry-wise) multiplication. A \textit{symmetric design} is a point-block incidence structure with $v$ points and $v$ blocks where every block contains $k$ points and every pair of points occurs in $\lambda$ blocks. For example any finite projective geometry is a symmetric design. A \textit{linked system of symmetric designs} is a graph with $w$ fibers where the adjacency structure between any two fibers is the incidence structure of a symmetric design and these pairwise incidence relations are ``consistent", in a sense, when looking at triples of fibers. A set of $k$ \textit{mutually unbiased bases} is a collection of $k$ unitary bases for $\mathbb{C}^d$ where the inner product of any pair of vectors from distinct bases has modulus $\frac{1}{\sqrt{d}}$. For this research, I use tools from a wide range of fields such as graph and group theory, linear algebra, algebraic geometry, and coding theory.\par
While carrying out this research, I have also served as a TA each term with assignments including managing the Stratton Tutoring Center, helping organize teaching teams, working with and observing PLAs who are struggling with conferences, organizing my own conferences, and assisting professors with online courses. I also taught my own section of Discrete Mathematics over the summer. I have presented in the Discrete Seminar multiple times, traveled to Rose-Hulman Institute of Technology to give a research presentation as a recruitment event for WPI, and have been invited to present original work at the three different conferences mentioned previously. I have also been accepted for a summer research program with the NSA for the second summer in a row.


\end{document}