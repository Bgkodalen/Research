\documentclass[11pt]{article}
\usepackage{amssymb}
\usepackage{amsmath}
\usepackage{mathtools}
\usepackage{enumerate}
\usepackage{esint}
\usepackage{siunitx}
\usepackage{fullpage}
\usepackage{graphicx}
\usepackage{caption}
\usepackage{subcaption}
\usepackage{wrapfig}
\usepackage{epstopdf}
\usepackage{float}
\usepackage{natbib}
\newcommand{\conj}[1]{\overline{#1}}
\newcommand{\newpar}{\vspace{5mm}\par}
\newcommand{\vnorm}[1]{\left\|#1\right\|}
\usepackage{amsthm}
\usepackage{units}
\usepackage{tikz}


\newtheorem{theorem}{Theorem}[section]
\newtheorem{lemma}{Lemma}[section]
\newtheorem{remark}{Remark}
\newtheorem{definition}{Definition}

\begin{document}
	
	\begin{center}
{\sc PhD Student Research Prize Essay -- Brian G. Kodalen}
\end{center}

The combinatorial objects known as association schemes arise in group theory, extremal graph theory, coding theory, the design of experiments, and even Quantum information theory. My research at WPI focuses mainly on $Q$-polynomial association schemes. One may think of a $d$-class association scheme as a $d+1$ dimensional matrix algebra closed under the entrywise product. In 1999 Van Dam showed that the 3-class $Q$-antipodal association schemes were equivalent to linked systems of symmetric designs (LSSDs); these arise in the study of doubly transitive permutation groups. Results in 2009-2010 drew connections between LSSDs and real mutually unbiased bases (MUBs). An LSSD has a block structure consisting of $w$ fibers of size $v$; prior to my work, the only known examples have $v=2^{2n}$. In a paper recently accepted for publication by Algebraic Combinatorics, I have classified exactly when the connection can be drawn between LSSDs and MUBs. This paper introduced new geometric objects called ``linked simplices" and proved their equivalence with LSSDs. This insight led to new infinite families of examples where $v = 16^nt$ for odd $t\rightarrow \infty$. Further, every example I constructed yields MUBs. In addition to new tools, this work employed previous results of MacNeish, Muzychuk, Xiang, Wocjan, and Beth. The paper finishes by classifying those families of symmetric designs which could produce non-trivial LSSDs. Out of the 21 known families of symmetric designs, I have shown that ten of these families will never produce large ($w>2$) LSSDs and another 6 families can only produce large LSSDs in very restricted cases. I have presented this work both at the CMS winter meeting in Canada as well as the AMS sectional meeting in Ohio this year. One reviewer of the paper wrote ``This paper is an outstanding piece of work and is clearly worthy of publication".\par

Another completed project was to extend a result for metric association schemes to the general case. In metric schemes, any connected basis relation (i.e. symmetric 01-matrix) is $k$-connected where $k$ is the valency. Further, we find that the minimal disconnecting sets are the neighborhoods of vertices. A long-standing conjecture of Brouwer is that this holds for the connected basis relations of any symmetric association scheme. In joint work with Bill Martin, I have proven that if $R_i$ is a connected relation of a symmetric association scheme, then for any vertex $x$, deleting a proper subset of the neighborhood of $x$ will not disconnect the graph. More generally, I have shown that deleting $x$ with its neighborhood will leave behind at most one non-singleton component where the singleton components can only occur in specific imprimitive schemes. Our work then explores the various implications of this result. This represents the first progress towards Brouwer's conjecture since 1996. The results of this project were published in the Electronic Journal of Combinatorics; I also presented this at the Ninth Discrete Geometry and Algebraic Combinatorics Conference in Texas last May.\par

My current work focuses on $4$-class $Q$-bipartite schemes, where we recently were able to show that this family of schemes can be generated by exactly 3 integral parameters: the eigenvalues of the 2-class scheme which appears as the quotient space. This knowledge allows us to zero in on those 2-class schemes which admit such double covers. In analogy with the metric case, where our space is $\text{span}\left\{f_i(A):i=0,1,\dots\right\}$ for 01-matrix $A$ and orthogonal polynomials $f_0(t),\dots$, in the co-metric situation the algebra is $\text{span}\left\{f_i\circ (G):i=0,1,\dots\right\}$ for a fixed Gram matrix $G$ and orthogonal polynomials $f_0(t),\dots$ applied entrywise. Further, the matrices $f_i\circ(G)$ are the generators of the positive semi-definite cone of our algebra. A semi-definite programming exploration, joint with Wei-Hsuan Yu (ICERM), led to non-existence results in isolated cases where certain Gegenbauer polynomials (zonal polynomials satisfying $\Delta f = 0$) map $G$ outside this cone, contradicting a result of Delsarte, et al.\ stating that $f\circ(G)$ must be psd for any Gegenbauer $f$ and Gram matrix $G$. Inspired by these computations, I have extended these results analytically to find bounds on the parameters of the 4-class objects. Comparing these bounds with previously known results, I have ruled out an infinite number of potential association schemes. Computations also indicate utility in the 3-class case; ongoing work aims to generalize the results to any dimension.\par


The significance of the research outlined above lines in the applicability of various $Q$-polynomial schemes with few classes. In the $Q$-polynomial case, the matrix algebra is entirely determined by the Gram matrix of a finite subset of the unit sphere (spherical code); the number of classes is the number of non-zero angles that appear. For example, all five platonic solids are association schemes. Like the icosahedron, 3-class $Q$-bipartite schemes correspond to sets of equiangular lines, while sets of lines with two angles arise from 4-~and $5$-class schemes (depending on if the second angle is $90^\circ$ or not). Equiangular lines have been studied for over 70 years, with a resurgence of interest at the onset of quantum computing where the complex analogues include the most efficient non-von Neumann measurements. The $2$-angle analogues give effective cubature rules, allowing us to approximate spherical integrals when the integrand is ``close" to a small degree polynomial. The $4$-class cases correspond to MUBs when the scheme is also $Q$-antipodal; these give rise to an unbreakable quantum communication algorithm as well as sets of ``independent" von Neumann measurements. If we just consider $Q$-antipodal schemes, the $3$-class case discussed before is also equivalent to sets of ``bent functions" which have been applied in coding theory and cryptography. Specific types of these LSSDs give us Kerdock sets which are nonlinear binary codes with more codewords than any known linear code with comparable parameters. These sets are additionally linked to lattices, root systems, and finite permutation groups.\par
A \textit{(symmetric) association scheme} is an edge-coloring of the complete graph (with loops all colored the same) with the added property that the vector space spanned by the corresponding adjacency matrices is closed under standard matrix multiplication and entry-wise product. By diagonalizing the basis matrices, we form a second basis of primitive idempotents, one for each eigenspace. A metric (co-metric) association scheme is one in which the adjacency (idempotent) matrices are determined by a single generator and a family of orthogonal polynomials, applied entrywise in the co-metric case. Generalizing a finite projective geometry, a \textit{symmetric design} is a point-block incidence structure with $v$ points and $v$ blocks where every block contains $k$ points and every pair of points occurs in $\lambda$ blocks. A \textit{linked system of symmetric designs} (LSSD) is a graph with $w$ fibers where the subgraph induced on two fibers is the incidence graph of a symmetric design and given any fiber ($X$), the incidence between $X$ and two other fibers $Y$ and $Z$, uniquely determines the incidence between $Y$ and $Z$. A set of $k$ \textit{mutually unbiased bases} ($k$ MUBs) is a collection of $k$ unitary bases for $\mathbb{C}^d$ where the inner product of any pair of vectors from distinct bases has modulus $\frac{1}{\sqrt{d}}$. For this research, I use tools from a wide range of fields such as graph theory, group theory, linear algebra, algebraic geometry, and coding theory.\par
While carrying out this research, I have also worked as a TA throughout the year and taught my own section of Discrete Mathematics over the summer. I have presented my research in the Discrete Seminar multiple times, at Rose-Hulman Institute of Technology as a recruitment event for WPI, and at the three conferences mentioned previously. I have also been accepted for a summer research program with the NSA for the second summer in a row.
\end{document}