\chapter{Introduction}\label{introduction}
\begin{chapquote}{Bertrand Russell, 1907}
		``Mathematics, rightly viewed, possesses not only truth, but supreme beauty"
\end{chapquote}
Many coding theorist know the story of the Nordstrom-Robinson code \cite{Nordstrom1967}. Robinson, a professor at the University of Iowa at the time, gave an introductory talk on coding theory to high school students in the mid 1960's in which he discussed both linear and nonlinear codes. After giving the best possible linear code of length 15 and minimum distance 5 (the Johnson code), Robinson pointed out that the best known upper bound on nonlinear codes with the same properties was a factor of 2 greater. One of the students, Nordstrom, responded to the challenge to find such a code and, through trial and error, was able to produce the ``Nordstrom-Robinson code"---a binary code fitting the upper bound exactly. While this code attracted attention quickly, it was discovered within a few years that the extended version (one extra information bit) was the first of two infinite families of nonlinear codes, first the Preparata codes in 1968 \cite{Preparata1968} and four years later the Kerdock codes \cite{Kerdock1972}. Even more interesting was that these two families had dual parameters---a notion which was defined initially only for linear codes. It was not until two decades later in 1994 that Hammons et al.\ \cite{Hammons1994} found that these two families are $\bbZ_4$-linear and, in fact, dual to each other in the traditional sense. Apart from this interesting example of duality, the Kerdock codes in particular coincide with optimal examples of other objects. For one, Goethals communicated to Cameron \cite{Cameron1972} that one may build examples of linked systems of symmetric designs (see Section \ref{kerdock}) which are optimal in regards to the number of fibers \cite{Noda1974}. These LSSDs became known as the Cameron-Seidel scheme, becoming the prototypical example of LSSDs even to this day. A second (though not independent) relationship is that each Kerdock code gives rise to a set of real mutually unbiased bases with the maximum number of bases in that given dimension \cite{Calderbank1997}---this is the only known infinite family of real MUBs achieving the upper bound. Finally, de Caen noticed that one may use the Cameron-Seidel scheme to build $\frac{2}{9}\left(d+1\right)^2$ real equiangular lines whenever $d = 3\left(2^{2t-1}\right)-1$ for some positive integer $t$. This was the first known example of an infinite family of equiangular lines for which the number of lines scaled quadratically with the dimension.

Central to many of these connections is the fact that the Cameron-Seidel scheme---derived from the Kerdock codes---admits a 3-class association scheme on its points. First introduced by Bose and Nair \cite{Bose1939} in 1939 with connections to certain block designs, the algebraic structure known as an ``association scheme" was formally defined later in 1959 by Bose and Shimamoto \cite{Bose1952} as a set of relations on a point set fulfilling strong regularity properties (see Chapter \ref{association}). It was not until seven years later that Bose and Mesner \cite{Bose1959} described the equivalence between association schemes and Schur-closed matrix algebras---vector spaces of matrices closed under two distinct matrix products. Around the same time Wielandt was expanding on the theory of Schur (\cite{Wielandt1964},\cite{Schur1933}) to understand the commuting algebra, or centralizer ring, of permutation groups. These two concepts were generalized together by Higman in 1967 \cite{Higman1967} who discussed so-called ``coherent configurations". Shortly thereafter Biggs introduced a generalization of distance-transitive graphs known as ``distance regular graphs" \cite{Biggs1971}, showing that the adjacency matrix of any such graph generates the matrix algebra of an association scheme with very particular properties. Over the next few years, Biggs continued to develop the notion of distance-regular graphs and their relationship with association schemes, culminating in parts of his book Algebraic Graph Theory \cite{Biggs1974}. Arguably one of the most influential works on this topic is the thesis of Delsarte in 1973 \cite{Delsarte1973}, developed seemingly independently of Biggs. In this thesis, he lays out the definitions and parameters central to association schemes, discusses subsets of associations schemes, and defines both $P$-polynomial (metric) and $Q$-polynomial (cometric) association schemes; the former is equivalent to Bigg's notion of distance regular graphs while the latter will be the main focus of this thesis. He focuses primarily on the Johnson and Hamming schemes, defining more clearly the notion of duality within these two schemes and bringing to the forefront the connections between association schemes and coding theory.

The two decades that followed brought with them many new results concerning polynomial association schemes, especially those that are $P$-polynomial. Authors such as Biggs (\cite{Biggs1976},\cite{Biggs1980},\cite{Biggs1986}), Damerell (\cite{Damerell1973},\cite{Damerell1981},\cite{Damerell1981b},\cite{Damerell1981c}), Gardiner (\cite{Gardiner1974},\cite{Gardiner1980},\cite{Gardiner1981},\cite{Gardiner1982}), Meredith (\cite{Meredith1976}), and Smith (\cite{Smith1971},\cite{Smith1974},\cite{Smith1974b},\cite{Smith1975}) continued to develop our understanding of distance-regular and distance-transitive graphs. Meanwhile authors such as Terwilliger \cite{Terwilliger1986} and Neumaier \cite{Neumaier1985} focused on specific families, classifying all Johnson and Hamming schemes---the major examples of metric association schemes. Terwilliger then went on to work towards classifying association schemes which are both $P$- and $Q$-polynomial in papers such as \cite{Terwilliger1987} and \cite{Terwilliger1988}. Two very important books were compiled on this topic, first by Bannai and Ito in ``Algebraic Combinatorics I" \cite{Bannai1984} and later by Brouwer, Cohen, and Neumaier in ``Distance-Regular Graphs" \cite{Brouwer1989}.

Despite the great attention applied to $P$-polynomial schemes, it seems not much progress was made in understanding their $Q$-polynomial analogues until Dickie's thesis \cite{Dickie1995} in 1995 and two papers of Suzuki (\cite{Suzuki1998},\cite{Suzuki1998-2}) three years later. In the latter two papers, Suzuki showed that, apart from cycles, any $Q$-polynomial association scheme may have at most two $Q$-polynomial orderings and that any imprimitive $Q$-polynomial association scheme must be either $Q$-bipartite or $Q$-antipodal (except possibly for two sporadic cases which were later ruled out). These results were analogues of results for distance regular graphs dating as much as three decades prior, yet the method for proving these results was quite different. It would seem that these papers brought a resurgence of interest in cometric association schemes as the following two decades brought many new results. Some results included finding equivalences between certain classes of cometric association schemes and other geometric structures; for instance \cite{VanDam1999} discusses $3$-class $Q$-antipodal schemes while \cite{LeCompte2010} focuses on $4$-class schemes which are both $Q$-antipodal and $Q$-bipartite. New examples were found including families discovered by Penttila and Williford \cite{Penttila2011}, another family found by Moorhouse and Williford \cite{Moorhouse2016}, and many new sporadic examples found by Gavin King \cite{King2018}. While far from an exhaustive list, this author would be amiss without also mentioning results of Suda (\cite{Suda2011},\cite{Suda2012}), van Dam et al.\ \cite{vanDam2013}, Martin and Williford \cite{Williford2009}, and Martin et al.\ \cite{Martin2007}.

In this thesis, we begin by defining association schemes and their associated Bose-Mesner algebras. The remainder of Chapter \ref{association} consists of the various definitions which occur within this field such as the parameters, feasibility conditions, substructures, and polynomial structures---some of which are mentioned above. We then focus on the matrix algebra in Chapter \ref{psdcone}, where we examine the cone of positive semi-definite matrices. Here, we introduce methods to build other line systems, such as equiangular lines, as well as develop a new feasibility condition on association schemes using a theorem of Sch\"{o}nberg's---we explicitly calculate many of these new conditions for cometric schemes in particular. In Chapter \ref{3class}, we recall the definition of linked systems of symmetric designs (LSSDs) defined by Cameron \cite{Cameron1972} in the context of groups with multiple doubly transitive permutation representations. We review past results on such objects including the Cameron-Seidel scheme mentioned previously, results of Noda on the maximum size of any such object, and their equivalence with 3-class $Q$-antipodal association schemes which was found by van Dam. We then introduce a new geometric object called ``sets of linked simplices" which we show are also equivalent to LSSDs. Using this new equivalence, we investigate when we may build real mutually unbiased bases from these association schemes as well as construct new examples using parameters distinct from those of the Cameron-Seidel scheme. The following chapter focuses on $4$-class $Q$-bipartite association schemes, beginning with a motivation of how these association schemes naturally occur. While we do not develop an equivalence in this chapter between the association schemes and so-called orthogonal projective double covers, we make substantial progress towards that goal. In the final chapter, we investigate the connectivity of relations in association schemes in general. With the goal of understanding the ``nearest-neighbor graph" of a cometric association scheme, we show that any connected graph within an association scheme remains connected after deleting the entire neighborhood of a vertex (including the vertex itself). We use this result to show that any graph within an association scheme with connectivity equal to 2 must be a cycle. In the appendix, we finish be surveying the known infinite families of symmetric designs in order to determine which families may produce LSSDs.
