\chapter{Introduction}\label{introduction}
\begin{chapquote}{Bertrand Russell, 1907}
		``Mathematics, rightly viewed, possesses not only truth, but supreme beauty"
\end{chapquote}
Often the most immediate applications of much of our work will be found in coding theory; we begin by introducing one such connection. A binary code of length $n$ may simply be viewed as a subset of $\mathbb{Z}_2^n$. While there are countless different ways to form binary codes, consider the Reed Muller codes for a moment. This family of codes, denoted $\mathcal{R}(t,m)$ for $0\leq t\leq m$, is formed by first defining a codeword for each polynomial in $\bbZ_2[x_1,\dots,x_m]$ by evaluating the polynomial at every point in $\mathbb{Z}_2^m$. Then $\mathcal{R}(t,m)$ is the linear span of all codewords corresponding to polynomials of degree $t$ or less. Building a code in this way produces what is called a \emph{linear code} --- that is, a code which forms a subspace, rather than simply a subset. In describing a linear code, we often give a \emph{generator matrix} --- a matrix whose row span is the desired code. For instance, $\mathcal{R}(1,4)$ uses the five polynomials $1$, $x_1$, $x_2$, $x_3$, and $x_4$, resulting in the following generator matrix
\[\left[\begin{array}{cccccccccccccccc}
1 & 1 & 1 & 1 & 1 & 1 & 1 & 1 & 1 & 1 & 1 & 1 & 1 & 1 & 1 & 1\\
0 & 0 & 0 & 0 & 0 & 0 & 0 & 0 & 1 & 1 & 1 & 1 & 1 & 1 & 1 & 1\\
0 & 0 & 0 & 0 & 1 & 1 & 1 & 1 & 0 & 0 & 0 & 0 & 1 & 1 & 1 & 1\\
0 & 0 & 1 & 1 & 0 & 0 & 1 & 1 & 0 & 0 & 1 & 1 & 0 & 0 & 1 & 1\\
0 & 1 & 0 & 1 & 0 & 1 & 0 & 1 & 0 & 1 & 0 & 1 & 0 & 1 & 0 & 1\\
\end{array}\right].\]

This linear code contains $32$ distinct codewords, though the size of the code increases rapidly with $t$. In fact, $\mathcal{R}(2,4)$ contains $2048$ distinct codewords. Unfortunately, it is not only the number of codewords we typically care about, another main parameter we are interested in is the \emph{minimum distance} --- the smallest number of distinct entries between any pair of codewords. It is in this parameter that we pay for the extra codewords in the higher order Reed Muller code; the minimum distance of $\mathcal{R}(1,4)$ is $8$, while $\mathcal{R}(2,4)$ has a minimum distance of only half that. Thus the question arises: is there a code somewhere between these two which consists of a large number of codewords, yet with high minimum distance. To make the question more precise, what is the largest subcode one may find within $\mathcal{R}(2,4)$ such that the minimum distance is $6$? One may show that any generator matrix cannot have more than 7 rows and thus we will not find any linear subcode with more than 128 codewords. However, we may do better than this if we do not require linearity. Thus, we will instead define a code explicitly by providing a polynomial for each and every codeword. First, consider the eight quadratic polynomials
\[\begin{aligned}p_1 &=x_1x_2+x_1x_3+x_1x_4+x_2x_3+x_2x_4+x_3x_4,\\
p_2 &= x_1x_2+x_2x_3+x_3x_4,\\p_3 &=x_1x_2+x_2x_4+x_4x_3,\\p_4 &=x_1x_3+x_3x_2+x_2x_4,\\p_5 &=x_1x_3+x_3x_4+x_4x_2,\\p_6 &=x_1x_4+x_4x_2+x_2x_3,\\p_7 &=x_1x_4+x_4x_3+x_3x_2,\\p_8 &=0.\end{aligned}\]
We then form 256 polynomials by adding any one of these quadratics to each of the 32 polynomials with degree 0 or 1. One may verify that the resultant code has minimum distance 6. This code is known as the (extended) Nordstrom-Robinson code, the first in an infinite family of non-linear codes which may be defined similarly by taking cosets of the first order Reed Muller codes.

It turns out this code has an interesting history behind it. The code, originally given in \cite{Nordstrom1967}, was found by a high-school student after attending an introductory talk at his school. Robinson, a professor at the University of Iowa at the time, gave the talk in the mid 1960's in which he discussed both linear and nonlinear binary codes. After introducing the best possible linear code of length 15 and minimum distance 5 (the double-error-correcting BCH code), Robinson pointed out that the upper bound on nonlinear codes with the same length and minimum distance was a factor of 2 greater --- yet no such code was known.  Nordstrom responded to the challenge and, through trial and error, was able to produce what is now known as the Nordstrom-Robinson code. This code attracted attention quickly and within a few years it was discovered that the extended version (as described above) may be generalized to two infinite families of nonlinear codes, first the Preparata codes in 1968 \cite{Preparata1968} and four years later the Kerdock codes \cite{Kerdock1972}.

Perhaps one of the most intriguing question arising from these families at the time was the notion that they were dual to each other --- despite the notion of ``duality" being a property of linear codes. For linear codes, we define the ``dual-code" as the null-space of the generator matrix --- that is, the dual code consists of all codewords which are orthogonal to every codeword of the original code. MacWilliams \cite{MacWilliams1963} had further developed this notion of duality by using weight enumerators to calculate the parameters of the dual code without having to explicitly calculate the code. It was through these MacWilliams identities that it was noticed the Preparata and Kerdock codes had dual parameters, despite there being no known dual mapping from one to the other. It was not until two decades later in 1994 that Hammons et al.\ \cite{Hammons1994} showed that these two codes are the images of vector spaces in $\bbZ_4^n$ --- that is, they are linear when viewed as codes of length 8 with an alphabet of size 4. Further, as $\bbZ_4$-linear codes, they were shown to be dual in the traditional sense.

Outside of the question of duality, the Kerdock codes have many other, quite fascinating, connections. In the early 1970s, Cameron introduced a type of multi-partite graph called a ``linked system of symmetric designs" (see \cite{Cameron1972}). Around that time, Goethals communicated to Cameron that one may build examples of such objects using the Kerdock codes; these examples where shown to be optimal in regards to the number of fibers \cite{Noda1974}. This family of LSSDs became known as the Cameron-Seidel scheme (see Section \ref{kerdock}), remaining the archetypal example of LSSDs even to this day. A second (though not completely independent) connection of Kerdock codes is that of real mutually unbiased bases. Here, we look for orthonormal bases in $\bbR^m$ where vectors from distinct bases have an inner product of $\pm\frac{1}{\sqrt{m}}$. With connections to quantum cryptography and Euclidean geometry, mutually unbiased bases have been an area of interest for quite some time now. It was shown using quadratic forms \cite{Cameron1973} that the Kerdock sets not only gave examples of real MUBs, but that these examples were optimal with respect to the number of bases \cite{Calderbank1997} --- this is the only known infinite family of real MUBs achieving this upper bound. A similar problem is that of finding lines in $\bbR^m$ in which any pair of lines intersect in a fixed angle; such sets of lines are called ``equiangular lines". Gerzon showed that the upper bound on the number of lines in $\bbR^m$ is given by $\frac{m(m+1)}{2}$ \cite{Lemmens1973}, yet the known constructions all scaled linearly with the dimension. It was not until nearly 30 years later that de Caen \cite{deCaen2000} used the Cameron-Seidel scheme to build $\frac{2}{9}\left(d+1\right)^2$ real equiangular lines whenever $d = 3\left(2^{2t-1}\right)-1$ for some positive integer $t$, resulting in the first (and only) known infinite family with quadratic scaling.

Central to many of these connections is the fact that the Cameron-Seidel scheme --- derived from the Kerdock codes --- admits a 3-class association scheme on its points. Within the study of association schemes, we will often find rich connections to other areas of mathematics. In this thesis we will examine a type of association schemes known as ``cometric association schemes" (see Section \ref{poly} for the definition) which include many of the objects mentioned already. Further, we find that within this field lies another odd notion of duality. While duality is well-defined for certain association schemes, namely those for which there exists a group acting sharply transitively on the points, there is no clear way to build the ``dual" of a general association scheme. Despite this, we may define duality formally --- as was done for non-linear codes using the MacWilliams identities --- and find concrete examples of formally dual association schemes without a clear mapping from one to the other. We finish this introduction with a brief history of association schemes as well as an outline of this thesis.

First introduced by Bose and Nair \cite{Bose1939} in 1939 with connections to certain block designs, the algebraic structure known as an ``association scheme" was formally defined later in 1959 by Bose and Shimamoto \cite{Bose1952} as a set of relations on a point set fulfilling strong regularity properties (see Chapter \ref{association}). It was not until seven years later that Bose and Mesner \cite{Bose1959} described the equivalence between association schemes and Schur-closed matrix algebras --- vector spaces of matrices closed under two distinct matrix products. Around the same time Wielandt was expanding on the theory of Schur (\cite{Wielandt1964},\cite{Schur1933}) to understand the commuting algebra, or centralizer ring, of permutation groups. These two concepts were generalized together by Higman in 1967 \cite{Higman1967} who discussed so-called ``coherent configurations". Shortly thereafter Biggs introduced a generalization of distance-transitive graphs known as ``distance regular graphs" \cite{Biggs1971}, showing that the adjacency matrix of any such graph generates the matrix algebra of an association scheme with very particular properties. Over the next few years, Biggs continued to develop the notion of distance-regular graphs and their relationship with association schemes, culminating in parts of his book Algebraic Graph Theory \cite{Biggs1974}. Arguably one of the most influential works on this topic is the thesis of Delsarte in 1973 \cite{Delsarte1973}, developed seemingly independently of Biggs. In this thesis, he lays out the definitions and parameters central to association schemes, discusses subsets of associations schemes, and defines both $P$-polynomial (metric) and $Q$-polynomial (cometric) association schemes; the former is equivalent to Bigg's notion of distance regular graphs while the latter will be the main focus of this thesis. He focuses primarily on the Johnson and Hamming schemes, defining more clearly the notion of duality within these two schemes and bringing to the forefront the connections between association schemes and coding theory.

The two decades that followed brought with them many new results concerning polynomial association schemes, especially those that are $P$-polynomial. Authors such as Biggs (\cite{Biggs1976},\cite{Biggs1980},\cite{Biggs1986}), Damerell (\cite{Damerell1973},\cite{Damerell1981},\cite{Damerell1981b},\cite{Damerell1981c}), Gardiner (\cite{Gardiner1974},\cite{Gardiner1980},\cite{Gardiner1981},\cite{Gardiner1982}), Meredith (\cite{Meredith1976}), and Smith (\cite{Smith1971},\cite{Smith1974},\cite{Smith1974b},\cite{Smith1975}) continued to develop our understanding of distance-regular and distance-transitive graphs. Meanwhile authors such as Terwilliger \cite{Terwilliger1986} and Neumaier \cite{Neumaier1985} focused on specific families, classifying all Johnson and Hamming schemes --- the major examples of metric association schemes. Terwilliger then went on to work towards classifying association schemes which are both $P$- and $Q$-polynomial in papers such as \cite{Terwilliger1987} and \cite{Terwilliger1988}. Much of what is known has been compiled into books, first by Bannai and Ito in ``Algebraic Combinatorics I" \cite{Bannai1984}, then by Brouwer, Cohen, and Neumaier in ``Distance-Regular Graphs" \cite{Brouwer1989}, and most recently by Bailey in ``Association Schemes" \cite{Bailey2004}.

Despite the great attention applied to $P$-polynomial schemes, it seems not much progress was made in understanding their $Q$-polynomial analogues until Dickie's thesis \cite{Dickie1995} in 1995 and two papers of Suzuki (\cite{Suzuki1998},\cite{Suzuki1998-2}) three years later. In the latter two papers, Suzuki showed that, apart from cycles, any $Q$-polynomial association scheme may have at most two $Q$-polynomial orderings and that any imprimitive $Q$-polynomial association scheme must be either $Q$-bipartite or $Q$-antipodal (except possibly for two sporadic cases which were later ruled out). These results were analogues of results for distance regular graphs dating as much as three decades prior, yet the method for proving these results was quite different. It would seem that these papers brought a resurgence of interest in cometric association schemes as the following two decades brought many new results. Some results included finding equivalences between certain classes of cometric association schemes and other geometric structures; for instance \cite{VanDam1999} discusses $3$-class $Q$-antipodal schemes while \cite{LeCompte2010} focuses on $4$-class schemes which are both $Q$-antipodal and $Q$-bipartite. New examples were found including families discovered by Penttila and Williford \cite{Penttila2011}, another family found by Moorhouse and Williford \cite{Moorhouse2016}, and many new sporadic examples found by Gavin King \cite{King2018}. While far from an exhaustive list, this author would be amiss without also mentioning results of Suda (\cite{Suda2011},\cite{Suda2012}), van Dam et al.\ \cite{vanDam2013}, Martin and Williford \cite{Williford2009}, and Martin et al.\ \cite{Martin2007}.

In this thesis, we begin by defining association schemes and their associated Bose-Mesner algebras. The remainder of Chapter \ref{association} consists of the various definitions which occur within this field such as the parameters, feasibility conditions, substructures, and polynomial structures --- some of which are mentioned above. We then focus on the matrix algebra in Chapter \ref{psdcone}, where we examine the cone of positive semi-definite matrices. Here, we introduce methods to build other line systems, such as equiangular lines, as well as develop a new feasibility condition on association schemes using a theorem of Sch\"{o}nberg's --- we explicitly calculate many of these new conditions for cometric schemes in particular. In Chapter \ref{3class}, we recall the definition of linked systems of symmetric designs (LSSDs) defined by Cameron \cite{Cameron1972} in the context of groups with multiple doubly transitive permutation representations. We review past results on such objects including the Cameron-Seidel scheme mentioned previously, results of Noda on the maximum size of any such object, and their equivalence with 3-class $Q$-antipodal association schemes which was found by van Dam. We then introduce a new geometric object called ``sets of linked simplices" which we show are also equivalent to LSSDs. Using this new equivalence, we investigate when we may build real mutually unbiased bases from these association schemes as well as construct new examples using parameters distinct from those of the Cameron-Seidel scheme. The following chapter focuses on $4$-class $Q$-bipartite association schemes, beginning with a motivation of how these association schemes naturally occur. While we do not develop an equivalence in this chapter between the association schemes and so-called orthogonal projective double covers, we make substantial progress towards that goal. In the final chapter, we investigate the connectivity of relations in association schemes in general. With the goal of understanding the ``nearest-neighbor graph" of a cometric association scheme, we show that any connected graph within an association scheme remains connected after deleting the entire neighborhood of a vertex (including the vertex itself). We use this result to show that any graph within an association scheme with connectivity equal to 2 must be a cycle. In the appendix, we finish be surveying the known infinite families of symmetric designs in order to determine which families may produce LSSDs.
