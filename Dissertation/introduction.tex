\chapter{Introduction}\label{introduction}
\begin{chapquote}{Bertrand Russell, 1907}
		``Mathematics, rightly viewed, possesses not only truth, but supreme beauty"
\end{chapquote}
Much of the motivation for the theory of association schemes arises from coding theory; for the purpose of illustration, we will use this application as an entry point into our discussion of association schemes. A binary code of length $n$ may simply be viewed as a subset of $\mathbb{Z}_2^n$. First consider the parity check code on two bits: $P = \left\{000,011,101,110\right\}.$ This code has the additional property that it forms a subspace of $\mathbb{Z}_2^3$, not just a subset; any code with this property is called a \emph{linear code}. Given a linear code, $C$, we may represent the code using a \emph{generator matrix} -- a matrix whose rows form a basis for $C$; we say the \emph{dimension} of $C$ is the number of rows in the generator matrix. For instance, the parity check code on two bits may be described as $P = \text{rowspan}\left[\begin{array}{ccc}
1 & 1 & 0\\
0 & 1 & 1\\
\end{array}\right]$ and thus has dimension $2$. We may equivalently define this code via $P = \text{null}\left[\begin{array}{ccc}
1 & 1 & 1\\
\end{array}\right]$. The dual code of a linear code $C$, denoted $C^\perp$, consists of the subspace formed by swapping the two previous matrices. Returning to our example, we find $P^\perp = \text{null}\left[\begin{array}{ccc}
1 & 1 & 0\\
0 & 1 & 1\\
\end{array}\right] = \text{rowspan}\left[\begin{array}{ccc}
1 & 1 & 1\\
\end{array}\right].$ Given a code $C$ of length $n$, we may form graphs $\Gamma_1,\dots,\Gamma_n$ on $C$ where two codewords are adjacent in $\Gamma_i$ if and only if they differ in exactly $i$ positions. Using $P$ as our code, we find $\Gamma_1$ and $\Gamma_3$ are both empty while $\Gamma_2\simeq K_4$, the complete graph on four vertices. Using the dual code $P^\perp$ instead, we find that $\Gamma_1$ and $\Gamma_2$ are both empty while $\Gamma_3\simeq K_2$. The interaction of these two codes and their corresponding graphs will be discussed later as subobjects of one association scheme called the 3-cube.

We now move to a family of codes known as the Reed Muller codes, denoted $\mathcal{R}(t,m)$ for $t\geq 0$ and $m\geq 1$. For fixed $t$ and $m$, $\mathcal{R}(t,m)$ is a linear code of length $2^m$ with dimension $\sum_{i=0}^t\binom{m}{i}$. While there are many ways to represent the codewords of this family, we will use a construction relying on binary polynomials. Let $P_t\subset\bbZ_2[x_1,\dots,x_m]$ be the space of binary polynomials of degree $t$ or less on $m$ variables. We begin by imposing an ordering on the elements of $\bbZ_2^m$, say $p_1,\dots,p_{2^m}$. Then, for each $f\in P_t$, we build the corresponding codeword, $c_f$, by evaluating $f$ at each point of $\bbZ_2^m$ in order; that is, the $i^\text{th}$ element of $c_f$ is given by $f(e_i)$. As $P_t$ is a vector space, we may create codewords for each polynomial in some basis of $P_t$ and use the resultant codewords as the rows of our generator matrix; that is, if $\left\{f_1,\dots,f_\ell\right\}$ is a basis for $P_t$ then $c_{f_1},\dots,c_{f_\ell}$ is a basis for the binary code. Since we have indexed each entry of each codeword by some element of $\bbZ_2^m$, we find that this code has length $2^m$. Further, consider the basis of $P_t$ given by the set of monomials. This basis has $\sum_{i=0}^t\binom{m}{i}$ polynomials, giving us the dimension of our code. For example, $\mathcal{R}(1,4)$ may be generated using the following generator matrix $M$, where the rows of $M$ are indexed by the basis $1$, $x_1$, $x_2$, $x_3$, and $x_4$, while the columns are indexed by the elements of $\mathbb{Z}_2^4$ ordered lexicographically. Then the element in row $p$ and columns $x$ of $M$ is $p(x)$.
\[M = \left[\begin{array}{cccccccccccccccc}
1 & 1 & 1 & 1 & 1 & 1 & 1 & 1 & 1 & 1 & 1 & 1 & 1 & 1 & 1 & 1\\
0 & 0 & 0 & 0 & 0 & 0 & 0 & 0 & 1 & 1 & 1 & 1 & 1 & 1 & 1 & 1\\
0 & 0 & 0 & 0 & 1 & 1 & 1 & 1 & 0 & 0 & 0 & 0 & 1 & 1 & 1 & 1\\
0 & 0 & 1 & 1 & 0 & 0 & 1 & 1 & 0 & 0 & 1 & 1 & 0 & 0 & 1 & 1\\
0 & 1 & 0 & 1 & 0 & 1 & 0 & 1 & 0 & 1 & 0 & 1 & 0 & 1 & 0 & 1\\
\end{array}\right].\]
Here, the coordinates are indexed by $0000,$ $0001,$ $0010,$ $0011,$ etc. This example, with $t=1$, contains 32 distinct codewords, though the size of the code increases rapidly with $t$. In fact, $\mathcal{R}(2,4)$ contains $2048$ distinct codewords. Unfortunately, it is not only the number of codewords we typically care about. Another main parameter we are interested in is the \emph{minimum distance} --- the smallest number of entries in which unequal codewords may differ. It is in this parameter that we pay for the extra codewords in the higher order Reed Muller code; the minimum distance of $\mathcal{R}(1,4)$ is 8, while $\mathcal{R}(2,4)$ has a minimum distance of only half that. Given the large minimum distance of $\mathcal{R}(1,4)$ and the large size of $\mathcal{R}(2,4)$, the question arises: what is the largest subcode of $\mathcal{R}(2,4)$ such that the minimum distance is six? One may show that any generator matrix cannot have more than seven rows and thus we will not find any linear subcode with more than 128 codewords. However, we may do better than this if we do not require linearity. Thus, we will instead define a code explicitly by providing a polynomial for each and every codeword. First, consider the eight quadratic polynomials
\[\begin{aligned}p_1 &=x_1x_2+x_1x_3+x_1x_4+x_2x_3+x_2x_4+x_3x_4,\\
p_2 &= x_1x_2+x_2x_3+x_3x_4,\\p_3 &=x_1x_2+x_2x_4+x_4x_3,\\p_4 &=x_1x_3+x_3x_2+x_2x_4,\\p_5 &=x_1x_3+x_3x_4+x_4x_2,\\p_6 &=x_1x_4+x_4x_2+x_2x_3,\\p_7 &=x_1x_4+x_4x_3+x_3x_2,\\p_8 &=0.\end{aligned}\]
Each of these determines a coset of $\mathcal{R}(1,4)$ inside $\mathcal{R}(2,4)$ by adding the resultant codeword to each of the words in $\mathcal{R}(1,4)$; for example, the coset corresponding to $p_8$ is $\mathcal{R}(1,4)$ itself. The union of these cosets gives us a code with 256 distinct words with minimum distance six. This code is known as the (extended) Nordstrom-Robinson code, the first in an infinite family of non-linear codes which may be defined similarly by taking cosets of the first order Reed Muller codes inside the respective second order Reed-Muller code.

It turns out this code has an interesting history behind it. The code, originally given in \cite{Nordstrom1967}, was found by a high-school student after attending an introductory talk at his school. John Robinson, a professor at the University of Iowa at the time, gave the talk in the mid 1960's in which he discussed both linear and non-linear binary codes. After introducing the best possible linear code of length 15 and minimum distance 5 (the double-error-correcting BCH code), Robinson pointed out that the upper bound on non-linear codes with the same length and minimum distance was a factor of 2 greater --- yet no such code was known.  Alan Nordstrom responded to the challenge and, through trial and error, was able to produce what is now known as the Nordstrom-Robinson code. This code attracted attention quickly and within a few years it was discovered that the extended version (as described above) may be generalized to two infinite families of non-linear codes, first the Preparata codes in 1968 \cite{Preparata1968} and four years later the Kerdock codes \cite{Kerdock1972}.

Perhaps one of the most intriguing questions arising from these families at the time was the notion that they were formally dual --- despite the notion of ``duality" being a property of linear codes. Recall that we define the dual of a linear code as the null-space of the generator matrix --- that is, the dual code consists of all codewords which are orthogonal to every codeword of the original code. Using the MacWilliams identity \cite{MacWilliams1963}, this notion was generalized to a statement about the parameters of codes. For a binary code $C$ of length $n$, MacWilliams defined the \emph{weight distribution} as the sequence of numbers $A_t = \left\vert\left\{c\in C\mid w(c)=t\right\}\right\vert$ where $w(c)$ counts the number of non-zero entries of codeword $c$. Then the \emph{weight enumerator polynomial} is given by
\[W(C;x,y) = \sum_{t=0}^n A_t x^ty^{n-t}.\]
MacWilliams showed that any pair of dual codes $C$ and $C^\perp$ must satisfy the identity
\begin{equation}W(C^\perp;x,y) = \frac{1}{\vert C\vert}W(C;y-x,y+x).\label{macwill}\end{equation}
Thus MacWilliams defined the notion of formal duality; we say two codes are \emph{formally dual} if they satisfy Equation \eqref{macwill}. For linear codes, this allows us to show certain linear codes do not exist --- that is, given $A_t$, if any coefficient of $\frac{1}{\vert C\vert}W(C;y-x,y+x)$ is not a non-negative integer, we have a quick proof that $C^\perp$ does not exist based solely on its purported weight enumerator. The converse however is not true; in fact, we may find a non-linear code $C$ for which no formally dual code exists, while the right hand side of Equation \eqref{macwill} has only non-negative integer coefficients. Therefore it is important to emphasize that the notion of formal duality is a statement of the parameters of a code, not the code itself. In fact, a linear code may have many codes formally dual to it, despite always having a unique dual. As an example in the non-linear case, the weight enumerator of the Nordstrom-Robinson code is
\[y^{16}+112x^6y^{10}+30x^8y^{8}+112x^{10}y^6+x^{16}.\]
One may check that the RHS of Equation \eqref{macwill} results in the same polynomial; we therefore say that the Nordstrom-Robinson code is \emph{formally self-dual}. More generally, one finds that the Kerdock and Preparata codes are formally dual codes. However, since this duality is based solely on the parameters, it does not provide a way to construct one family from the other. It was not until two decades later in 1994 that Hammons, Kumar, Calderbank, Sloane, and Sole \cite{Hammons1994} showed that certain codes with these parameters are the images of submodules of $\bbZ_4^n$ under the Gray map --- that is, they are linear when viewed as codes of length 8 with an alphabet of size 4. This was illuminated further by Calderbank, Cameron, Kantor, and Seidel \cite{Calderbank1997} who gave a geometric path from the binary Kerdock codes to $\bbZ_4$-Kerdock codes. Thus, while there cannot exist linear binary codes with these parameters, one may two construct $\bbZ_4$-submodules of $\bbZ_4^n$ corresponding to each parameter set which are dual in the traditional sense.

Outside the question of duality, the Kerdock codes have many other, quite fascinating, connections. In the early 1970s, Cameron \cite{Cameron1972} introduced a type of multipartite graph called a \emph{linked system of symmetric designs} (``LSSDs", refer to Chapter \ref{3class} for more detail). Around that time, Goethals communicated to Cameron that one may build examples of such objects using the Kerdock codes; these examples where shown to be optimal with respect to the number of fibers \cite{Noda1974}. This family of LSSDs became known as the Cameron-Seidel association scheme (see Section \ref{kerdock}), remaining the archetypal example of LSSDs even to this day. A second (though not completely independent) use of Kerdock codes is in the construction of real mutually unbiased bases. Here, we look for orthonormal bases in $\bbR^m$ where vectors from distinct bases have an inner product of $\pm\frac{1}{\sqrt{m}}$. With connections to quantum cryptography and Euclidean geometry, mutually unbiased bases have been an area of interest for quite some time now. It was shown using quadratic forms \cite{Cameron1973} that the Kerdock sets not only gave examples of real MUBs, but that these examples were optimal with respect to the number of bases \cite{Calderbank1997} --- this is the only known infinite family of real MUBs achieving this upper bound.

A similar problem is that of finding lines in $\bbR^m$ in which any pair of lines intersect in a fixed angle; such sets of lines are called ``equiangular lines". Gerzon showed that the upper bound on the number of lines in $\bbR^m$ is given by $\frac{m(m+1)}{2}$ \cite{Lemmens1973}, yet the known constructions all scaled linearly with the dimension. It was not until nearly 30 years later that de Caen \cite{deCaen2000} used the Cameron-Seidel scheme to build $\frac{2}{9}\left(d+1\right)^2$ real equiangular lines whenever $d = 3\left(2^{2t-1}\right)-1$ for some positive integer $t$, resulting in the first known infinite family of size quadratic in the dimension.

We therefore find the Kerdock codes, the first example of which was discovered by a high school student, have deep connections to many other areas of study including design theory, quantum cryptography, and equiangular lines; objects such as these clearly warrant further study. Central to many of these connections is the fact that the Cameron-Seidel scheme --- the graphs given the distinct distances in any Kerdock code --- forms a 3-class association scheme. A \textit{symmetric $d$-class association scheme} (see Chapter \ref{association} for a more thorough definition) may be thought of as a edge-coloring of the complete graph using $d$ colors such that: given any colors $c_1$, $c_2$, and $c_3$, the number of $c_1,c_2,c_3$ triangles containing a fixed $c_1$-edge depend only on the colors chosen, not on the edge. We also include the ``0-color" as the graph of loops where we define a ``triangle" containing a loop as a loop paired with any incident edge. Using the Kerdock codes as an example, we color the edges by distance between codewords and this tells us that any pair of words at distance $i$ have a constant number of words distance $j$ from one and $k$ from the other, independent of the pair chosen.

Within the study of association schemes, we will often find rich connections to other areas of mathematics. In this thesis we will examine a type of association schemes known as ``cometric" (see Section \ref{poly} for the definition); this class includes many of the objects mentioned already. Within the field of association schemes, we find many parameters which describe the structure of an association scheme (see Chapter \ref{association}) analogous to the weight distribution of a binary code. Using these parameters, we arrive at a notion of formal duality of association schemes; two association schemes are formally dual if the first and second eigenmatrices of one are swapped for the other. Just as formal duality for general codes arose from explicit duality of linear codes, formal duality in association schemes arises from character duality of abelian groups. Given an abelian group $G$, the dual group $G^*$ is given by taking the set of characters of $G$ where for $\chi,\psi\in G^*$ and $x\in G$, $(\chi*\psi) (x) = \chi(x)*\psi(x)$. If there exists an abelian group acting sharply transitively on the points of an association scheme, then the dual scheme is well-defined. However, without this added structure, there is no clear way to build the ``dual" of a general association scheme. Despite this, we may define duality formally, at the parameter level, and find concrete examples of formally dual pairs of association schemes without a clear way of constructing one from the other. We finish this introduction with a brief history of association schemes followed by an outline of this thesis.

First introduced by Bose and Nair \cite{Bose1939} in 1939 with connections to certain block designs, the algebraic structure known as an ``association scheme" was formally defined later in 1952 by Bose and Shimamoto \cite{Bose1952} as a set of relations on a point set satisfying certain strong regularity properties (see Chapter \ref{association}). It was not until seven years later that Bose and Mesner \cite{Bose1959} described the equivalence between association schemes and Schur-closed matrix algebras --- commutative vector spaces of matrices closed under two distinct matrix products. Around the same time Wielandt was expanding on the theory of Schur (\cite{Wielandt1964},\cite{Schur1933}) to understand the commuting algebra, or centralizer ring, of permutation groups. These two concepts were generalized together by Higman in 1967 \cite{Higman1967} who discussed so-called ``coherent configurations". Shortly thereafter Biggs introduced a generalization of distance-transitive graphs known as ``distance regular graphs" \cite{Biggs1971}, showing that the adjacency matrix of any such graph generates the matrix algebra of an association scheme with very particular ``polynomial" properties. Over the next few years, Biggs continued to develop the notion of distance-regular graphs and their relationship with association schemes, culminating in parts of his book Algebraic Graph Theory \cite{Biggs1974}. Arguably one of the most influential works on this topic is the thesis of Delsarte in 1973 \cite{Delsarte1973}, developed seemingly independently of Biggs. In this thesis, he lays out the definitions and parameters central to association schemes, discusses subsets of associations schemes, and defines both $P$-polynomial (metric) and $Q$-polynomial (cometric) association schemes; the former are equivalent to Biggs' notion of distance regular graphs while the latter remained largely unexplored until decades later. These cometric examples will be the main focus of this thesis. Delsarte devoted particular attention to the Johnson and Hamming schemes, defining more clearly the notion of duality within these two schemes and bringing to the forefront the connections between association schemes and coding theory.

The two decades that followed brought with them many new results concerning polynomial association schemes, especially those that are $P$-polynomial. Authors such as Biggs, Damerell, Gardiner, Meredith, and Smith (see \cite{Brouwer1989} for a list of their relevant publications) continued to develop our understanding of distance-regular and distance-transitive graphs. Meanwhile authors such as Terwilliger \cite{Terwilliger1986} and Neumaier \cite{Neumaier1985} focused on specific families, developing parameter characterizations of Johnson and Hamming schemes --- the major examples of metric association schemes. Terwilliger then went on to work towards classifying association schemes which are both $P$- and $Q$-polynomial in papers such as \cite{Terwilliger1987} and \cite{Terwilliger1988}. Much of what is known has been compiled into books, first by Bannai and Ito in \cite{Bannai1984}, then by Brouwer, Cohen, and Neumaier in \cite{Brouwer1989}, and most recently by Bailey in \cite{Bailey2004}.

Despite the great attention devoted to $P$-polynomial schemes, it seems not much progress was made in understanding their $Q$-polynomial analogues until Dickie's thesis \cite{Dickie1995} in 1995 and two papers of Suzuki (\cite{Suzuki1998},\cite{Suzuki1998-2}) three years later. In the latter two papers, Suzuki showed, apart from cycles, any $Q$-polynomial association scheme may have at most two $Q$-polynomial orderings and any imprimitive $Q$-polynomial association scheme must be either $Q$-bipartite or $Q$-antipodal (except possibly for two sporadic cases which were later ruled out). These results were analogues of results for distance regular graphs dating as much as three decades prior, yet the method for proving these results was quite different. These papers triggered a resurgence of interest in cometric association schemes as the following two decades brought many new results. Some results included finding equivalences between certain classes of cometric association schemes and other geometric structures; for instance \cite{VanDam1999} discusses $3$-class $Q$-antipodal schemes while \cite{LeCompte2010} focuses on $4$-class schemes which are both $Q$-antipodal and $Q$-bipartite. New examples were found, including families discovered by Penttila and Williford \cite{Penttila2011}, another family found by Moorhouse and Williford \cite{Moorhouse2016}, and many new sporadic examples found by Gavin King \cite{King2018}. While far from an exhaustive list, this author would be remiss without also mentioning papers of Suda (\cite{Suda2011},\cite{Suda2012}), van Dam, Martin and Muzychuk \cite{vanDam2013}, Martin and Williford \cite{Williford2009}, and Martin, Muzychuk and Williford \cite{Martin2007}.

In this thesis, we begin by defining association schemes and their associated Bose-Mesner algebras. The remainder of Chapter \ref{association} consists of the various definitions which occur within this field such as the parameters, feasibility conditions, substructures, and polynomial structures --- some of which are mentioned above. We then focus on the matrix algebra in Chapter \ref{psdcone}, where we examine the cone of positive semidefinite matrices. Here, we introduce methods to build other line systems, such as equiangular lines, as well as develop a new feasibility condition on association schemes using a theorem of Sch\"{o}nberg --- we explicitly calculate many of these new conditions for cometric schemes, in particular. In Chapter \ref{3class}, we recall the definition of linked systems of symmetric designs (LSSDs) defined by Cameron \cite{Cameron1972} and explore constructions of these as well as connections to other objects. We review past results on such objects including the Cameron-Seidel scheme mentioned previously, results of Noda on the maximum size of any such object, and their equivalence with 3-class $Q$-antipodal association schemes, a result of van Dam. We then introduce a new geometric object called a ``set of linked simplices" and we show that these are equivalent to LSSDs. Using this new equivalence, we investigate when we may build real mutually unbiased bases from these association schemes as well as construct new examples using parameters distinct from those of the Cameron-Seidel scheme. Chapter 5 focuses on $4$-class $Q$-bipartite association schemes, beginning with a motivational discussion indicating how these association schemes naturally occur. While we do not develop an equivalence in this chapter between the association schemes and so-called ``orthogonal projective double covers", we make substantial progress towards that goal. In the final chapter, Chapter 6, we investigate the connectivity of relations in association schemes in general. With the goal of understanding the ``nearest-neighbor graph" of a cometric association scheme, we show that any connected graph (in the absence of twin vertices --- vertices with the same neighborhood) within an association scheme remains connected after deleting the entire neighborhood of a vertex (including the vertex itself). We use this result to show that any graph within an association scheme with connectivity equal to 2 must be a cycle. In the appendix, we finish be surveying the known infinite families of symmetric designs in order to determine which families could yield the parameter sets of LSSDs with more than two fibers.
