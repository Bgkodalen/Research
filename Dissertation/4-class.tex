\chapter{4-class $Q$-bipartite: Double covers of SRGs}
\label{4classbip}
\begin{restatable*}{thm}{fourclasssixzero}\label{thm60}
	Suppose we have a feasible parameter set for a $4$-class association scheme which is $Q$-bipartite but not $Q$-antipodal. Let $k=P_{01}$, $r=P_{21}$, and $s=P_{41}$ where $P$ is the first eigenmatrix using the natural ordering. Then the scheme is realizable only if $s=-n^2$ for some integer $n>1$ and
	\[15n^4(2n^2-3)r^2 + (n^6-45kn^2+76k)n^2r+k(16k+n^6)(n^2-2)\geq 0.\]
\end{restatable*}

In this chapter we investigate the specific case of $4$-class $Q$-bipartite schemes. LeCompte et al.\ \cite{LeCompte2010} proved that real mutually unbiased bases are equivalent to 4-class $Q$-bipartite schemes which are also $Q$-antipodal. This extra system of imprimitivity arises as one of the quotient graphs is a union of cliques, the other being complete multipartite. Here, we will instead restrict ourselves to the case where our association schemes are not $Q$-antipodal thus assuming that both quotient graphs are connected. We will begin by showing that the parameters of any such scheme are determined completely by three integral parameters and then recast Theorem \ref{cometricbnds} in terms of these three parameters. Let $(X,\mathcal{R})$ be a 4-class $Q$-bipartite association scheme with $Q$-polynomial ordering $E_0,E_1,\dots,E_4$ and natural ordering $A_0,A_1,\dots A_4$. We know from Theorem \ref{suzuki} that the quotient of $(X,\mathcal{R})$ has exactly two non-trivial relations and thus must be strongly regular. Let $(v,k,\lambda,\mu)$ be the parameters of the quotient strongly regular graph which contains $R_1$ as a subgraph. Let $k>r>s$ be the eigenvalues of this SRG with corresponding multiplicities $1$, $f$, and $g$. Since $(X,\mathcal{R})$ is not $Q$-antipodal, we must have $k>r$ and $s>-k$. The $Q$ matrix of this SRG will be
\[\tilde{Q} = \left[\begin{array}{ccc}
1 & f & g\\
1 & \frac{fr}{k} & \frac{gs}{k}\\
1 & \frac{f(1+r)}{k+1-v} & \frac{g(1+s)}{k+1-v}
\end{array}\right].\]
We may use this information to build the first and second eigenmatrices of our $4$-class $Q$-bipartite scheme as follows.
\begin{thm}
	\label{Pmat}
	Let $(X,\mathcal{R})$ be a 4-class $Q$-bipartite association scheme with relations ordered naturally. Let the quotient SRG have $v$ vertices and spectrum $k^1,r^f,s^g$ with $k>r>s$. Then the first and second eigenmatrices are as follows:
	\[P = \left[\begin{array}{crcrr}
	1 & k & 2(v-1-k) & k & 1\\
	1 & \frac{k}{n} & 0 & -\frac{k}{n} & -1\\
	1 & r& -2(1+r) & r & 1\\
	1 & -n & 0 & n & -1\\
	1 & s & -2(s+1) & s & 1\\
	\end{array}\right]\qquad Q = \left[\begin{array}{crcrc}
	1 & m & f & \frac{mk}{n^2} & g\\
	1 & \frac{m}{n} & \frac{fr}{k}  & -\frac{m}{n} & \frac{gs}{k}\\
	1 & 0 & \frac{f(r+1)}{k+1-v}  & 0& \frac{g(1+s)}{k+1-v}\\
	1 & -\frac{m}{n} & \frac{fr}{k} & \frac{m}{n} & \frac{gs}{k}\\
	1 & -m & f & \frac{mk}{n^2} & g\\
	\end{array}\right]\]
	where $s = -n^2$.
\end{thm}
\begin{proof}
	We begin by building all of $Q$ and then employ the use of our orthogonality properties. Note that column 0 of $Q$ comes by definition. From Theorem [\cite{Brouwer2003},\cite{Martin2007}], $Q_{1,1} = -Q_{3,1}\neq 0 = Q_{2,1}$, so we define $n = \frac{m}{Q_{1,1}} = -\frac{m}{Q_{3,1}}$ and column 1 is given. The first three entries of columns $2$ and $4$ follow from the parameters of our quotient scheme while the remaining two entries of each columns follow from Corollary \ref{evenpoly}. Finally column 3 may be found using the first orthogonality condition (specifically that $\displaystyle{\sum_j Q_{ij} = \vert X\vert\delta_{i0}}$). From here we have that 
	\[Q = \left[\begin{array}{crccc}
	1 & m & f & v-m & g\\
	1 & \frac{m}{n} & \frac{fr}{k}  & -\frac{m}{n} & \frac{gs}{k}\\
	1 & 0 & \frac{f(r+1)}{k+1-v}  & 0& \frac{g(1+s)}{k+1-v}\\
	1 & -\frac{m}{n} & \frac{fr}{k} & \frac{m}{n} & \frac{gs}{k}\\
	1 & -m & f & m-v & g\\
	\end{array}\right],\]
	matching our theorem in all but two places.	Since we have ordered the relations using the natural ordering, the valencies of our relations are given by $[1,k,2(v-1-k),k,1]$. This allows us to derive an expression for $q_{01}^1$ using \cite[Theorem.~2.3.2.]{Brouwer1989} which gives
	\[q_{ij}^k = \frac{1}{\vert X\vert m_k}\sum_{l=0}^d\left(v_lQ_{li}Q_{lj}Q_{lk}\right)\]
	where $m_k$ and $v_l$ are the multiplicities and valencies of the $k^\text{th}$ and $l^\text{th}$ relations respectively. We find that $q_{01}^1 = \frac{1}{2vm}\left(2m^2+\frac{2km^2}{n^2}\right)$, however we know from Theorem \ref{kreinidentities} that $q_{01}^1=1$, resulting in $\frac{km}{n^2} = v-m$. This completes our proof for the second eigenmatrix and we may use the second orthogonality condition to find $P$ noting that the first row of $P$ is the valencies of our relations. Thus
	\[P = \left[\begin{array}{crcrr}
	1 & k & 2(v-1-k) & k & 1\\
	1 & \frac{k}{n} & 0 & -\frac{k}{n} & -1\\
	1 & r& -2(1+r) & r & 1\\
	1 & -n & 0 & n & -1\\
	1 & s & -2(s+1) & s & 1\\
	\end{array}\right].\]
	We again use our equation for Krein parameters one more time to find $q_{11}^4 = \frac{mg(n^2+s)}{n^2v}$. Since $q_{11}^4=0$ due to our cometric property, we have that $s = -n^2$.
\end{proof}
\begin{cor}
	The parameters of a 4-class $Q$-bipartite scheme are uniquely determined by the eigenvalues of the quotient SRG.
\end{cor}
\begin{proof}
	Our first eigenmatrix only requires $v,k,r,s,$ and $n$. However since $n>0$ (from the natural ordering of relations), $n = \sqrt{-s}$. Further \cite{Brouwer1989} states that  $v = \frac{(k-r)(k-s)}{k+rs}$. 
\end{proof}
Before moving to examine the effect of Sch\"{o}nberg's theorem on 4-class $Q$-bipartite schemes, we mention a few parameter bounds arising from the feasibility conditions FC1-FC3 and show how they restrict the space of feasible parameters.
\begin{thm}
	\label{bounds}
	Suppose we have a feasible parameter set for a $4$-class association scheme which is $Q$-bipartite but not $Q$-antipodal. Let $k=P_{01}$, $r=P_{21}$, and $s=P_{41}$ where $P$ is the first eigenmatrix using the natural ordering. The following must hold with $n:=\sqrt{-s}$ and $\mu=k+rs$:
	\begin{enumerate}[label=(\roman*)]
		\item $\mu\geq n(r+n)$,
		\item $n\vert \mu$ and $n\vert k$,
		%\item $r\leq \frac{k-n^2}{n(n+1)}$,
		\item $r\geq \frac{2k}{3n^2}-\frac{n^2}{3}$,
		\item $kn^2(n^2-1)\geq \mu(n^2+r)$
	\end{enumerate}
	Further, $n$ is an integer greater than 1.
\end{thm}
\begin{proof}
	First note that $k$, $r$, and $s$ are the eigenvalues of a strongly regular graph and thus integral (we assume here that the SRG is not a conference graph). For $(i)$ and $(ii)$, note that	$p_{13}^1 = \frac{(n-1)(\mu-n(r+n))}{2n}$. FC2 tells us that this must be a non-negative integer, and therefore we must either have $-s = n = 1$ or $\mu-n(r+n)\geq0$. As $s=-1$ implies our SRG is a union of cliques (and thus $(X,\mathcal{R})$ is $Q$-antipodal), we may ignore this case and $(i)$ follows. Since $\gcd(n,n-1)=1$, we have that $n\vert (\mu-n(r+n))$ forcing $n\vert \mu$ and since $k=\mu+rn^2$, $(ii)$ follows. Next, $(iii)$ follows from the absolute bound $1+f \leq \frac{m(m+1)}{2}$ giving us $n^4+3n^2r-2k\geq 0$. Using another absolute bound, $(iv)$ follows from $\frac{v}{m}\leq f$. Finally, since $n = \sqrt{-s}$, if $n$ is not an integer, then columns one and three of $Q$ must be irrational. However Galois conjugation is an automorphism of our Bose-Mesner algebra and thus $E_0$, $E_3$, $E_2$, $E_1$, $E_4$ must be a second $Q$ polynomial ordering in this case, implying $q_{3,3}^4=0$. Using our $P$ and $Q$ matrices, we find that $q_{3,3}^4 = \frac{(k-r)(k+s)}{\mu}$. This means that whenever $n$ is irrational, either $r=k$ or $s = -k$, both of which imply $(X,\mathcal{R})$ is $Q$-antipodal.
\end{proof}

\begin{cor}
	\label{kbnds}
	Suppose we have a feasible parameter set for a $4$-class association scheme which is $Q$-bipartite but not $Q$-antipodal. Let $k=P_{01}$, $r=P_{21}$, and $s=P_{41}$ where $P$ is the first eigenmatrix using the natural ordering. Then
	\[\frac{k}{n^2}-1\leq \frac{(n+1)}{2}\left((n+1)(n^3-n-1)+\sqrt{(n-1)(n^7+3n^6+2n^5-4n^4-9n^3-3n^2+3n-1)}\right).\]
\end{cor}
\begin{proof}
	Using Theorem \ref{bounds}$(i)$ and $(iv)$, we have that $n(r+n)\leq \mu\leq \frac{kn^2(n^2-1)}{n^2+r}$. Using $\mu = k-rn^2$, these two inequalities give us
	\[\frac{k-n^4+\sqrt{n^8-2n^4k(2n^2+3)+k^2}}{2n^2}\leq r\leq \frac{k-n^2}{n(n+1)}.\]
	This implies that 
	\[k^2-n^2(n^5+2n^4-3n^2-3n+1)k+n^5(n^2+n-1)\leq0.\]
	When $k=1$ and $n>1$, the left hand side will be negative. Therefore this requires that $k$ is less than the positive root of this quadratic, giving us our bound.
\end{proof}
We now examine the bounds arising from Corollary \ref{Qbipbnds} as applied to our 4-class $Q$-bipartite association scheme. We begin by noting that $\theta_{31}\geq 0$ becomes Theorem \ref{bounds} $(i)$ when we use the parameters $k$, $r$, and $n$, thus making it equivalent to an absolute bound in this context. Next, we find that plugging in our parameters gives $\theta_{42}\geq 0$ and $\theta_{53}\geq 0$ if and only if $k\geq \frac{-rn^2}{n^2-2}$ and $k\geq -\frac{(3n^2-7)rn^2}{n^4-3n^2+6}$ respectively. Both of these bounds are vacuous since the right hand side will be negative for any choice of $n>1$. Finally one may show that $\theta_{31}\geq 0$ and $\theta_{60}\geq 0$ together imply $\theta_{51}\geq 0$ in the specific case of a 4-class $Q$-bipartite scheme. Therefore the only new restriction, not implied by FC1-FC3 is $\theta_{60}\geq 0$, resulting in the following theorem.
\fourclasssixzero
\begin{comment}\begin{thm}\label{thm60}
Suppose we have a feasible parameter set for a $4$-class association scheme which is $Q$-bipartite but not $Q$-antipodal. Let $k=P_{01}$, $r=P_{21}$, and $s=P_{41}$ where $P$ is the first eigenmatrix using the natural ordering. Then the scheme is realizable only if
\[15n^4(2n^2-3)r^2 + (n^6-45kn^2+76k)n^2r+k(16k+n^6)(n^2-2)\geq 0.\]
\end{thm}
\end{comment}
\begin{proof}
	Apply the parameters $k,r,$ and $s$ to Theorem \ref{Qbipbnds} $(v)$.
\end{proof}
We may pair this Theorem with Theorem \ref{bounds} to get the following corollary.
\begin{cor}\label{newkbnds}
	Suppose we have a feasible parameter set for a $4$-class association scheme which is $Q$-bipartite but not $Q$-antipodal. Let $k=P_{01}$, $r=P_{21}$, and $s=P_{41}$ where $P$ is the first eigenmatrix using the natural ordering. The following table gives an upper bound on the largest eigenvalue $k$ based on the smallest $s$ for $-4\leq s\leq 121$:
	\[\begin{tabular}{c|c|c|c|c|c|c|c|c|c|c}
	$n$ & 2 & 3 & 4 & 5 & 6 & 7 & 8 & 9 & 10 & 11\\\hline
	$k\leq$  & 56 & 891 & 5504 & 22297 & 85128 & 282828 & 867787 & 2609805 & 8468529 & 40926495\\
	\end{tabular}\]
\end{cor}
\begin{proof}
	Let $r_1\geq r_2$ be the two roots of $15n^4(2n^2-3)r^2 + (n^6-45kn^2+76k)n^2r+k(16k+n^6)(n^2-2)$. Then Theorem \ref{thm60} tells us that either $r\geq r_1$ or $r\leq r_2$. Pairing this with Theorem $\ref{bounds}$ we find that $r\geq r_1$ and $\mu\geq n(r+n)$ together restrict $k$ via
	\[\begin{aligned}	\frac{k}{n^3(n^2-1)}&\leq \frac{n^7+2n^6-3n^4-17n^3+45n^2+14n-76}{-2(n^4-13n^3+15n^2+12n-32)(n^2-1)}\\
	&\qquad+\frac{\sqrt{n^{10}+4n^9+6n^8+2n^7-35n^6+22n^5+145n^4-72n^2+32n+16}}{-2(n^4-13n^3+15n^2+12n-32)}.\end{aligned}\]
	Secondly, $r\leq r_2$ with $r\geq\frac{2k}{3n^2}-\frac{n^2}{3}$ implies that $k\leq \frac{3n^6-5n^4}{2}$. Taking the maximum of these two bounds for each $2\leq n\leq 11$ results in the values given in the table. In each case apart from $n=11$, this is a reduction from the bound given in Theorem \ref{kbnds}
\end{proof}
We conclude this chapter by noting the impact of Theorem \ref{thm60} on the feasible parameter space of 4-class $Q$-bipartite association schemes. In the table below we list the number of feasible schemes for a given $n>0$ when only considering conditions FC1, FC2, and FC3. We also list the number of feasible schemes when we include Theorem \ref{thm60} as a feasibility condition.
\[\begin{tabular}{c|c|c}\label{feasible4class}
$n$ & \# of feasible parameter sets & \# of feasible parameter sets satisfying Theorem \ref{thm60}\\\hline
2 & 6 & 5\\
3 & 60 & 44\\
4 & 223 & 140\\
5 & 473 & 334\\
6 & 1015& 701\\
7 & 1256& 952\\
8 & 2256& 1659\\
\end{tabular}\]
The following figures display the original feasibility conditions and the new bound due to $\theta_{60}\geq 0$. We display the graphs for $n=7$, noting that similar graphs may be generated for any $n>1$.
\begin{figure}[h]
	\begin{subfigure}[h]{0.5\textwidth}
		\includegraphics[scale=.5]{bounds7py.png}
	\end{subfigure}
	\begin{subfigure}[h]{0.5\textwidth}
		\includegraphics[scale=.5]{geg7.png}
	\end{subfigure}
	\caption[4-class $Q$-bipartite bounds]{These figures pertain to the case $n = 7$. On the left we have two absolute bounds and a bound due to the non-negativity of an intersection number. In green, we have plotted every parameter set which is feasible under FC1-FC3. On the right, we have replaced the bounds with the bound $\theta_{60}\geq 0$. Any parameter set contained within the parabola is not realizable.}
\end{figure}