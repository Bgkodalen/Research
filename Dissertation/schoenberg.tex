\chapter{Sch\"{o}nberg's Theorem}\label{schoenberg}
In Chapter \ref{association}, we introduced a set of orthogonal polynomials $q_0(t),\dots,q_d(t)$ with the property that $q_i\circ\left(\frac{1}{\vert X\vert} E_1\right) = \frac{1}{\vert X\vert} E_i$ for any $Q$-polynomial association scheme $(X,\mathcal{R})$ with $Q$-polynomial ordering $E_0,E_1,\dots,E_d$. We observed that a polynomial $f(t) = \sum_{i=0}^d c_i q_i(t)$ will produce a matrix with eigenvalues $c_0,\dots,c_d$ when applied to $E_1$ entrywise. In this chapter, we introduce a second set of orthogonal polynomials, known as the Gegenbauer polynomials, which have been shown to preserve the positive semi-definite property. Using these, we examine the relation of these polynomials with the orthogonal polynomials previously mentioned and compare the Gegenbauer cone with the positive semi-definite cone seen in Chapter \ref{psdcone}. This comparison allows us to derive new bounds on the krein parameters of association schemes.\\
Let $m$ be a fixed positive integer and $X\subset S^{m-1}$ be a finite set of distinct unit vectors. Let $G_X$ denote the Gram matrix of $X$, then $G_X$ is positive semi-definite ($G_X\succeq 0$). A function $f:[-1,1]\rightarrow\mathbb{R}$ is \textit{positive definite} if, for every such finite subset $X$, $f\circ(G_X)\succeq 0$. In 1942, Sch\"{o}nberg proved that a continuous function $f:[-1,1]\rightarrow \mathbb{R}$ is positive definite if and only if $f$ is expressible as a non-negative linear combination of the Gegenbauer polynomials. These Gegenbauer polynomials may be defined using the three-term recurrence:
\[\begin{aligned}
Q_k(t) &= \frac{(2k+m-4)tQ_{k-1}(t) - (k-1)Q_{k-2}(t)}{k+m_3}, k\geq 2\\
Q_0(t) &= 1\qquad Q_1(t) = t.
\end{aligned}\]
Here we have normalized the Gegenbauer polynomials so that $Q_i(1) = 1$. Below we give the first six Gegenbauer polynomials:
\[\begin{aligned}
&Q_0^m(t)=1\qquad
Q_1^m(t)=t\qquad
Q_2^m(t)=\frac{mt^2 - 1}{m-1}\qquad
Q_3^m(t)=\frac{(m+2)t^3 - 3t}{m-1}\qquad\\
Q_4^m(t)=&\frac{(m+4)(m+2)t^4 - 6(m+2)t^2+3}{m^2-1}\qquad
Q_5^m(t)=\frac{(m+6)(m+4)t^5-10(m+4)t^3+15t}{m^2-1}\\
\end{aligned}\]
\begin{figure}[h]
	\begin{center}
	\includegraphics[scale=.4]{gegenbauer_polynomials.PNG}
	\caption{Gegenbauer polynomials with degree 1 through degree 5 with $m=10$.}\label{gegpic}
\end{center}


\end{figure}