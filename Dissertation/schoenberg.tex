\chapter{Polynomial ring of Association schemes}\label{schoenberg}
Chapter $\ref{psdcone}$ examined a particular set of matrices in the Bose-Mesner algebra of an association scheme which gave us Gram matrices of finite sets $X\subset \mathbb{S}^{m-1}$ for some fixed positive integer $m$. In particular, we found that the existence of any association scheme implies the existence of sets of vectors with very few possible inner products. While we may use this correlation to create new spherical $t$-distance sets anytime we have an association scheme, we may also use this implication to rule out possible parameter sets based on the properties of the resultant vector sets. For instance, Delsarte et.\ al.\ in \cite{Delsarte1975} found bounds on the size of any $t$-distance set based solely on the dimension of the ambient space and the inner products allowed between vectors. Later, in \cite{Delsarte1977}, they were also able to prove a $t$-distance set is a spherical design if and only if the first $s$ Gegenbauer polynomials summed over the inner products give zero as a result. This characterization of spherical designs allowed Suda \cite{Suda2011} to characterize when a the first idempotent in a $Q$-polynomial ordering would result in a spherical design. Many of these results are based around a family of single-variable polynomials known as the Gegenbauer polynomials, and a result of Sch\"{o}nberg \cite{Schoenberg1942} which implies that each Gegenbauer polynomial will result in a positive semi-definite matrix when applied to a Gram matrix elementwise.\par
In this chapter, we will define the Gegenbauer polynomials and review relevant definitions and theorems. We will then use a consequence of Sch\"{o}nberg's theorem to provide a general constraint on the minimal idempotents of association schemes. We then leverage properties of association schemes to derive parameter restrictions on a general association scheme, noting where these restrictions become non-trivial. We then examine these restrictions more closely in the setting of $Q$-polynomial association schemes where we give three non-trivial constraints. Finally, we will give examples of new parameter sets which are ruled out by these constraints and consider the case of $4$-class $Q$-bipartite association schemes where one constraint in particular greatly restricts the feasible parameter sets.\par
\section{Gegenbauer Polynomials}
In this section we will use polynomial rings and quotient rings, thus we begin this section with a brief review of the topic, referring to \cite{Gallian}(TODO: Bad citation here) for a more complete introduction. In all that follows let $m$ be a fixed positive integer and define $\scR:=\bbR[x_1,\dots,x_m]$. A \emph{monomial} is defined as a (possibly empty) product of the variables $x_1,\dots,x_m$ and given a monomial $t =\prod_{i=1}^{m}x_i^{d_i}$ $d_i\in \bbZ^+$, the \emph{degree} of $t$ is defined as $\text{deg}(t) = \sum_i d_i$. A polynomial $f\in\scR$ may be represented uniquely as a (finite) linear combination of distinct monomials $f =\sum_i\alpha_it_i$ and $\text{deg}(f) = \max\left\{\text{deg}(t_i)\right\}$. For each variable $x_j$, we define the \emph{derivative with respect to $x_j$} of a monomial as $\frac{\partial}{\partial x_j}t = d_jx_j^{d_j-1}\prod_{i\neq j}x_i^{d_i}$ and extend the definition linearly for any polynomial in $\scR$. Let $f\in \scR$ be given, $f$ is \emph{homogeneous} if there exists some constant $d\in\bbZ^+$ such that $\text{deg}(t)=d$ for every monomial $t$ in $f$. Further, $f$ is \emph{harmonic} if $\Delta f = \sum_i \frac{\partial}{\partial x_i}\left(\frac{\partial}{\partial x_i}\left(f\right)\right) = 0$.\par
Let $f\in\scR$ be given and define the ideal of $f$ as $(f) = \left\{gf:g\in\scR\right\}$. We then define equivalence classes $[g]_f = \left\{h\in\scR: g-h\in(f)\right\}$. The quotient ring $\nicefrac{\scR}{(f)}$ is given as the set of equivalence classes $\left\{[g]_f:g\in\scR\right\}$. We will often suppress the subscript if it is clear in the context. Let $S^{m-1}\subset \bbR^m$ be the $m-1$ dimensional sphere and we define the set of polynomials on the sphere as
\[\text{Pol}(S^{m-1}):= \nicefrac{\bbR[x_1,\dots,x_m]}{(1-\sum_ix_i^2)}.\]
We say a polynomial $f\in\bbR$ is \emph{harmonic on the sphere} if there exists a harmonic polynomial $g\in[f]$. Similarly, a polynomial is \emph{homogeneous on the sphere} if there exists a homogeneous polynomial $g\in[f]$. Finally, we say a polynomial $f\in\bbR$ is \emph{zonal} if there exists a vector $a\in \bbR$ and a single-variable polynomial $g(t)\in\bbR[t]$ such that $f(x) = g(\left<a,x\right>)$ for all $x\in S^{m-1}$. Note that since $h\in \text{Pol}(S^{m-1})$ implies $h(x) = 0$ for all $x\in S^{m-1}$, this requirement is independent of the representative chosen from the equivalence class.\par
We now introduce a particular set of polynomials arising from the context of spherical harmonics polynomials. The Gegenbauer polynomials in dimension $m$ are defined using the three-term recurrence:
\[\begin{aligned}
Q_k^m(t) &= \frac{(2k+m-4)tQ_{k-1}(t) - (k-1)Q_{k-2}(t)}{k+m-3} \qquad k\geq 2,\\
Q^m_0(t) &= 1\qquad Q^m_1(t) = t.
\end{aligned}\]
We use these initial polynomials so that $Q^m_i(1) = 1$ for all $i\geq 0$. We will suppress the superscript of $m$ anytime this is clear in the context. The first six Gegenbauer polynomials are as follows,
\[\begin{aligned}
&Q_0(t)=1,\qquad
Q_1(t)=t,\qquad
Q_2(t)=\frac{mt^2 - 1}{m-1},\qquad
Q_3(t)=\frac{(m+2)t^3 - 3t}{m-1},\qquad\\
Q_4(t)=&\frac{(m+4)(m+2)t^4 - 6(m+2)t^2+3}{m^2-1},\qquad
Q_5(t)=\frac{(m+6)(m+4)t^5-10(m+4)t^3+15t}{m^2-1}\\
\end{aligned}\]\newpage
\begin{figure}[!h]
	\begin{center}
		\includegraphics[scale=.4]{gegenbauer_polynomials.PNG}
		\caption[Gegenbauer polynomials]{Gegenbauer polynomials with degree 1 through degree 5 with $m=10$.}\label{gegpic}
	\end{center}
\end{figure}
\begin{figure}[!h]
	\begin{center}
		\includegraphics[scale=.4]{roots.PNG}
		\caption[Roots of Gegenbauer polynomials]{Roots of the five Gegenbauer polynomials plotted in Figure \ref{gegpic}.}\label{rootpic}
	\end{center}
\end{figure}
\begin{thm}
	For each $m,i\in\bbZ^+$ and $a\in\bbR^m$, the Gegenbauer polynomial $Q_i^m\left(\left<a,x\right>\right)$ is homogeneous on the sphere, harmonic on the sphere,
\end{thm}
\begin{proof}
	The zonal condition is satisfied trivially. The other two conditions follow from $F_k(x)\in\left[Q_i^m(\left<a,x\right>)\right]$ for 
	\[F_k(x)= \left<x,x\right>^{\lfloor\frac{k+1}{2}\rfloor}Q_k\left(\frac{\left<a,x\right>}{\left<x,x\right>}\right).\]
\end{proof}
We note that, up to scaling and rotation of the sphere, $F_k(x)$ as defined above is the unique degree $k$ polynomial with all three of these properties. We now consider a theorem of Sch\"{o}nberg's which uses Gegenbauer polynomials to characterize positive definite functions.\par
Let $X\subset S^{m-1}$ be a finite set of unit vectors. Let $G_X$ denote the Gram matrix of $X$, then $G_X$ is positive semi-definite ($G_X\succeq 0$). A function $f:[-1,1]\rightarrow\mathbb{R}$ is \textit{positive definite} if, for every such finite subset $X$, $f\circ(G_X)\succeq 0$.
\begin{thm}[Sch\"{o}nberg \cite{Schoenberg1942}]\label{schoen}
	Fix $m\in\mathbb{Z}^+$. A function $f:[-1,1]\rightarrow\mathbb{R}$ is positive definite if and only if $f(t) = \sum_{i} c_iQ_k^m(t)$ for non-negative constants $c_i$.\qed
\end{thm}
In particular, this means that $Q_k^m(t)$ is a positive definite function for any choice of $m$ and $k$. This leads to the following corollary,
\begin{cor}\label{schoen-as}
	Let $(X,\mathcal{R})$ be an association scheme with minimal idempotents $E_0,\dots,E_d$. Let $0\leq i\leq d$ be given and define $m_i:=\text{rank}\left(E_i\right)$. Then for any choice of $k>0$, there exists non-negative constants $c_j$ for $0\leq j\leq d$ so that
	\[Q_k^{m_i}\circ\left(\frac{\vert X\vert}{m_i}E_i\right) = \sum_j c_j E_j.\]
	Further $c_j$ is an eigenvalue of $Q_k^{m_i}\circ\left(\frac{\vert X\vert}{m_i}E_i\right)$ with multiplicity (at least) $m_j$ and is nonzero only if $E_j$ is contained in the subalgebra generated by $E_i$ using entrywise products.
\end{cor}
\begin{proof}
	Recall that the Bose-Mesner algebra given by $\BMA =\text{span}\left\{E_0,\dots,E_d\right\}$ is closed under entrywise products. Therefore we know that any entrywise polynomial of $E_i$ must be contained within the algebra. Therefore $Q_k^{m_i}\circ\left(\frac{\vert X\vert}{m_i}E_i\right)\in \BMA$ and we may write $Q_k^{m_i}\circ\left(\frac{\vert X\vert}{m_i}E_i\right) = \sum_j c_j E_j$ where each $c_j$ is the eigenvalue of the matrix in the eigenspace given by the idempotent $E_j$. Now, since  $E_i$ is an idempotent matrix with main diagonal given by $\frac{1}{\vert X\vert}Q_{11} = \frac{m_i}{\vert X\vert}$, we know that $\frac{\vert X\vert}{m_i}E_i$ is the Gram matrix of a set of unit vectors in $\mathbb{R}^{m_i}$. Therefore, Theorem \ref{schoenberg} tells us that $Q_k^{m_i}\circ\left(\frac{\vert X\vert}{m_i}E_i\right)$ must be positive semidefinite.
\end{proof}
Recall in Chapter \ref{association}, we found matrices $L_0^*,\dots,L_d^*$ with the homomorphism $\phi^*:\BMA\rightarrow\bbL = \left<L_0^*,\dots,L_d^*\right>$ which mapped entrywise products in $\BMA$ to matrix products in $\bbL$. Thus, for any polynomial $f\in\bbR[t]$ we have
\[\phi^*(f\circ(E_i)) = f(\phi^*(E_i)).\]
Using this homomorphism, we prove the following theorem.
\begin{thm}
	Let $(X,\mathcal{R})$ be an association scheme with krein parameters $\left\{q_{ij}^l\right\}_{0\leq i,j,l\leq d}$. For $0\leq i\leq d$, define $L_i = [q^l_{ij}]_{l,j}$. Let $0\leq i\leq d$ be given and define $m_i:=[L_0]_{i,i}$. Then for any choice of $k>0$, $Q_k^{m_i}\left(\frac{1}{m_i}L_i\right)$ must have only nonnegative entries.
\end{thm}
\begin{proof}
	Apply $\phi^*$ to both sides of the equation \[Q_k^{m_i}\circ\left(\frac{\vert X\vert}{m_i}E_i\right) = \sum_j c_j E_j\]
	noting that $q_{ij}^k\geq 0$ for all $0\leq i,j,k\leq d$ and thus each $L_i^*$ is nonnegative.
\end{proof}

\section{Cometric case}
In this section, we restrict to the case of cometric association schemes and determine the consequences of Corollary \ref{schoen-as}. Let $(X,\mathcal{R})$ be given and assume there exists a $Q$-polynomial ordering 



