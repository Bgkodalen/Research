\chapter{Association schemes}\label{association}
Association schemes arise in group theory, graph theory, design theory, coding theory and more. For example, if $X$ is a finite group with conjugacy classes $\cC[g] = \{hgh^{-1}:h\in X\}$ ($g\in X$), then the conjugacy class relations $R_g = \left\{ (a,b) \mid ab^{-1} \in \cC[g]  \right\}$ yield a 
commutative association scheme on the vertex set $X$. The orbits on $X\times X$ of any 
permutation group $G$ acting generously transitively on a set $X$  give a symmetric association scheme.
Some of the most well-studied association schemes are distance-regular graphs, including Moore graphs, distance-transitive graphs, strongly regular graphs, generalized polygons, etc. One studies 
$q$-ary error-correcting codes of length  $n$ as vertex subsets of the Hamming association scheme
$H(n,q)$ \cite[Sec.~9.2]{Brouwer1989} and one studies $t$-($v,k,\lambda$) designs as vertex subsets of the 
Johnson association scheme  $J(v,k)$ \cite[Sec.~9.1]{Brouwer1989}.  For an introduction to the 
extensive literature on the subject, the reader may consult \cite{Delsarte1973,Bannai1984,Brouwer1989,Godsil1993}, 
the survey \cite{Martin2009}, or the more recent book of  Bailey \cite{Bailey2005} which focuses on 
connections to the statistical design of experiments.

Let $X$ be a finite set of vertices. A \textit{symmetric d-class association scheme} (see \cite{Brouwer1989}) on $X$ is a pair $\mathcal{L} = (X,\mathcal{R})$ where $\mathcal{R} =\left\{R_0,R_1,\dots,R_d\right\}$ is a set of $d+1$ relations on $X$ satisfying the following properties:
\begin{itemize}
	\item $R_0$ is the identity relation;
	\item $\left\{R_0,R_1,\dots, R_d\right\}$ forms a partition of $X\times X$;
	\item $(x,y)\in R_i$ implies $(y,x)\in R_i$;
	\item for $0\leq i,j,k\leq d$ there exist \textit{intersection numbers} $p_{i,j}^k$ such that for any $(x,y)\in R_k$, the number of vertices $z$ for which $(x,z)\in R_i$ and $(z,y)\in R_j$ is equal to $p_{i,j}^k$ independent of our original choice of $x$ and $y$.
\end{itemize}
Throughout this paper, all association schemes are \emph{commutative}: we require
$p_{ij}^k=p_{ji}^k$ for all $i,j,k$.  The problems addressed here immediately reduce to the  
\emph{symmetric} case where $i'=i$ for all $i$; i.e., we will work with symmetric relations only. Often it becomes useful to order the vertices in $X$ and then represent each $R_i$ as a 01-matrix $A_i$ where the $(x,y)$ entry of $A_i$ is 1 if and only if $(x,y)\in R_i$. With this setting in mind, the defining properties above are encoded as:
\begin{itemize}
	\item $A_0 = I$;
	\item $\sum_i A_i = J$;
	\item for all $0\leq i\leq d$, $A_i^T = A_i$;
	\item for all $0\leq i,j\leq d$, $A_iA_j = \sum p_{i,j}^k A_k$.
\end{itemize}
The final condition tells us that $\mathbb{A} = \text{span}\left\{A_0,A_1,\dots A_d\right\}$ forms a matrix algebra under standard matrix multiplication. As our matrices are 01-matrices with disjoint support, this \emph{Bose-Mesner algebra} is also closed under Schur (element-wise) products. Using our symmetric property, we note that $p_{i,j}^k = p_{j,i}^k$ telling us that $A_iA_j = A_jA_i$ and our matrices commute with each other. This allows us to simultaneously diagonalize our matrices to give us $d+1$ orthogonal eigenspaces with projection operators $E_0,\dots,E_d$. As both $\left\{A_0,\dots,A_d\right\}$ and $\left\{E_0,\dots,E_d\right\}$ form bases for the Bose-Mesner algebra, there exists unique matrices $P$ and $Q$ so that
\begin{equation}
\label{PQmat}
A_i = \sum_{j} P_{ji} E_j,\qquad E_j = \frac{1}{\vert X\vert} \sum_{i} Q_{ij}A_i.
\end{equation}
We call $P$ and $Q$ the first and second eigenmatrices, respectively and note here that $P_{0i}$ is the valency of relation $R_i$ and $Q_{0j}$ is the rank of $E_j$. Finally, as our matrix algebra is closed under Schur products, we find that there exist structure constants $q_{i,j}^k$ such that for all $0\leq i,j\leq d$:
\[E_i\circ E_j = \frac{1}{\vert X\vert}\sum_k q_{i,j}^k E_k.\]
We call these parameters the Krein parameters of the association scheme. A \textit{$Q$-polynomial} (\textit{cometric}) association scheme is one in which the set $\left\{E_0,E_1,\dots,E_d\right\}$ may be ordered so that $q^{k}_{i,j} = 0$ whenever $k>i+j$ or $k<\vert i- j\vert$ and $q^{k}_{i,j}>0$ whenever $k = i+j$. Finally, we say an association scheme with $Q$-polynomial ordering $E_0,\dots,E_d$ is \textit{$Q$-antipodal} if $q^{k}_{d,d} >0$ when $k = 0$ but $q^{k}_{d,d} = 0$ for $0<k<d$. Given a $Q$-polynomial ordering $E_0,\dots,E_d$ we find it convenient to order relations so that $Q_{01}>Q_{11}>\dots>Q_{d1}$; we call this the natural ordering. We call these parameters the Krein parameters of the association scheme. We conclude this section by examining the Krein parameters of our scheme, and defining an algebra isomorphism from our Bose-Mesner algebra to a new matrix algebra. Defining $L_i^*$ such that
\[L_i^* = [q^k_{i,j}]_{k,j},\]
we may define the vector space $\mathbb{L}^* = \text{span}\left\{L_0^*,L_1^*,\dots,L_d^*\right\}$. From \cite[Lemma.~2.3.1(vi)]{BCN}, we have:
\begin{align}
L_i^*L_j^* = \sum_k q^m_{i,k}q^k_{j,l}&=\sum_k q^k_{i,j}q^m_{k,l}=\sum_{k}q^k_{i,j}L_k^*,\label{dblsum}\end{align}
showing that $\mathbb{L}^*$ is closed under matrix multiplication. Therefore we define a homomorphism $\phi^* : \mathbb{A}\rightarrow \mathbb{L}^*$ via taking $\phi^*(E_i) = \frac{1}{\vert X\vert}L_i^*$ for each $0\leq i\leq d$ and extending linearly. From \eqref{dblsum}, we see that
\[\phi^*(E_i\circ E_j) = \frac{1}{\vert X\vert}\sum_{k=0}^{d}q^k_{i,j}\phi^*\left(E_k\right) = \frac{1}{\vert X\vert^2}\sum_{k=0}^dq^k_{i,j}L_k^* = \left(\frac{1}{\vert X\vert}L_i^*\right)\left(\frac{1}{\vert X\vert}L_j^*\right) = \phi^*(E_i)\phi^*(E_j).\]
Therefore $\phi^*$ is an algebra isomorphism preserving the Schur product structure of $\mathbb{A}$.
\section{Strongly regular graphs--2-class association schemes}
A strongly regular graph with parameters $(v,k,\lambda,\mu)$ is a $k$-regular graph with $v$ points where every pair of adjacent (non-adjacent) vertices have exactly $\lambda$ ($\mu$) neighbors in common. Thus, a strongly regular graph $\Gamma$ corresponds to $2$-class association scheme where $\Gamma$ and $\overline{\Gamma}$ are the two non-trivial relations. Thus a 2-class association scheme has the following first and second eigenmatrices:
\begin{equation}\label{PQsrg}P = \left[\begin{array}{ccc}
1 & k & v-k-1\\
1 & r & -(r+1)\\
1 & s & -(s+1)
\end{array}\right]\qquad Q = \left[\begin{array}{ccc}
1 & f & g\\
1 & \frac{fr}{k} & \frac{gs}{k}\\
1 & \frac{f(1+r)}{k+1-v} & \frac{g(1+s)}{k+1-v}
\end{array}\right]\end{equation}
where $\Gamma$ has spectrum $k^1,r^f,s^g$. The following are two theorems which will prove useful later.
\begin{thm}\cite[Theorem.~1.3.1.(iii)]{BCN} Whenever $\mu>0$, the parameters of a strongly regular graph may be expressed in terms of $r$, $s$, and $\mu$ as
	\[k = \mu-rs, \qquad v = \frac{(k-r)(k-s)}{\mu},\qquad \lambda = \mu+r+s.\]
\end{thm}
\begin{thm}\cite{delsarte} 
	Let $\Gamma$ be a strongly regular graph with $v$ vertices, valency $k$, and smallest eigenvalue $-m$. If $C$ is a coclique of $\Gamma$, then
	\[\vert C\vert\leq v\left(1+\frac{k}{m}\right))^{-1}, \]
	with equality if and only if every vertex $\gamma\notin C$ has the same number of neighbors (namely $m$) in $C$.
\end{thm}
\section{Cometric Association Schemes}
Let $(X,\mathcal{R})$ be an $d$-class symmetric association scheme. We say $(X,\mathcal{R})$ is $Q$-polynomial, or \textit{cometric}, if there exists an ordering of the eigenspaces, say $E_0$, $E_1$,\dots, $E_d$, such that the Krein parameters satisfy the following conditions:
\begin{enumerate}
	\item $q^k_{i,j} = 0$ whenever $i+j<k$, and
	\item $q^{i+j}_{i,j}>0$ whenever $i+j\leq d$.
\end{enumerate}
Under these conditions, we see that for a given $0\leq j\leq d$, at most three possible choices for $k$ will allow $q_{1,k}^j>0$. Therefore let $c_j^* = q_{1,j-1}^j$, $a_j^* = q_{1,j}^j$ and $b_j^* = q_{1,j+1}^j$ for $0\leq j\leq d$ with the restriction that $b_d^*=c_0^*=0$. Given a cometric scheme, we define the Krein array as $\left\{b_0^*,b_1^*,\dots,b_{d-1}^*;c_1^*,c_2^*,\dots,c_d^*\right\}$ noting that for any $0\leq j\leq d$, $c_j^* + a_j^* + b_j^* = q^0_{1,1}$. We say a cometric scheme is $Q$-\emph{bipartite} if $a_j^*=0$ for all $0\leq j\leq d$. This is equivalent to the condition that $q_{i,j}^k=0$ whenever $i+j+k$ is odd. A cometric scheme is $Q$-\emph{antipodal} if $b_j^* = c_{d-j}^*$ for all $j$ except possibly $j = \lfloor \frac{d}{2}\rfloor$.\par
We may define orthogonal polynomials $q_j(t)$, $j=0,1,\dots,d$ by $q_0(t) = 1$, $q_1(t) = t$ and the three-term recurrence $tq_j(t) = c_{j+1}^*q_{j+1}(t) + a_j^*q_j(t) + b_{j-1}^*q_{j-1}(t)$. It follows that $\vert X\vert E_j = q_j(\vert X\vert E_1)$, for $j=0,1,\dots,d$ where matrix multiplication is computed entrywise. Since $\frac{1}{\vert X\vert}Q_{i,j}$ for $0\leq i\leq j$ are the entries of $E_j$, this also means that $q_j(Q_{i,1}) = Q_{j,1}$. Finally note that in the $Q$-bipartite case, $q_j(t)$ is an even polynomial if and only if $j$ is even.\par
The following two theorems will help us describe the quotient object we find in the $Q$-bipartite case where $I_r$ and $J_r$ denote the $r\times r$ identity and all ones matrix respectively:
\begin{thm}[\cite{MMW}] \label{mmw}The following are equivalent:
	\begin{enumerate}[label=(\roman*)]
		\item $(X,\mathbb{R})$ is imprimitive;
		\item for some $j>0$, $E_j$ has repeated columns;
		\item for some subset $\mathcal{I} = \left\{i_0=0,i_1,\dots,i_s\right\}$ of $\left\{0,1,\dots,d\right\}$ and some ordering of the vertices $\sum_{h=0}^s A_{i_h} = I_w\otimes J_r$ for integers $w$ and $r$ with $\vert X\vert=wr$, $1<w,r<\vert X\vert$;
		\item for some subset $\mathcal{J} = \left\{j_0=0,j_1,\dots,j_s\right\}$ of $\left\{0,1,\dots,d\right\}$ and some ordering of the vertices $\sum_{h=0}^s E_{j_h} = \frac{1}{r}\left(I_w\otimes J_r\right)$ for integers $w$ and $r$ with $\vert X\vert=wr$, $1<w,r<\vert X\vert$.
	\end{enumerate}
\end{thm}
Whenever $(iii)$ occurs, say with subset $\mathcal{I}$ as given in the theorem, we may partition our vertices into equivalence classes so that $x\sim y$ whenever $(x,y)\in R_i$ for some $i\in \mathcal{I}$. Let $X_1,\dots,X_w$ be the corresponding equivalence classes and define $\mathcal{R}' = \left\{R_i\in \mathcal{R} : i\in \mathcal{I}\right\}$. Then there exists \emph{subschemes} $(X_i,\mathcal{R}')$ for each equivalence class $X_i$. Further, we may define a \emph{quotient scheme} $(\tilde{X},\tilde{\mathcal{R}})$ of our original scheme with respect to $\mathcal{I}$ whose point set is the set of equivalence classes and whose relations are $R_{\tilde{i}} = \cup_{i\in \tilde{i}} R_i$ where each $\tilde{i} = \left\{0\leq j\leq d : p^x_{i,j} \text{ for }x\in \mathcal{I}\right\}$. Note that $\vert \tilde{\mathcal{R}}\vert = \vert \mathcal{J}\vert$ where $\mathcal{J}$ is the subset from Theorem $3.1(iv)$.
\begin{thm}[\cite{Suzuki}] \label{suzuki}Suppose $(X,\mathcal{R})$ is an imprimitive cometric association scheme. Then one of the following holds:
	\begin{enumerate}[label=(\roman*)]
		\item $(X,\mathcal{R})$ is $Q$-bipartite and $\mathcal{J} = \left\{0,2,4,\dots\right\}$
		\item $(X,\mathcal{R})$ is $Q$-antipodal and $\mathcal{J} = \left\{0,d\right\}$
		\item $(X,\mathcal{R})$ is a $4$-class scheme with Krein array $\left\{m,m-1,1,b_3^*;1,c_2^*,m-b_3^*,1\right\}$ and $\mathcal{J} = \left\{0,3\right\}$;
		\item $(X,\mathcal{R})$ is a $6$-class scheme with Krein array $\left\{m,m-1,1,b_3^*,b_4^*,1;1,c_2^*,m-b_3^*,1,c_5^*,m\right\}$ and $\mathcal{J} = \left\{0,3,6\right\}$.
	\end{enumerate}
\end{thm}
Cerzo and Suzuki \cite{cerzo} have shown that no association schemes of the third type exist.
\begin{cor}
	\label{SRG}
	The quotient scheme of a 4-class $Q$-bipartite association scheme is a strongly regular graph.
\end{cor}
\begin{proof}
	From Theorem \ref{suzuki}, we know $\mathcal{J} = \left\{0,2,4\right\}$ and therefore the quotient scheme has two nontrivial relations, forcing it to be strongly regular.
\end{proof}
\begin{thm}[\cite{BGKW},\cite{MMW}]
	\label{sym}
	If $(X,\mathcal{R})$ is $Q$-bipartite with $w$ dual bipartite classes of size $r$ each, then $r=2$. Under the natural ordering of relations, $\mathcal{I} = \left\{0,d\right\}$ and the sequence $m = Q_{01}>Q_{11}>\dots>Q_{d1}$ is symmetric about the origin. In particular, $Q_{\frac{d}{2},1} = 0$ whenever $d$ is even.
\end{thm}
\begin{cor}
	\label{evenpoly}
	If $(X,\mathcal{R})$ is $Q$-bipartite, then $Q_{i,j} = Q_{d-i,j}$ $(-Q_{d-i,j}$, resp.) whenever $j$ is even (odd).
\end{cor}
\begin{proof}
	$q_j(t)$ is even (odd) whenever $j$ is even (odd). Since $Q_{i,1} = -Q_{d-i,1}$ and $Q_{i,j} = q_j(Q_{i,1})$, the result follows.
\end{proof}

Let $(X,\cR)$ be a commutative $d$-class association scheme with \emph{basis relations}
$\cR=\{R_0,$ $\ldots,R_d\}$. For $1\le  i\le d$, we have
a (possibly directed) simple graph $\Gamma_i=(X,R_i)$ on $X$. For $a\in X$, the set $X$ is partitioned into \emph{subconstituents} $R_i(a) = \{ b\in X \mid (a,b)\in R_i \}$ ($0\le i\le d$) with respect to $a$. The association
scheme is \emph{symmetric} if all basis relations are symmetric; each $\Gamma_i$ may be considered as an
undirected graph in this case as $i'=i$ for all $i$. The association scheme  is \emph{primitive} 
\cite[Sec.~2.4]{Brouwer1989} if $\Gamma_i$ is connected for all $i=1,\ldots, d$ and \emph{imprimitive} otherwise. A 
\emph{system of imprimitivity} for $(X,\cR)$ is any non-trivial partition of $X$ consisting of the components of
some graph $(X,R)$ where $R$ is a union of basis relations. (The trivial partitions  $\{X\}$ and 
$\{ \{a\} \mid a\in X\}$ are not systems of imprimitivity.)
For each $i$, we  may construct  an undirected graph $H_i$ (possibly with loops) on vertex set 
$\{0,1,\ldots,d\}$, joining  $j$ to $k$ if $p_{ij}^k+p_{ik}^j > 0$. 
We call this the \emph{unweighted distribution diagram} corresponding to basis relation $R_i$.
%%%%%%%%%%%%  JUNE 16
%%%%%%%%%%%%
%%%%%%%%%%%% Check definition of H_i when scheme is not symmetric.

With reference to a fixed undirected graph $\Gamma$ % on $X$ 
with vertex set $V\Gamma$ and edge 
set $E\Gamma$,  we say that $a$ and $b$ are \emph{twins} if $a\neq b$ yet
$\Gamma(a)=\Gamma(b)$, where $\Gamma(a)$ denotes the set of neighbors of $a$ in graph $\Gamma$.
Write\footnote{Note that some authors assign another meaning to $\bot$; here, we follow \cite[p.\ 440]{Brouwer1989}.} 
$a^\bot = \{a \} \cup \Gamma(a)$. A graph $\Gamma$ is \emph{complete multipartite} if any two
non-adjacenct vertices are twins: i.e., the complement of $\Gamma$ is a union of complete graphs.