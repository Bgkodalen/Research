\documentclass{article}
\usepackage{amssymb}
\usepackage{amsmath}
\usepackage{mathtools}
\usepackage{enumerate}
\usepackage{esint}
\usepackage{siunitx}
%\usepackage{fullpage}
\usepackage[margin=1in,headheight=13.6pt]{geometry}
\usepackage{graphicx}
\usepackage{caption}
\usepackage{subcaption}
\usepackage{wrapfig}
\usepackage{epstopdf}
\usepackage{float}
%\usepackage{natbib}
\newcommand{\conj}[1]{\overline{#1}}
\newcommand{\newpar}{\vspace{5mm}\par}
\newcommand{\vnorm}[1]{\left\|#1\right\|}
\usepackage{amsthm}
\usepackage{units}
\usepackage{tikz}
\usepackage{fancyhdr}
\pagestyle{fancy}

\newtheorem{theorem}{Theorem}[section]
\newtheorem{lemma}{Lemma}[section]
\newtheorem{remark}{Remark}
\newcommand*{\swap}[2]{#2#1}

\lhead{Brian G. Kodalen}
\rhead{\today}

\begin{document}
\title{Bounds on Spherical Codes}
\section{Association schemes}
Let $X$ be a finite set of vertices. A \textit{symmetric d-class association scheme} (see \cite{BCN}) on $X$ is a pair $\mathcal{L} = (X,\mathcal{R})$ where $\mathcal{R} =\left\{R_0,R_1,\dots,R_d\right\}$ is a set of $d+1$ relations on $X$ satisfying the following properties:
\begin{itemize}
	\item $R_0$ is the identity relation;
	\item $\left\{R_0,R_1,\dots, R_d\right\}$ forms a partition of $X\times X$;
	\item $(x,y)\in R_i$ implies $(y,x)\in R_i$;
	\item for $0\leq i,j,k\leq d$ there exist \textit{intersection numbers} $p_{i,j}^k$ such that for any $(x,y)\in R_k$, the number of vertices $z$ for which $(x,z)\in R_i$ and $(z,y)\in R_j$ is equal to $p_{i,j}^k$ independent of our original choice of $x$ and $y$.
\end{itemize}
Often it becomes useful to order the vertices in $X$ and then represent each $R_i$ as a 01-matrix $A_i$ where the $(x,y)$ entry of $A_i$ is 1 if and only if $(x,y)\in R_i$. With this setting in mind, the defining properties above are encoded as:
\begin{itemize}
	\item $A_0 = I$;
	\item $\sum_i A_i = J$;
	\item for all $0\leq i\leq d$, $A_i^T = A_i$;
	\item for all $0\leq i,j,k\leq d$, $A_iA_j = \sum p_{i,j}^k A_k$.
\end{itemize}
The final condition tells us that $\mathbb{A} = \text{span}\left\{A_0,A_1,\dots A_d\right\}$ forms a matrix algebra under standard matrix multiplication. As our matrices are 01-matrices with disjoint support, this \emph{Bose-Mesner algebra} is also closed under Schur (element-wise) products. Using our symmetric property, we note that $p_{i,j}^k = p_{j,i}^k$ telling us that $A_iA_j = A_jA_i$ and our algebra is commutative. Therefore our adjacency matrices are simultaneously diagonalizable giving us $d+1$ orthogonal eigenspaces with projection operators $E_0,\dots,E_d$. As both $\left\{A_0,\dots,A_d\right\}$ and $\left\{E_0,\dots,E_d\right\}$ form bases for the Bose-Mesner algebra, there exists unique matrices $P$ and $Q$ so that
\begin{equation}
\label{PQmat}
A_i = \sum_{j} P_{ji} E_j,\qquad E_j = \frac{1}{\vert X\vert} \sum_{i} Q_{ij}A_i.
\end{equation}

We call $P$ and $Q$ the first and second eigenmatrices, respectively, and note here that $P_{0i}$ is the valency of relation $R_i$ and $Q_{0j}$ is the rank of $E_j$. Finally, as our matrix algebra is closed under Schur products, we find that there exist structure constants $q_{i,j}^k$ such that for all $0\leq i,j,k\leq d$:
\[E_i\circ E_j = \frac{1}{\vert X\vert}\sum_k q_{i,j}^k E_k.\]
We call these parameters the Krein parameters of the association scheme. We conclude this section by examining the Krein parameters of our scheme, and defining an algebra isomorphism from our Bose-Mesner algebra to a new matrix algebra. Defining $L_i^*$ such that
\[L_i^* = [q^k_{i,j}]_{k,j},\]
we may define the vector space $\mathbb{L}^* = \text{span}\left\{L_0^*,L_1^*,\dots,L_d^*\right\}$. From \cite[Lemma.~2.3.1(vi)]{BCN}, we have:
\begin{align}
L_i^*L_j^* = \sum_k q^m_{i,k}q^k_{j,l}&=\sum_k q^k_{i,j}q^m_{k,l}=\sum_{k}q^k_{i,j}L_k^*,\label{dblsum}\end{align}
showing that $\mathbb{L}^*$ is closed under matrix multiplication. Therefore we define a homomorphism $\phi^* : \mathbb{A}\rightarrow \mathbb{L}^*$ via taking $\phi^*(E_i) = \frac{1}{\vert X\vert}L_i^*$ for each $0\leq i\leq d$ and extending linearly. From \eqref{dblsum}, we see that
\[\phi^*(E_i\circ E_j) = \frac{1}{\vert X\vert}\sum_{k=0}^{d}q^k_{i,j}\phi^*\left(E_k\right) = \frac{1}{\vert X\vert^2}\sum_{k=0}^dq^k_{i,j}L_k^* = \left(\frac{1}{\vert X\vert}L_i^*\right)\left(\frac{1}{\vert X\vert}L_j^*\right) = \phi^*(E_i)\phi^*(E_j).\]
Therefore $\phi^*$ is an algebra isomorphism preserving the Schur product structure of $\mathbb{A}$.
\section{Co-metric Association schemes}
Let $(X,\mathcal{R})$ be an $d$-class symmetric association scheme. We say $(X,\mathcal{R})$ is $Q$-polynomial, or \textit{cometric}, if there exists an ordering of the eigenspaces, say $E_0$, $E_1$,\dots, $E_d$, such that the Krein parameters satisfy the following conditions:
\begin{enumerate}
	\item $q^k_{i,j} = 0$ whenever $i+j<k$, and
	\item $q^{i+j}_{i,j}>0$ whenever $i+j\leq d$.
\end{enumerate}
Under these conditions, let $c_j^* = q_{1,j-1}^j$, $a_j^* = q_{1,j}^j$ and $b_j^* = q_{1,j+1}^j$ for $0\leq j\leq d$ (letting $b_d^*=c_0^*=0$ for convenience). Then $c_j^* + a_j^* + b_j^* = q^0_{1,1}$ and we may define orthogonal polynomials $q_j(t)$, $j=0,1,\dots,d$ by $q_0(t) = 1$, $q_1(t) = t$ and the three-term recurrence $tq_j(t) = c_{j+1}^*q_{j+1}(t) + a_j^*q_j(t) + b_{j-1}^*q_{j-1}(t)$. It follows that $\vert X\vert E_j = q_j(\vert X\vert E_1)$, for $j=0,1,\dots,d$ where matrix multiplication is computed entrywise. Defining one final polynomial $q_{d+1}(t)$ as
\[q_{d+1}(t) = tq_d(t) - b_{d-1}^*q_{d-1}(t) - a_d^*q_{d}(t),\]
we see that $q_{d+1}(\vert X\vert E_1) = 0$. Therefore we consider the polynomial ring $R = \nicefrac{\mathbb{R}[t]}{q_{d+1}(t)}$ with $\text{span}\left\{q_0,\dots,q_d\right\}$ as a basis. For each polynomial $F\in R$, there exists unique constants $f_0,f_1,\dots, f_d\in\mathbb{R}$ so that $F = \sum_j f_j q_j$. Given our basis polynomials, $\left\{f_0,f_1,\dots,f_d\right\}$ is the set of eigenvalues of $F\left(\vert X\vert E_1\right)$. Therefore we say $F$ is a positive semi-definite (psd) function if $f_j\geq 0$ for $0\leq j\leq d$.\newpar




This tells us that every entry of $E_1$ must be a root of $p_{d+1}(x)$. Further, since $\left\{E_0,E_1,\dots E_d\right\}$ forms a set of linearly independent matrices, $p_{d+1}(x)$ must be the minimal (entrywise) polynomial of $E_1$. Therefore $E_1$ must contain exactly $d+1$ distinct entries and likewise column one of $Q$ must contain $d+1$ distinct entries. This immediately implies that every column of $E_1$ is distinct as the the $i^\text{th}$ column will be the only column whose $i^\text{th}$ entry is $Q_{0,1}$. We find it convenient to order the relations so that $Q_{01}>Q_{11}>\dots>Q_{d1}$; we call this the natural ordering wrt the $Q$-polynomial ordering $E_0,E_1,\dots,E_d$.\newpar
Now consider that $E_1$ is a positive semi-definite matrix of rank $m = Q_{0,1}$. Therefore there exists a $m\times \vert X\vert$ matrix $U$ such that $U^TU = E_1$. Further, since the main diagonal of $E_1$ is constant and every column is distinct, we must have that the columns of $U$ are $\vert X\vert$ distinct vectors of constant norm $\frac{m}{\vert X\vert}$. Therefore $\sqrt{\frac{\vert X\vert}{m}}U$ is a set of unit vectors in dimension $m$ with Gram matrix $\frac{\vert X\vert}{m}E_1$.
\section{Spherical Bound}
Let $\Omega_d$ denote the unit sphere in Euclidean space $\mathbb{R}^d$. For any $k\geq 0$ let $\text{Harm}(k) = \text{Harm}_d(k)$ denote the space of harmonic and homogeneous polynomials of degree $k$ on $\Omega_d$. Let $N_d = \dim\text{Harm}(k)$ and fix an orthogonal basis $\left\{W_{k,i}\right\}_{i=1..N}$ of $\text{Harm}(k)$. Let $Q_k^d$ denote the $k^\text{th}$ degree Gegenbauer polynomial for dimension $d$. From \cite[Theorem.~3.3.]{DGS} we have,
\[\sum_{i=1}^N W_{k,i}(\zeta)W_{k,i}(\eta) = Q_k(\langle\zeta,\eta\rangle); \qquad \zeta,\eta\in\Omega_d.\]
Given a finite set $X\subset \Omega_d$, define $H_k = [W_{k,i}(\zeta)],\qquad \zeta\in X, i\in \left\{1,2,\dots, N\right\}$. Then \cite[Theorem.~3.6.]{DGS},
\[H_kH_k^T = [Q_k(\langle \zeta,\eta\rangle)]_{\zeta,\eta\in X} = Q_k^\circ(G_X).\]
where $G_X$ is the Gram matrix of $X$ and $Q_k^\circ$ is the Gegenbauer polynomial of degree $k$ applied entrywise. Now for any vector $v\in \mathbb{R}^{\vert X\vert}$,
\[v^TH_kH_k^Tv = \vnorm{v^TH_k}^2\geq 0.\]
Therefore, we have the following result
\begin{theorem}[DGS]
	\label{psdgeg}
	Let $X$ be a set of unit vectors in $\mathbb{R}^m$ with Gram matrix $G$. Then,
	\[Q_k^\circ(G)\succeq0\]	
\end{theorem}
This leads to the following theorem:
\begin{theorem}
	Let $(X,\mathcal{R})$ be a $Q$-polynomial association scheme with $Q$-polynomial ordering $E_0,E_1,\dots,E_d$. Let $m=\text{rank}(E_1)$, define $L_1^* = [q^k_{1,j}]_{k,j}$ and
	\[F_k = Q_k\left(\frac{1}{m}L_1^*\right).\]
	for $k\geq 0$. Then $F_k$ must be non-negative for each $k\geq 0$.
\end{theorem}
\begin{proof}
	From above, we know that there exists a $m\times \vert X\vert$ matrix $U$ with unit vector columns such that $U^TU = \frac{\vert X\vert}{m}E_1$. Then, by \ref{psdgeg},
	\[Q_k^\circ\left(\frac{\vert X\vert}{m}E_1\right)\succeq0\]
	However, since $\left<E_i\right>$ is closed under entrywise multiplication, there exists constants $c_i$ such that
	\begin{equation}\label{Gkeig}Q_k^\circ\left(\frac{\vert X\vert}{m}E_1\right) = \sum_{i=0}^d\vert X\vert c_iE_i.\end{equation}
	Note, since the $E_i$'s represent orthogonal idempotents, these $\vert X\vert c_i$'s are exactly the eigenvalues of $Q_k^\circ\left(\frac{\vert X\vert}{m}E_1\right)$. Using $\phi^*$ on equation \ref{Gkeig}, we arrive at
	\[Q_k\left(\frac{1}{m}L_1^*\right) = \sum_{i=0}^dc_iL_i^*\]
	Since each $L_i^*$ is non-negative, our result follows. It is worth noting that since $[L_i^*]_{0,j} = m_j\delta_{i,j}$, it is sufficient to check the first row of $F_k$ to guarantee $c_i\geq 0$ for all $0\leq i\leq d$. This also means that not only does $Q_k^\circ\left(\frac{\vert X\vert}{m}E_1\right)\succeq 0$ imply $F_k$ is non-negative, but in fact these are equivalent conditions.
\end{proof}
\section{Computations \& Results}
The Gegenbauer polynomials in dimension $m$ are defined via $Q_0^{(m)}(t) = 1$, $Q_1^{(m)}(t) = t$ and the three term recurrence:
\[Q_k^{(m)} = \frac{(2k+m-4)tQ_{k-1}^{(m)}(t) - (k-1)Q_{k-2}^{(m)}(t)}{k+m-3}, \qquad k\geq 2.\]
Note these are normalized so that $Q_k^{(m)}(1) = 1$. Below we have listed the first six polynomials:
\[\begin{aligned}
Q_0^m(t)&=1\\
Q_1^m(t)&=t\\
Q_2^m(t)&=\frac{mt^2 - 1}{m-1}\\
Q_3^m(t)&=\frac{(m+2)t^3 - 3t}{(n-1)}\\
Q_4^m(t)&=\frac{(m+4)(m+2)t^4 - 6(m+2)t^2+3}{m^2-1}\\
Q_5^m(t)&=\frac{(m+6)(m+4)t^5-10(m+4)t^3+15t}{m^2-1}\\
Q_6^m(t)&=\frac{(m+8)(m+6)(m+4)t^6-15(m+6)(m+4)t^4+45(m+4)t^2-15}{(m+3)(m+1)(m-1)}\\
\end{aligned}\]
Using these polynomials we arrive at the following results
\subsection{Example Constraints}
\begin{itemize}
	\item $4^{\text{th}}$ degree constraint:
	\[(q^1_{11})^2+q^1_{12}q^2_{11}\geq\frac{2m(m-1)}{m+2}\]
	\item $5^\text{th}$ degree constraint.\\
	Assume:
	\begin{itemize}
		\item[$\cdot$] $m>2$
		\item[$\cdot$] $q^1_{11}>0$
	\end{itemize}
	Then:
	\[(q^1_{11})^2+\left(2+\frac{q^2_{12}}{q^1_{11}}\right)q^1_{12}q^2_{11}\geq\frac{4m(2m-3)}{m+6}\]
\end{itemize}
\subsection{Example Results}
Each scheme listed below is using the $(v,m[a-z])$ notation used in Williford's tables available online.\\
The following 3-class schemes have the property that the $(0,0)$ entry in $F_5$ is negative:
\[\left\{(441,20),(576,23),(729,26),(1015,28),(1240,30),(1548,35),(1836,35),(1944,29),(1976,25),(1000,27a),(1331,30a)\right\}.\]
All but the last two were not previously ruled out.
The following 4-class schemes have the property that the $(0,0)$ entry in $F_6$ is negative:
\[\left\{(594,9)^*,(4968,27),(5280,30),(5436,27),(6148,29),(7776,27)^*,(8432,31)^\dagger,(8478,27)^*,(9984,24)^*,(9984,32)^\dagger\right\}.\]
The marked schemes also have negative entries in lower degree $F$'s. Specifically those marked with a $*$ have negative entries in both $F_5$ and $F_4$ while the schemes with $\dagger$ have negative entries in $F_5$.\newpage
\subsection{Example Calculations}
Consider the 3-class scheme with $v=1015$ and $m=28$. This scheme has
\[L_1^* = \left[ \begin {array}{cccc} 0&28&0&0\\ \noalign{\medskip}1& 7.59&
19.0&0\\ \noalign{\medskip}0& 1.44& 13.95& 13.0\\ \noalign{\medskip}0
&0& 12.61& 15.0\end {array} \right]\]
The following are computed $F$'s for this scheme:
\[\begin{aligned}
{\it F2}&= \left[ \begin {array}{cccc} 0.0& 0.28&
0.72& 0.0\\ \noalign{\medskip} 0.010& 0.11& 0.55& 0.32
\\ \noalign{\medskip} 0.0019& 0.041& 0.47& 0.49\\ \noalign{\medskip}
0.0& 0.024& 0.49& 0.49\end {array} \right] \qquad
&{\it F3}= \left[ \begin {array}{cccc}  0.011& 0.050& 0.59& 0.35
\\ \noalign{\medskip} 0.0018& 0.055& 0.49& 0.46\\ \noalign{\medskip}
0.0016& 0.036& 0.48& 0.48\\ \noalign{\medskip} 0.00092& 0.034& 0.48&
0.48\end {array} \right] \\
{\it F4}&= \left[ \begin {array}{cccc}  0.0020& 0.031& 0.46& 0.50
\\ \noalign{\medskip} 0.0011& 0.034& 0.48& 0.49\\ \noalign{\medskip}
0.0012& 0.036& 0.48& 0.48\\ \noalign{\medskip} 0.0013& 0.036& 0.48&
0.48\end {array} \right] \qquad
&{\it F5}= \left[ \begin {array}{cccc} - 0.00017& 0.032& 0.46& 0.50
\\ \noalign{\medskip} 0.0011& 0.032& 0.48& 0.49\\ \noalign{\medskip}
0.0012& 0.036& 0.48& 0.48\\ \noalign{\medskip} 0.0013& 0.036& 0.48&
0.48\end {array} \right] \\
{\it F6}&= \left[ \begin {array}{cccc}  0.0010& 0.033& 0.48& 0.48
\\ \noalign{\medskip} 0.0012& 0.035& 0.48& 0.48\\ \noalign{\medskip}
0.0013& 0.036& 0.48& 0.48\\ \noalign{\medskip} 0.0013& 0.036& 0.48&
0.48\end {array} \right] 
\end{aligned}\]
\newpage
Consider the 4-class scheme with $v=9984$ and $m=24$. This scheme has
\[L_1^* = \left[ \begin {array}{ccccc} 0&24&0&0&0\\ \noalign{\medskip}1&0&23&0&0
\\ \noalign{\medskip}0& 1.85&0& 22.15&0\\ \noalign{\medskip}0&0& 1.33&0
& 22.67\\ \noalign{\medskip}0&0&0&24&0\end {array} \right] 
\]
The following are computed $F$'s for this scheme:
\[\small\begin{aligned}
{\it F2}&= \left[ \begin {array}{ccccc} { 0.0}& 0.0& 1.0
& 0.0& 0.0\\ \noalign{\medskip} 0.0& 0.077& 0.0& 0.92& 0.0
\\ \noalign{\medskip} 0.0034& 0.0& 0.087& 0.0& 0.91
\\ \noalign{\medskip} 0.0& 0.0045& 0.0& 1.0& 0.0\\ \noalign{\medskip}
0.0& 0.0& 0.058& 0.0& 0.94\end {array} \right]\qquad
&{\it F3}= \left[ \begin {array}{ccccc}  0.0& 0.00017& 0.0& 1.0& 0.0
\\ \noalign{\medskip} 0.0000072& 0.0& 0.056& 0.0& 0.94
\\ \noalign{\medskip} 0.0& 0.0045& 0.0& 1.0& 0.0\\ \noalign{\medskip}
0.00020& 0.0& 0.060& 0.0& 0.94\\ \noalign{\medskip} 0.0& 0.0048& 0.0&
1.0& 0.0\end {array} \right]\\
{\it F4}&= \left[ \begin {array}{ccccc}  0.0000081& 0.0&- 0.058& 0.0&
1.1\\ \noalign{\medskip} 0.0&- 0.0044& 0.0& 1.0& 0.0
\\ \noalign{\medskip}- 0.00019& 0.0& 0.056& 0.0& 0.94
\\ \noalign{\medskip} 0.0& 0.0049& 0.0& 1.0& 0.0\\ \noalign{\medskip}
0.00023& 0.0& 0.060& 0.0& 0.94\end {array} \right]\qquad
&{\it F5}= \left[ \begin {array}{ccccc}  0.0&- 0.0052& 0.0& 1.0& 0.0
\\ \noalign{\medskip}- 0.00021& 0.0& 0.051& 0.0& 0.95
\\ \noalign{\medskip} 0.0& 0.0041& 0.0& 1.0& 0.0\\ \noalign{\medskip}
0.00020& 0.0& 0.060& 0.0& 0.94\\ \noalign{\medskip} 0.0& 0.0049& 0.0&
1.0& 0.0\end {array} \right]\\
{\it F6}&= \left[ \begin {array}{ccccc} - 0.00026& 0.0& 0.071& 0.0&
0.93\\ \noalign{\medskip} 0.0& 0.0052& 0.0& 0.99& 0.0
\\ \noalign{\medskip} 0.00024& 0.0& 0.060& 0.0& 0.94
\\ \noalign{\medskip} 0.0& 0.0048& 0.0& 1.0& 0.0\\ \noalign{\medskip}
0.00020& 0.0& 0.060& 0.0& 0.94\end {array} \right] 
\end{aligned}
\]
\begin{thebibliography}{10}
	
	\bibitem{DGS}
	P.~{Delsarte}, J.M.~{Goethals}, and J.J.~{Seidel}.
	\newblock {\em Spherical Codes and Designs}.
	\newblock {Geometriae Dedicata}, 6(3):363-388, Sep. 1977.
	
	\bibitem{BCN}
	A.~Brouwer, A.~Cohen, and A.~Neumaier.
	\newblock {\em Distance-Regular Graphs}.
	\newblock Springer-Verlag Berlin, 1989.
	
	\bibitem{Colbourn}
	C.~J. Colbourn and J.~H. Dinitz.
	\newblock {\em Handbook of Combinatorial Designs, Second Edition (Discrete
		Mathematics and Its Applications)}.
	\newblock Chapman \& Hall/CRC, 2006.
	
	\bibitem{DMM}
	E.~R.~{\swap{Dam}{van }}, W.~Martin, and M.~Muzychuk.
	\newblock Uniformity in Association Schemes and Coherent Configurations:
	Cometric $Q$-Antipodal Schemes and Linked Systems.
	\newblock {\em Journal of Combinatorial Theory, Series A}, 120:1401--1439,
	2013.
	
	
	\bibitem{MMW}
	W.~J. Martin, M.~Muzychuk, and J.~Williford.
	\newblock Imprimitive Cometric Association Schemes: Constructions and Analysis.
	\newblock {\em Journal of Algebraic Combinatorics}, 25(4):399--415, Jun 2007.
	
\end{thebibliography}
\end{document}
