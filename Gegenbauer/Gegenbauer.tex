\documentclass{article}
\usepackage{amssymb}
\usepackage{amsmath}
\usepackage{mathtools}
\usepackage{enumerate}
\usepackage{esint}
\usepackage{siunitx}
%\usepackage{fullpage}
\usepackage[margin=1in,headheight=13.6pt]{geometry}
\usepackage{graphicx}
\usepackage{caption}
\usepackage{subcaption}
\usepackage{wrapfig}
\usepackage{epstopdf}
\usepackage{float}
\usepackage{natbib}
\newcommand{\conj}[1]{\overline{#1}}
\newcommand{\newpar}{\vspace{5mm}\par}
\newcommand{\vnorm}[1]{\left\|#1\right\|}
\usepackage{amsthm}
\usepackage{units}
\usepackage{tikz}
\usepackage{fancyhdr}
\pagestyle{fancy}

\newtheorem{theorem}{Theorem}[section]
\newtheorem{lemma}{Lemma}[section]
\newtheorem{remark}{Remark}

\lhead{Brian G. Kodalen}
\rhead{\today}

\begin{document}
\title{Bounds on Spherical Codes}
\section{Association schemes}
Let $X$ be a finite set of vertices. A \textit{symmetric d-class association scheme} (see \cite{BCN}) on $X$ is a pair $\mathcal{L} = (X,\mathcal{R})$ where $\mathcal{R} =\left\{R_0,R_1,\dots,R_d\right\}$ is a set of $d+1$ relations on $X$ satisfying the following properties:
\begin{itemize}
	\item $R_0$ is the identity relation;
	\item $\left\{R_0,R_1,\dots, R_d\right\}$ forms a partition of $X\times X$;
	\item $(x,y)\in R_i$ implies $(y,x)\in R_i$;
	\item for $0\leq i,j,k\leq d$ there exist \textit{intersection numbers} $p_{i,j}^k$ such that for any $(x,y)\in R_k$, the number of vertices $z$ for which $(x,z)\in R_i$ and $(z,y)\in R_j$ is equal to $p_{i,j}^k$ independent of our original choice of $x$ and $y$.
\end{itemize}
Often it becomes useful to order the vertices in $X$ and then represent each $R_i$ as a 01-matrix $A_i$ where the $(x,y)$ entry of $A_i$ is 1 if and only if $(x,y)\in R_i$. With this setting in mind, the defining properties above are encoded as:
\begin{itemize}
	\item $A_0 = I$;
	\item $\sum_i A_i = J$;
	\item for all $0\leq i\leq d$, $A_i^T = A_i$;
	\item for all $0\leq i,j,k\leq d$, $A_iA_j = \sum p_{i,j}^k A_k$.
\end{itemize}
The final condition tells us that $\mathbb{A} = \text{span}\left\{A_0,A_1,\dots A_d\right\}$ forms a matrix algebra under standard matrix multiplication. As our matrices are 01-matrices with disjoint support, this \emph{Bose-Mesner algebra} is also closed under Schur (element-wise) products. Using our symmetric property, we note that $p_{i,j}^k = p_{j,i}^k$ telling us that $A_iA_j = A_jA_i$ and our algebra is commutative. Therefore our adjacency matrices are simultaneously diagonalizable giving us $d+1$ orthogonal eigenspaces with projection operators $E_0,\dots,E_d$. As both $\left\{A_0,\dots,A_d\right\}$ and $\left\{E_0,\dots,E_d\right\}$ form bases for the Bose-Mesner algebra, there exists unique matrices $P$ and $Q$ so that
\begin{equation}
\label{PQmat}
A_i = \sum_{j} P_{ji} E_j,\qquad E_j = \frac{1}{\vert X\vert} \sum_{i} Q_{ij}A_i.
\end{equation}
We call $P$ and $Q$ the first and second eigenmatrices, respectively, and note here that $P_{0i}$ is the valency of relation $R_i$ and $Q_{0j}$ is the rank of $E_j$. Finally, as our matrix algebra is closed under Schur products, we find that there exist structure constants $q_{i,j}^k$ such that for all $0\leq i,j,k\leq d$:
\[E_i\circ E_j = \frac{1}{\vert X\vert}\sum_k q_{i,j}^k E_k.\]
We call these parameters the Krein parameters of the association scheme. We conclude this section by examining the intersection numbers and Krein parameters of our scheme, and defining algebra homomorphisms from our Bose-Mesner algebra to two other matrix algebras. Defining $L_i$ and $L_i^*$ such that
\[L_i = [p^k_{i,j}]_{k,j}\qquad L_i^* = [q^k_{i,j}]_{k,j},\]
\cite[Lemma.~2.1.1(vi)]{BCN} and \cite[Lemma.~2.3.1(vi)]{BCN} gives the following equations:
\begin{align}L_iL_j = \sum_k p^m_{i,k}p^k_{j,l}&=\sum_k p^k_{i,j}p^m_{k,l}= \sum_{k} p^k_{i,j}L_k;\label{dblsum:1}\\
L_i^*L_j^* = \sum_k q^m_{i,k}q^k_{j,l}&=\sum_k q^k_{i,j}q^m_{k,l}=\sum_{k}q^k_{i,j}L_k^*.\label{dblsum:2}\end{align}
Therefore, $\mathbb{L}= \langle L_0,L_1,\dots,L_d\rangle$ and $\mathbb{L}^* = \langle L_0^*,L_1^*,\dots,L_d^*\rangle$ are two matrix algebras under standard matrix multiplication. Further, we define $\phi: \mathbb{A}\rightarrow \mathbb{L}$ and $\phi^* : \mathbb{A}\rightarrow \mathbb{L}^*$ via
\[\phi(A_i) = \frac{1}{\vert X\vert}L_i\qquad\phi(XY) = \phi(X)\phi(Y)\]
and
\[\phi^*(E_i) = \frac{1}{\vert X\vert}L_i^*\qquad\phi^*(X\circ Y) = \phi^*(X)\phi^*(Y).\]
Therefore $\phi$ preserves the matrix multiplication structure of $\mathbb{A}$ while $\phi^*$ preserves the Schur multiplication structure of $\mathbb{A}$.
\section{Co-metric Association schemes}
Let $(X,\mathcal{R})$ be an $d$-class symmetric association scheme. We say $(X,\mathcal{R})$ is $Q$-polynomial, or \textit{cometric}, if there exists an ordering of the eigenspaces, say $E_0,E_1,\dots,E_d$, such that the Krein parameters satisfy the following conditions:
\begin{enumerate}
	\item $q^k_{i,j} = 0$ whenever $i+j<k$, and
	\item $q^{i+j}_{i,j}>0$ whenever $i+j\leq d$.
\end{enumerate}
Under these conditions, we define orthogonal polynomials $p_j(t)$, $j=0,1,\dots,d$ by $p_0(t) = 1$, $p_1(t) = t$ and the three-term recurrence $tp_j(t) = q^{j-1}_{1,j}p_{j-1}(t) + q^{j}_{1,j}p_j(t) + q^{j+1}_{1,j}p_{j+1}(t)$. It follows that $vE_j = q_j(vE_1)$, for $j=0,1,\dots,d$ where matrix multiplication is done entrywise. Defining one final polynomial $p_{d+1}(t)$ as
\[p_{d+1}(t) = xp_d(t) - q^{d-1}_{1,d}p_{d-1}(t) - q^d_{1,d}p_{d}(t),\]
we see that $p_{d+1}(vE_1) = 0$. This tells us that every entry of $E_1$ must be a root of $p_{d+1}(x)$. Further, since $\left\{E_0,E_1,\dots E_d\right\}$ forms a set of linearly independent matrices, $p_{d+1}(x)$ must be the minimal (entrywise) polynomial of $E_1$. Therefore $E_1$ must contain exactly $d+1$ distinct entries and likewise column one of $Q$ must contain $d+1$ distinct entries. This immediately implies that every column of $E_1$ is distinct as the the $i^\text{th}$ column will be the only column whose $i^\text{th}$ entry is $Q_{0,1}$. We also find it convenient to order the relations so that $Q_{01}>Q_{11}>\dots>Q_{d1}$; we call this the natural ordering wrt the $Q$-polynomial ordering $E_0,E_1,\dots,E_d$.\newpar
Now consider that $E_1$ is a positive semi-definite matrix of rank $m = Q_{0,1}$. Therefore there exists a $m\times \vert X\vert$ matrix $U$ such that $U^TU = E_1$. Further, since the main diagonal of $E_1$ is constant and every column is distinct, we must have that the columns of $U$ are $\vert X\vert$ distinct vectors of constant norm $\frac{m}{\vert X\vert}$. Therefore $\sqrt{\frac{\vert X\vert}{m}}U$ is a set of unit vectors in dimension $m$ with Gram matrix $\frac{\vert X\vert}{m}E_1$.
\section{Spherical Bound}
Let $\Omega_d$ denote the unit sphere in Euclidean space $\mathbb{R}^d$. For any $k\geq 0$ let $\text{Harm}_d(k)$ denote the  space of harmonic, homogeneous polynomials on $\Omega_d$. Let $N_d = \dim\text{ Harm}_d(k)$ and fix an orthogonal basis $\left\{W_{k,i}\right\}_{i=1..N}$ of $\text{Harm}(k)$. Let $Q_k^d$ denote the $k^\text{th}$ degree Gegenbauer polynomial for dimension $d$. From \cite[Theorem.~3.3.]{DGS} we have,
\[\sum_{i=1}^N W_{k,i}(\zeta)W_{k,i}(\eta) = Q_k(\langle\zeta,\eta\rangle); \qquad \zeta,\eta\in\Omega_d.\]
Given a finite set $X\subset \Omega_d$, define $H_k = [W_{k,i}(\zeta)],\qquad \zeta\in X, i\in \left\{1,2,\dots, N\right\}$. Then \cite[Theorem.~3.6.]{DGS},
\[H_kH_k^T = [Q_k(\langle \zeta,\eta\rangle)]_{\zeta,\eta\in X} = Q_k^\circ(G_X).\]
where $G_X$ is the Gram matrix of $X$ and $Q_k^\circ$ is the Gegenbauer polynomial of degree $k$ applied entrywise. Now for any vector $v\in \mathbb{R}^{\vert X\vert}$,
\[v^TH_kH_k^Tv = \vnorm{v^TH_k}^2\geq 0.\]
Therefore, we have the following result
\begin{theorem}[DGS]
	\label{psdgeg}
	Let $X$ be a set of unit vectors in $\mathbb{R}^m$ with Gram matrix $G$. Then,
	\[Q_k^\circ(G)\succeq0\]	
\end{theorem}
This leads to the following theorem:
\begin{theorem}
	Let $(X,\mathcal{R})$ be a $Q$-polynomial association scheme with $Q$-polynomial ordering $E_0,E_1,\dots,E_d$. Let $m=\text{rank}(E_1)$, define $L_1^* = [q^k_{1,j}]_{k,j}$ and
	\[F_k = Q_k\left(\frac{1}{m}L_1^*\right).\]
	for $k\geq 0$. Then $F_k$ must be non-negative for each $k\geq 0$.
\end{theorem}
\begin{proof}
	From above, we know that there exists a $m\times \vert X\vert$ matrix $U$ with unit vector columns such that $U^TU = \frac{\vert X\vert}{m}E_1$. Then, by \ref{psdgeg},
	\[Q_k^\circ\left(\frac{\vert X\vert}{m}E_1\right)\succeq0\]
	However, since $\left<E_i\right>$ is closed under entrywise multiplication, there exists constants $c_i$ such that
	\begin{equation}\label{Gkeig}Q_k^\circ\left(\frac{\vert X\vert}{m}E_1\right) = \sum_{i=0}^d\vert X\vert c_iE_i.\end{equation}
	Note, since the $E_i$'s represent orthogonal idempotents, these $\vert X\vert c_i$'s are exactly the eigenvalues of $Q_k^\circ\left(\frac{\vert X\vert}{m}E_1\right)$. Using $\phi^*$ on equation \ref{Gkeig}, we arrive at
	\[Q_k^\circ\left(\frac{1}{m}L_1^*\right) = \sum_{i=0}^dc_iL_i^*\]
	Since each $L_i^*$ is non-negative, our result follows. It is worth noting that since $[L_i^*]_{0,j} = m_j\delta_{i,j}$, it is sufficient to check the first row of $F_k$ to guarantee $c_i\geq 0$ for all $0\leq i\leq d$. This also means that not only does $Q_k^\circ\left(\frac{\vert X\vert}{m}E_1\right)\succeq 0$ imply $F_k$ is non-negative, but in fact these are equivalent conditions.
\end{proof}
\section{Computations \& Results}
The Gegenbauer polynomials in dimension $m$ are defined via $Q_0^{(n)}(t) = 1$, $Q_1^{(n)}(t) = t$ and the three term recurrence:
\[Q_k^{(m)} = \frac{(2k+m-4)tQ_{k-1}^{(m)}(t) - (k-1)Q_{k-2}^{(m)}(t)}{k+m-3}, \qquad k\geq 2.\]
Note these are normalized so that $Q_k^{(m)}(1) = 1$. Below we have listed the first six polynomials:
\[\begin{aligned}
Q_0^n(t)&=1\\
Q_1^n(t)&=t\\
Q_2^n(t)&=\frac{nt^2 - 1}{n-1}\\
Q_3^n(t)&=\frac{(n+2)t^3 - 3t}{(n-1)}\\
Q_4^n(t)&=\frac{(n+4)(n+2)t^4 - 6(n+2)t^2+3}{n^2-1}\\
Q_5^n(t)&=\frac{(n+6)(n+4)t^5-10(n+4)t^3+15t}{n^2-1}
\end{aligned}\]
Using these polynomials we arrive at the following results
\subsection{Example Constraints}
\begin{itemize}
	\item $4^{\text{th}}$ degree constraint:
	\[(q^1_{11})^2+q^1_{12}q^2_{11}\geq\frac{2m(m-1)}{m+2}\]
	\item $5^\text{th}$ degree constraint.\\
	Assume:
	\begin{itemize}
		\item[$\cdot$] $m>2$
		\item[$\cdot$] $q^1_{11}>0$
	\end{itemize}
	Then:
	\[(q^1_{11})^2+\left(2+\frac{q^2_{12}}{q^1_{11}}\right)q^1_{12}q^2_{11}\geq\frac{4m(2m-3)}{m+6}\]
\end{itemize}
\subsection{Example Results}
Each scheme listed below is using the $(v,m[a-z])$ notation used in Williford's tables available online.\\
The following 3-class schemes have the property that $F_5$ contains negative values:
\[\left\{(441,20),(576,23),(729,26),(1015,28),(1240,30),(1548,35),(1836,35),(1944,29),(1976,25),(1000,27a),(1331,30a)\right\}.\]
All but the last two were not previously ruled out.
The following 4-class schemes have the property that $F_6$ contains negative values:
\[\left\{(594,9),(4968,27),(5280,30),(5436,27),(6148,29),(7776,27),(8432,31),(8478,27),(9984,24),(9984,32)\right\}.\]
Most (all?) of these 4-class schemes also produce negative values in $F_5$ and $F_4$.
\end{document}
