\documentclass{beamer}
\usepackage{amssymb}
\usepackage{amsmath}
\usepackage{mathtools}
\usepackage{enumerate}
\usepackage{esint}
\usepackage{siunitx}
\usepackage{bbm}
\usepackage{graphicx}
\usepackage{caption}
\usepackage{subcaption}
\usepackage{wrapfig}
\usepackage{epstopdf}
\usepackage{float}
%\usepackage{cite}
\usepackage{tikz}
%\usepackage{collcell}

\newtheorem{thm}{Theorem}
\newtheorem{prop}{Proposition}
\newtheorem{lem}{Lemma}
\newtheorem{cor}{Corollary}
\newtheorem{conjec}{Conjecture}
\newtheorem{remark}{Remark}

\usetikzlibrary{calc,arrows,quotes,angles}
%\usepackage{arydshln}
\usetikzlibrary{shapes,decorations}
\usepackage{multicol}
\usepackage{booktabs}
\usepackage{beamerthemesplit}


\newcommand{\conj}[1]{\overline{#1}}
\newcommand{\newpar}{\vspace{5mm}\par}
\newcommand{\vnorm}[1]{\left\|#1\right\|}
\newcommand{\ket}[1]{\left\vert#1\right>}
\newcommand{\bra}[1]{\left<#1\right\vert}
\usepackage{amsthm}
\mode<presentation>{}
\newcommand{\tabitem}{~~\llap{\textbullet}~~}

%%% Beamer colors
\usetheme{Madrid}
\usecolortheme{beaver}
\definecolor{Maroon}{cmyk}{0,0.8,0.68,0.32}
%\setbeamercolor*{palette primary}{fg=black,bg=Maroon}
%\setbeamercolor*{palette secondary}{fg=white,bg=structure.fg!60!white}
%\setbeamercolor*{palette tertiary}{fg=white,bg=structure.fg!90!white}
%\setbeamercolor*{palette quaternary}{fg=white,bg=black}

%\setbeamercolor*{sidebar}{use=structure,bg=structure.fg}

%\setbeamercolor*{palette sidebar primary}{fg=structure.fg!10}
%\setbeamercolor*{palette sidebar secondary}{fg=white}
%\setbeamercolor*{palette sidebar tertiary}{fg=structure.fg!50}
%\setbeamercolor*{palette sidebar quaternary}{fg=white}

%\setbeamercolor*{titlelike}{fg=structure.fg}

%\setbeamercolor*{separation line}{}
%\setbeamercolor*{fine separation line}{}

%\setbeamercolor{title}{fg=Maroon}
\setbeamercolor{block title}{fg=white, bg=red!75!black}
\setbeamercolor{block body}{bg=gray!30}
%\setbeamercolor{title}{use=structure, bg=gray!10}

\AtBeginSection[]
{
	\begin{frame}[noframenumbering]
	\frametitle{Outline}
	\tableofcontents[currentsection]
	\end{frame}
}


\begin{document}
\title{Cometric Association Schemes}
\date[September 14, 2018]{PhD Thesis Proposal\\September 14, 2018}
%\author{Brian Kodalen}
\institute[WPI]{Worcester Polytechnic Institute}
\frame{\titlepage}

\begin{frame}
\frametitle{Outline}
	\tableofcontents
\end{frame}

\section{Association schemes}
\begin{frame}
	\frametitle{Symmetric Association Scheme}
	A \textbf{symmetric association scheme} is an ordered pair $(X,\mathcal{R})$ where $X$ is a finite set and $\mathcal{R} = \left\{R_0,\dots,R_d\right\}$ is a partition of $X\times X$ into $d+1$ symmetric binary relations satisfying
	\begin{itemize}
		\item $R_0$ is the identity relation on $X$;
		\item for $0\leq i,j,k\leq d$, there exists $p^k_{ij}\in\mathbb{Z}$ such that
		\[\vert R_i(a)\cap R_j(b)\vert = p^k_{ij}\]
		whenever $(a,b)\in R_k$.
	\end{itemize}
\end{frame}

\begin{frame}
\frametitle{Example: Heawood Graph}
\centering
\scalebox{.9}{\begin{tikzpicture}[shorten >=1pt,auto,node distance=2cm,
thin,main node/.style = {circle,draw, minimum size = 5pt}]
\def\r{3.5}
\foreach \t in {0,...,13}
{{\node[main node] at ({\r*cos(180*(1+\t / 7))},{\r*sin(180*(\t / 7))}) (\t) {\t};}
{\node[main node] at ({\r*cos(180*(1+\t / 7))},{\r*sin(180*(\t / 7))}) (\t) {\t};}}
\draw[-] (13)--(0)--(1)--(2)--(3)--(4)--(5)--(6)--(7)--(8)--(9)--(10)--(11)--(12)--(13);
\draw[-] (0)--(5)--(4)--(9)--(8)--(13)--(12)--(3)--(2)--(7)--(6)--(11)--(10)--(1);
%base
\only<2->{
\foreach \t in {0}
{{\node[main node] at ({\r*cos(180*(1+\t / 7))},{\r*sin(180*(\t / 7))}) (\t) {\t};}
	{\node[main node,fill=blue] at ({\r*cos(180*(1+\t / 7))},{\r*sin(180*(\t / 7))}) (\t) {\t};}}}
%distance 1
\only<3->{
\foreach \t in {1,5,13}
{{\node[main node] at ({\r*cos(180*(1+\t / 7))},{\r*sin(180*(\t / 7))}) (\t) {\t};}
	{\node[main node,fill=red] at ({\r*cos(180*(1+\t / 7))},{\r*sin(180*(\t / 7))}) (\t) {\t};}}}
%distance 2
\only<4->{
\foreach \t in {2,10,12,8,4,6}
{{\node[main node] at ({\r*cos(180*(1+\t / 7))},{\r*sin(180*(\t / 7))}) (\t) {\t};}
	{\node[main node,fill=green] at ({\r*cos(180*(1+\t / 7))},{\r*sin(180*(\t / 7))}) (\t) {\t};}}}
%distance 3
\only<5->{
\foreach \t in {3,7,9,11}
{{\node[main node] at ({\r*cos(180*(1+\t / 7))},{\r*sin(180*(\t / 7))}) (\t) {\t};}
	{\node[main node,fill=yellow] at ({\r*cos(180*(1+\t / 7))},{\r*sin(180*(\t / 7))}) (\t) {\t};}}}
\end{tikzpicture}}
\end{frame}

\begin{frame}
\frametitle{Bose-Mesner Algebra}
Let $A_i$ denote the adjacency matrix of the graph $(X,R_i)$.
\[\mathbb{A} = \text{span}_\mathbb{R}(A_0,A_1,\dots,A_d)\]
\begin{itemize}
	\item commutative algebra of symmetric matrices;
	\item closed under entrywise products.
\end{itemize}
\vfill
Second basis using idempotents:
\[\mathbb{A} = \text{span}_\mathbb{R}(E_0,E_1,\dots,E_d)\]

\end{frame}




\begin{frame}
\frametitle{$P$-polynomial (metric)}
$(X,\mathcal{R})$ is {\em$P$-polynomial} if we can order the relations so that:
\[p^k_{ij}\begin{cases}
= 0, & k>i+j \text{ or } k<\vert i-j\vert\\
> 0, & k = i+j
\end{cases}\]
Examples:
\begin{multicols}{2}
	\begin{itemize}
	\item SRGs
	\item Generalized Quadrangles
	\item Projective Planes
	\item Moore Graphs
	\end{itemize}
\end{multicols}
\end{frame}

\begin{frame}
\frametitle{Q-polynomial (cometric)}
$(X,\mathcal{R})$ is {\em $Q$-polynomial} if we can order the idempotents so that:
\[q^{k}_{ij}\begin{cases}
=0 & k>i+j\text{ or }k<\vert i-j\vert\\
>0 & k = i+j\\
\end{cases}\]
\begin{itemize}
\item\textit{Q-antipodal} if $q^{k}_{dd}>0 \iff k\in\left\{0,d\right\}$\\
\item\textit{Q-bipartite} if $q^{k}_{ij}=0 \text{ whenever } i+j+k\text{ is even.}$
\end{itemize}
\end{frame}

\begin{frame}
	\frametitle{$P$-polynomial vs. $Q$-polynomial}
	\begin{table}
		\centering
		\begin{tabular}{lll}
			\multicolumn{3}{c}{Major theorems} \\[.5\normalbaselineskip]
			Theorem & Metric & Cometric \\
			\midrule
			1.~Multiple polynomial orderings& 1980 & 1998- \\
			 &  & \quad2014 \\[.5\normalbaselineskip]
			2.~Imprimitive classification & 1971- & 1998- \\
			 & \quad1980 & \quad2010 \\[.5\normalbaselineskip]
			3.~Unimodality of parameters & 1978 & still open \\[.5\normalbaselineskip]
			\midrule
		\end{tabular}
	\end{table}
\end{frame}

\begin{frame}
	\frametitle{A conjecture of Bannai \& Ito}
	\begin{conjec}[Bannai and Ito]
		For $d$ sufficiently large, a primitive association scheme with $d$ classes is metric if and only if it is cometric.
	\end{conjec}
	\vspace{1cm}
	\[\text{How about low $d$?}\]
\end{frame}

\begin{frame}
	\frametitle{$Q$-polynomial association schemes with low class number}
	\begin{tikzpicture}
		\def\s{2}
		\def\x{0}
		\def\y{0}
		\def\t{2}
		\def\l{.2}
		\def\b{7.5}
		\def\labeloffset{1.5}
		\node at (\x-2.5,\y) (classes) {d};
		\draw[-] (\x-3,\y-\l) -- (\x-2,\y-\l) -- (\x-2,\y);
		
		\draw[-] (\x-3,\y-\t/2) -- (\x-2,\y-\t/2);
		\node at (\x-2.5,\y-\t) () {3};
		\draw[-] (\x-3,\y-3*\t/2) -- (\x-2,\y-3*\t/2);
		\node at (\x-2.5,\y-2*\t) () {4};
		\draw[-] (\x-3,\y-5*\t/2) -- (\x-2,\y-5*\t/2);
		\node at (\x-2.5,\y-3*\t) () {5};
		\draw[-] (\x-3,\y-7*\t/2) -- (\x-2,\y-7*\t/2);
		
		\only<2>{\draw[-,color=blue] (\x-3,\y-\t/2) -- (\x-2,\y-\t/2);
		\node[color=blue] at (\x-2.5,\y-\t) () {3};
		\draw[-,color=blue] (\x-2,\y-\t/2) -- (\x,\y);
		\draw[-,color=blue] (\x-2,\y-3*\t/2) -- (\x,\y-\b);}
		
		\only<2-3>{\draw[-,color=blue] (\x-3,\y-3*\t/2) -- (\x-2,\y-3*\t/2);}
		
		\only<3>{\node[blue] at (\x-2.5,\y-2*\t) () {4};
		\draw[-,color=blue] (\x-2,\y-3*\t/2) -- (\x,\y);
		\draw[-,color=blue] (\x-2,\y-5*\t/2) -- (\x,\y-\b);}
		
		\only<3-4>{\draw[-,blue] (\x-3,\y-5*\t/2) -- (\x-2,\y-5*\t/2);}
		
		\only<4>{\node[blue] at (\x-2.5,\y-3*\t) () {5};
		\draw[-,blue] (\x-3,\y-7*\t/2) -- (\x-2,\y-7*\t/2);
		\draw[-,color=blue] (\x-2,\y-5*\t/2) -- (\x,\y);
		\draw[-,color=blue] (\x-2,\y-7*\t/2) -- (\x,\y-\b);}
		
		
		
		\draw[-] (\x+2*\s,\y) -- (\x+2*\s,\y-\b);
		\draw[-] (\x,\y-2*\s) -- (\x+4*\s,\y-2*\s);
		\node at (\x+\s,\y-\s+\labeloffset) (prim) {primitive};
		\node at (\x+3*\s,\y-\s+\labeloffset) (bip) {$Q$-bipartite};
		\node at (\x+\s,\y-3*\s+\labeloffset) (ant) {$Q$-antipodal};
		\node at (\x+3*\s,\y-3*\s+\labeloffset) (both) {both};
		\only<1>{
			\node[align=center] at (\x+\s,\y-\s) (prim) {Spherical \\ $d$-distance set};
			\node[align=center] at (\x+3*\s,\y-1.2*\s) (bip) {Inner products: \\ $\left\{-1,\pm \alpha,\dots\right\}$\\\\``lines through\\ the origin"};
			\node[align=center] at (\x+\s,\y-3*\s) (ant) {Fibers with \\ inner products:\\ $1>\beta_1>\alpha_2>\beta_2>\dots$};
			\node[align=center] at (\x+3*\s,\y-3*\s) (both) {};
		}
		\only<2>{
				\node[align=center] at (\x+\s,\y-\s) (prim) {Spherical\\$3$-distance set};
				\node[align=center] at (\x+3*\s,\y-\s) (bip) {Equiangular lines \\ (two-graphs)\\\\(Haantjes `48)};
				\node[align=center] at (\x+\s,\y-3*\s) (ant) {``\textit{LSSD}"\\$[Thm]$\\(Cameron `72)};
				\node[align=center] at (\x+3*\s,\y-3*\s) (both) {Trivial $LSSD$\\\\ uses $(v,1,0)$};
		}
		\only<3>{
			\node[align=center] at (\x+\s,\y-\s) (prim) {Spherical\\$4$-distance set};
			\node[align=center] at (\x+3*\s,\y-\s) (bip) {Two unique angles\\ (w/ $90^\circ$)\\\\Double covers of SRGs};
			\node[align=center] at (\x+\s,\y-3*\s) (ant) {``\textit{LSSRD}"\\$[Thm]$\\(Higman `95)};
			\node[align=center] at (\x+3*\s,\y-3*\s) (both) {Mutually Unbiased \\ Bases\\$[Thm]$\\(DGS `75)};
		}
		\only<4>{
			\node[align=center] at (\x+\s,\y-\s) (prim) {Spherical\\$5$-distance set};
			\node[align=center] at (\x+3*\s,\y-\s) (bip) {Two unique angles\\ (w/ob $90^\circ$)\\\\Double covers of SRGs};
			\node[align=center] at (\x+\s,\y-3*\s) (ant) {``\textit{LSSRD}"\\$[Thm]$\\(Higman `95)};
			\node[align=center] at (\x+3*\s,\y-3*\s) (both) {bip-double of\\ SRG $\left(q_{11}^1=0\right)$\\$[Thm]$};
		}
	\end{tikzpicture}
\end{frame}

\section{Connectivity}
\begin{frame}
	\frametitle{Background}
	\begin{itemize}
		\item[] Brouwer \& Mesner, 1985
		\item The vertex connectivity of a SRG is equal to its valency, and the only disconnecting sets of minimum size are the neighborhoods of its vertices.
	\end{itemize}
\begin{minipage}{.58\textwidth}
	\begin{itemize}
		\item[] Brouwer \& Koolen, 2009
		\item Replaced ``SRG" with ``DRG".
	\end{itemize}
\end{minipage}
\begin{minipage}{.4\textwidth}
	\begin{tikzpicture}[shorten >=1pt,auto,node distance=2cm,
	thin,main node/.style = {circle,draw, inner sep = 1pt, minimum size = 3pt}]
	\node[main node] at (0,0) (1) {1};
	\node[main node,fill=red] at (.5,.7) (2) {6};
	\node[main node] at (1.5,.7) (3) {7};
	\node[main node] at (1,2.2) (4) {9};
	\node[main node,fill=red] at (2,0) (5) {2};
	\node[main node] at (2,1.7) (6) {8};
	\node[main node] at (0,1.7) (7) {10};
	\node[main node,fill=red] at (-1,2) (8) {5};
	\node[main node] at (3,2) (9) {3};
	\node[main node] at (1,3) (10) {4};
	\draw[-] (1)--(5)--(9)--(10)-- (8)--(1)--(2)--(4)--(3)--(7)--(6)--(2);
	\draw[-] (7)--(8)--(10)--(4)--(3)--(5)--(9)--(6);
	\end{tikzpicture}
	\end{minipage}
\end{frame}

\begin{frame}
\frametitle{Theorem on Connectivity}
\begin{thm}[K.,Martin]
	Let $(X,\mathcal{R})$ be a symmetric association scheme. Assume the graph $\Gamma = (X,R_i)$ is connected and not complete multipartite. Let $H=H_i$ be the corresponding unweighted distribution diagram on $\left\{0,1,\dots,d\right\}$. The following are equivalent:
	\begin{enumerate}
		\item there exists $a\in X$ for which the subgraph $\Gamma\backslash a^\perp$ is connected;
		\item for all $a\in X$, the subgraph $\Gamma\backslash a^\perp$ is connected;
		\item the subgraph $H\backslash \left\{0,i\right\}$ is connected;
		\item $\Gamma$ contains no twins.
	\end{enumerate}
\end{thm}  
\end{frame}


\begin{frame}
\frametitle{$\Gamma \rightarrow H$ via the projection map }
\[\begin{tikzpicture}[shorten >=1pt,auto,node distance=1.5cm,
thin,main node/.style = {circle,draw, inner sep = 1pt, minimum size = 8pt}]
\node at (1.5,3) (g) {$\Gamma$};
\node at (6.5,3) (h) {$H$};
\node[main node] (1) {a};
\node[main node,fill=red] [right of = 1](2) {};
\node[main node,fill=red] [above of = 1](3) {};
\node[main node,fill=blue] [right of = 3](4) {};
\node[main node,fill=red] [above right of = 1](5) {};
\node[main node,fill=blue] [right of = 5](6) {};
\node[main node,fill=blue] [above of = 5] (7) {};
\node[main node,fill=green] [right of = 7](8) {};

\node[main node,] [right of = 6](9) {0};
\node[main node,fill=red] [right of = 9](10) {1};
\node[main node,fill=blue] [right of = 10](11) {2};
\node[main node,fill=green] [right of = 11](12) {3};

\draw[-] (3)--(1)--(2)--(4)--(3)--(7)--(8)--(6)--(5)--(7);
\draw[-] (1)--(5)--(6)--(2)--(4)--(8);
\draw[-] (9)--(10)--(11)--(12);


{\draw[->,green,line width = 0.5mm] (7) -- (8);
 \draw[->,green,line width = 0.5mm] (8) -- (4);}
{\draw[->,green,line width = 0.5mm] (11) -- (12);
\draw[->,green,line width = 0.5mm] (12) -- (11);}
\draw[->,purple,line width = 0.5mm] (1)->(2);
\draw[->,purple,line width = 0.5mm] (2)->(4);
\draw[->,purple,line width = 0.5mm] (4)->(3);
\draw[->,purple,line width = 0.5mm] (9)->(10);
\draw[->,purple,line width = 0.5mm] (10)->(11);
\draw[->,purple,line width = 0.5mm] (11)->(10);
\end{tikzpicture}\]
\begin{itemize}
	\item Need to show $\Gamma\backslash a^\perp$ is connected if and only if $H\backslash \left\{0,1\right\}$ is as well.
\end{itemize}
\end{frame}


\section{Linked simplices}
\begin{frame}
\frametitle{Two simplices in $\mathbb{R}^3$}
\begin{columns}
	\begin{column}{.6\textwidth}
		\begin{figure}
			\includegraphics[scale=.2,trim={2cm 0cm 0cm 2cm}, clip]{twotetrahedra}
			\caption*{\tiny ``Compound of two tetrahedra"\\by Tomruen licensed under CC SA 1.0}
		\end{figure}
	\end{column}
	\begin{column}{.35\textwidth}
		\begin{tikzpicture}[shorten >=1pt,auto,node distance=2cm,
		thin,main node/.style = {circle,draw}]
			\node[main node,fill=yellow] at (0,0) (1) {};
			\node[main node,fill=yellow] at (0,1) (2) {};
			\node[main node,fill=yellow] at (0,2) (3) {};
			\node[main node,fill=yellow] at (0,3) (4) {};
			\node[main node,fill=red] at (2,0) (5) {};
			\node[main node,fill=red] at (2,1) (6) {};
			\node[main node,fill=red] at (2,2) (7) {};
			\node[main node,fill=red] at (2,3) (8) {};
			\draw[-] (1) -- (6) -- (3) -- (8) -- (1);
			\draw[-] (2) -- (7) -- (4) -- (5) -- (2);
			\draw[-] (1) -- (7);
			\draw[-] (2) -- (8);
			\draw[-] (3) -- (5);
			\draw[-] (4) -- (6);
			\node[align=center] at (1,-1) () {Adjacency denotes an\\ inner product of $-\frac{1}{3}$.};		
		\end{tikzpicture}
	\end{column}
\end{columns}\end{frame}

\begin{frame}
\frametitle{Three fibers?}

		\[\begin{tikzpicture}[shorten >=1pt,auto,node distance=2cm,
		thin,main node/.style = {circle,draw}]
		\node[main node,fill=yellow] at (0,0) (y1) {};
		\node[main node,fill=yellow] at (-1,-1) (y2) {};
		\node[main node,fill=yellow] at (-2,-2) (y3) {};
		\node[main node,fill=yellow] at (-3,-3) (y4) {};
		\node[main node,fill=red] at (2,0) (r1) {};
		\node[main node,fill=red] at (3,-1) (r2) {};
		\node[main node,fill=red] at (4,-2) (r3) {};
		\node[main node,fill=red] at (5,-3) (r4) {};
		\node[main node,fill=black] at (-1.6,-5) (b1) {};
		\node[main node,fill=black] at (.2,-5) (b2) {};
		\node[main node,fill=black] at (1.8,-5) (b3) {};
		\node[main node,fill=black] at (3.6,-5) (b4) {};
		
		\only<1-3,6>{\draw[-] (y1)-- (r2);
		\draw[-] (y1)-- (r3);
		\draw[-] (y1)-- (r4);
		\draw[-] (y2)-- (r1);
		\draw[-] (y2)-- (r3);
		\draw[-] (y2)-- (r4);
		\draw[-] (y3)-- (r2);
		\draw[-] (y3)-- (r1);
		\draw[-] (y3)-- (r4);
		\draw[-] (y4)-- (r2);
		\draw[-] (y4)-- (r3);
		\draw[-] (y4)-- (r1);}
		\only<2-3,6>{
			\draw[-] (b1)-- (r2);
			\draw[-] (b1)-- (r3);
			\draw[-] (b1)-- (r4);
			\draw[-] (b2)-- (r1);
			\draw[-] (b2)-- (r3);
			\draw[-] (b2)-- (r4);
			\draw[-] (b3)-- (r2);
			\draw[-] (b3)-- (r1);
			\draw[-] (b3)-- (r4);
			\draw[-] (b4)-- (r2);
			\draw[-] (b4)-- (r3);
			\draw[-] (b4)-- (r1);}
		\only<3,6>{
			\draw[-] (y1)-- (b2);
			\draw[-] (y1)-- (b3);
			\draw[-] (y1)-- (b4);
			\draw[-] (y2)-- (b1);
			\draw[-] (y2)-- (b3);
			\draw[-] (y2)-- (b4);
			\draw[-] (y3)-- (b2);
			\draw[-] (y3)-- (b1);
			\draw[-] (y3)-- (b4);
			\draw[-] (y4)-- (b2);
			\draw[-] (y4)-- (b3);
			\draw[-] (y4)-- (b1);}
		\only<4-5>{\draw[-,very thin,gray] (y1)-- (r2);
			\draw[-,very thin,gray] (y1)-- (r3);
			\draw[-,very thin,gray] (y1)-- (r4);
			\draw[-,very thin,gray] (y2)-- (r1);
			\draw[-,very thin,gray] (y2)-- (r3);
			\draw[-,very thin,gray] (y2)-- (r4);
			\draw[-,very thin,gray] (y3)-- (r2);
			\draw[-,very thin,gray] (y3)-- (r1);
			\draw[-,very thin,gray] (y3)-- (r4);
			\draw[-,very thin,gray] (y4)-- (r2);
			\draw[-,very thin,gray] (y4)-- (r3);
			\draw[-,very thin,gray] (y4)-- (r1);
			\draw[-,very thin,gray] (b1)-- (r2);
			\draw[-,very thin,gray] (b1)-- (r3);
			\draw[-,very thin,gray] (b1)-- (r4);
			\draw[-,very thin,gray] (b2)-- (r1);
			\draw[-,very thin,gray] (b2)-- (r3);
			\draw[-,very thin,gray] (b2)-- (r4);
			\draw[-,very thin,gray] (b3)-- (r2);
			\draw[-,very thin,gray] (b3)-- (r1);
			\draw[-,very thin,gray] (b3)-- (r4);
			\draw[-,very thin,gray] (b4)-- (r2);
			\draw[-,very thin,gray] (b4)-- (r3);
			\draw[-,very thin,gray] (b4)-- (r1);
			\draw[-,very thin,gray] (y1)-- (b2);
			\draw[-,very thin,gray] (y1)-- (b3);
			\draw[-,very thin,gray] (y1)-- (b4);
			\draw[-,very thin,gray] (y2)-- (b1);
			\draw[-,very thin,gray] (y2)-- (b3);
			\draw[-,very thin,gray] (y2)-- (b4);
			\draw[-,very thin,gray] (y3)-- (b2);
			\draw[-,very thin,gray] (y3)-- (b1);
			\draw[-,very thin,gray] (y3)-- (b4);
			\draw[-,very thin,gray] (y4)-- (b2);
			\draw[-,very thin,gray] (y4)-- (b3);
			\draw[-,very thin,gray] (y4)-- (b1);}
		\only<4>{
			\draw[-,thick] (r1) -- (y2);
			\draw[-,red,thick] (r1)-- (b2);
			\draw[-,red,thick] (r1)-- (b3);
			\draw[-,red,thick] (r1)-- (b4);
			\draw[-,yellow,thick] (y2)-- (b1);
			\draw[-,yellow,thick] (y2)-- (b3);
			\draw[-,yellow,thick] (y2)-- (b4);	}
		\only<5>{
			\draw[dashed,thick] (r1) -- (y1);
			\draw[-,red,thick] (r1)-- (b2);
			\draw[-,red,thick] (r1)-- (b3);
			\draw[-,red,thick] (r1)-- (b4);
			\draw[-,yellow,thick] (y1)-- (b2);
			\draw[-,yellow,thick] (y1)-- (b3);
			\draw[-,yellow,thick] (y1)-- (b4);	}
		\end{tikzpicture}\]\pause\pause\pause\pause\pause
		\vspace{-1cm}
		\begin{center}
			Not representable as (distinct) simplices in $\mathbb{R}^3$
		\end{center}
\end{frame}

\begin{frame}
\frametitle{Def: Linked Systems of Symmetric Designs (Cameron)}
Let $\Gamma$ be a graph with vertex set $X$ and adjacency relation $\sim$. We say $\Gamma$ is a \textbf{linked system of symmetric $(v,k,\lambda)$ designs (LSSD) with $w$ fibers} if it is possible to partition $X$ into $w$ vertex subsets $X_1,\dots,X_w$ such that
\begin{itemize}
	\item no edge joints two vertices in the same fiber $X_i$
	\item the subgraph induced between any $X_i$ and $X_j$ $(i\neq j)$ is the incidence graph of some symmetric $(v,k,\lambda)$ design (so $\vert X_i\vert = v$ for all $i$)
	\item for distinct $i,j,h$, if $a\in X_i$ and $b\in X_j$,
	\[\left\vert \Gamma(a)\cap\Gamma(b)\cap X_h\right\vert = \begin{cases}
	\mu \text{ if } a\sim b;\\
	\nu \text{ if } a\not\sim b.
	\end{cases} \]
\end{itemize}
\end{frame}

\begin{frame}
\frametitle{Def: Linked Simplices}
Regular simplex:
\begin{itemize}
	\item $v$ unit vectors spanning $\mathbb{R}^{v-1}$;
	\item pairwise inner products of $-\frac{1}{v-1}$.
\end{itemize}
\vfill
Simplices $A$ and $B$ are ``linked" if there exists $-1\leq\gamma,\delta<1$ with:
\[\forall (a,b)\in A\times B: \left<a,b\right> \in\left\{\gamma,\delta\right\}\]
\vfill
Simplices $A_1,\dots,A_w$ are ``linked" if every pair is linked using the same constants.
\end{frame}



\begin{frame}
\frametitle{Linked Simplices vs. LSSDs}
\begin{thm}[K.]
	Let $\left\{a_i\right\}$ and $\left\{b_j\right\}$ be linked simplices in $\mathbb{R}^{v-1}$ with inner products $\gamma$ and $\delta$. For each $j$, let $B_j=\left\{a_i:\left<a_i,b_j\right>=\gamma\right\}$. Then $\left(\left\{a_i\right\},\left\{B_j\right\}\right)$ is a symmetric $2$-design.
\end{thm}
\begin{thm}[K.]
	Let $\left\{a_i\right\}$, $\left\{b_i\right\}$, and $\left\{c_i\right\}$ be three linked simplices in $\mathbb{R}^{v-1}$ with inner products $\gamma$ and $\delta$. For each $1\leq j,k\leq v$, let $B_j=\left\{a_i:\left<a_i,b_j\right>=\gamma\right\}$ and $C_k=\left\{a_i:\left<a_i,c_k\right>=\gamma\right\}$. Then there exists integers $\mu$ and $\nu$ such that
	\[\forall j,k: \left\vert B_j\cap C_k\right\vert = \begin{cases}
	\mu & \left<b_j,c_k\right> = \gamma\\
	\nu & \left<b_j,c_k\right> = \delta
	\end{cases}\]
\end{thm}
\end{frame}

\begin{frame}
\frametitle{New LSSDs using Menon parameters}
\begin{thm}
	Given a regular Hadamard matrix of order $s$ and an orthogonal array of size $s^2\times N$, there exists a $LSSD$ with $v=s^2$ and $w=N$.
\end{thm}
\begin{cor}
	For sufficiently large $s$, there exists a $LSSD(s^2,k,\lambda;w)$ with $w\geq s^\frac{1}{14.8}$ if there exists a regular Hadamard matrix of order $s$.
\end{cor}
\begin{cor}
	For any $n\geq 1$ and $w>2$, there exists an odd $t$ permitting a $LSSD(16^{n}t,k,\lambda;w)$.
\end{cor}
\end{frame}

%\begin{frame}
%\frametitle{Table of new LSSDs}
%Given a regular Hadamard matrix of order $4t^2$, there exists a $LSSD(16t^4,k,\lambda;w(t))$.\vfill
%\scalebox{.7}{\begin{tabular}{c|*{17}{c}}
%		$t$ & 1 & 2 & 3 & 4 & 5 & 6 & 7 & 8 & 9 & 10& 11 & 12 & 13 & 14 & 15 & 16 & 17 \\\hline
%		$w(t)$&5&17&9&65&10&12&8&257&10&17&17&10&10&10&29&1025&10\\\\
%		$t$ & 18 & 19 & 20& 21 & 22 & 23 & 24 & 25 & 26 & 27 & 28 & 29 & 30& 31 & 32 & 33 \\\hline
%		$w(t)$&26&11&26&11&17&11&32&10&17&10&50&30&30&12&4097&32\\\\
%		$t$ & 34 & 35 & 36 & 37 & 38 & 39 & 40 & 41 & 42 & 43 & 44 & 45 & 46 & 47 & 48 & 49 &\\\hline
%		$w(t)$&18&32&65&32&18&32&26&13&20&32&65&17&32&32&30&17
%\end{tabular}}\vfill
%\end{frame}

\section{Sch\"{o}nberg's theorem}
\begin{frame}
	\frametitle{Sch\"{o}nberg's theorem}
	A function $f:[-1,1]\rightarrow \mathbb{R}$ is \textit{positive definite} if, for every finite subset $X$, $f\circ\left(G_X\right)\succeq 0$, where $G_X$ is the Gram matrix of $X$.\vspace{1cm}
	\begin{thm}[Sch\"{o}nberg,`45]
		A continuous function $f:[-1,1]\rightarrow \mathbb{R}$ is positive definite if and only if $f$ is expressible as a non-negative linear combination of the Gegenbauer polynomials.
	\end{thm}
\end{frame}

\begin{frame}
	\frametitle{Gegenbauer polynomials}
	\[\begin{aligned}
	Q_0^{(m)} &= 1 \qquad Q_1^{(m)} = t\\
	Q_{k}^{(m)} &= \frac{(2k+m-4)tG_{k-1}^{(m)}(t)-(k-1)G_{k-2}^{(m)}(t)}{k+m-3}
	\end{aligned}\]\vfill
	$$Q_0^m(t)=1\qquad	Q_1^m(t)=t\qquad Q_2^m(t)=\frac{mt^2 - 1}{m-1}$$
	\begin{center}
	\scalebox{.9}{$Q_3^m(t)=\frac{(m+2)t^3 - 3t}{(n-1)}\qquad Q_4^m(t)=\frac{(m+4)(m+2)t^4 - 6(m+2)t^2+3}{m^2-1}$}\\\vspace{4mm}
	\scalebox{.9}{$Q_5^m(t)=\frac{(m+6)(m+4)t^5-10(m+4)t^3+15t}{m^2-1}$}%\\\vspace{2mm}
	%\scalebox{.9}{$Q_6^m(t)=\frac{(m+8)(m+6)(m+4)t^6-15(m+6)(m+4)t^4+45(m+4)t^2-15}{(m+3)(m+1)(m-1)}$}
	\end{center}
\end{frame}

\begin{frame}
	\frametitle{$Q$-polynomial consequences}
	\begin{itemize}
	\item Using $Q_4^m(t)$:\[\left(q_{11}^1\right)^2 + q_{12}^1q_{11}^2\geq\frac{2m(m-1)}{m+2}\]
	\item Using $Q_5^m(t)$:\[\left(q_{11}^1\right)^2 + \left(2+\frac{q^2_{12}}{q^1_{11}}\right)q_{12}^1q_{11}^2\geq\frac{4m(2m-3)}{m+6}\]
	rules out 3-class primitive $Q$-polynomial schemes with $\left(\vert X\vert;m_0\right)$:
	\[\begin{aligned}\left\{(441; 20)\right.&; (576; 23); (729; 26); (1015; 28); (1240; 30); (1548; 35);\\ &\left. (1836; 35); (1944; 29); (1976; 25); (1000; 27); (1331; 30)\right\}
	\end{aligned}\]
	\end{itemize}
\end{frame}

\begin{frame}
\frametitle{4-class $Q$-bipartite}
For fixed smallest eigenvalue -49:
\only<1>{\begin{figure}
		\includegraphics[width=2.5in,height=2.5in]{onlybound7}
\end{figure}}
\only<2>{\begin{figure}
		\includegraphics[width=2.5in,height=2.5in]{bounds7}
\end{figure}}
\only<3>{\begin{figure}
		\includegraphics[width=2.5in,height=2.371in]{geg7}
\end{figure}\vspace{.129in}}
\end{frame}

\begin{frame}[noframenumbering]
\centering{\Huge Thanks for your time!}
\end{frame}


\end{document}


\section{Linked simplices}
\subsection*{test}
\begin{frame}
\frametitle{(real) Mutually Unbiased Bases}
A set of $w$ orthonormal bases of $\mathbb{R}^n$ such that
\[\left<a,b\right> = \pm\frac{1}{\sqrt{n}}\]
whenever $a$ and $b$ are from distinct bases.\\\pause
\vfill
What is the maximum size, $M(n)$, of a set of MUBs in $\mathbb{R}^n$?
\begin{itemize}
	\item 30 year old problem. (Ivonovic `81, Wootters \& Fields `89)
	\item $M(n)>1$ only if $n=4t$ for some $t\in\mathbb{N}$. (Boykin et.\ al.\ `08)
	\item $M(n)\leq n/2+1$. (Delsarte, Goethals, Seidel `75)
	\item Tight examples exist whenever $n = 4^t$. (Cameron, Seidel `73)
\end{itemize}
\end{frame}
\begin{frame}
	\frametitle{(real) Equiangular Lines}
A set of $w$ vectors in $\mathbb{R}^n$ such that
\[\left<a,b\right> = \pm \alpha\]
for fixed $\alpha$ whenever $a\neq b$.\\\pause
\vfill
What is the maximum size, $N(n)$, of a set of equiangular lines in $\mathbb{R}^n$?
\begin{itemize}
	\item 70 year old problem. (Haantjes `48)
	\item Closely related to graphs. (Van Lint \& Seidel `66)
	\item $N(n)\leq\binom{n+1}{2}$. (Gerzon `71)
	\item If $N(n)>2n$, $\frac{1}{\alpha}$ is an odd integer. (Neumann `71)
\end{itemize}
\end{frame}

\begin{frame}
\begin{columns}
	\begin{column}{0.7\textwidth}
	\frametitle{Two simplices in $\mathbb{R}^2$}
	\[\begin{tikzpicture}
		\coordinate (a) at (0,0);
		\coordinate (b) at (0,3);
		\coordinate (c) at (0,-3);
		\coordinate (d) at ($(0,-3/2) + sqrt(3)*(3/2,0)$);
		\coordinate (e) at ($(0,3/2) + sqrt(3)*(3/2,0)$);
		
		\draw[blue,->,thick] (0,0) -> (0,3);
		\draw[blue,->,thick] (0,0) -> ($(0,-3/2) + sqrt(3)*(3/2,0)$);
		\draw[blue,->,thick] (0,0) -> ($(0,-3/2) + sqrt(3)*(-3/2,0)$);\pause
		\draw[red,->,thick] (0,0) -> (0,-3);
		\draw[red,->,thick] (0,0) -> ($(0,3/2) + sqrt(3)*(3/2,0)$);
		\draw[red,->,thick] (0,0) -> ($(0,3/2) + sqrt(3)*(-3/2,0)$);\pause
		\draw pic["$\alpha$",draw=orange,<->,angle eccentricity=1.2,angle radius=1cm] {angle=b--a--c};
		\draw pic["$\beta$",draw=purple,<->,angle eccentricity=1.2,angle radius=1cm] {angle=d--a--e};
	\end{tikzpicture}\]
	\end{column}
	\begin{column}{0.3\textwidth}
	\only<3->{
	\[\begin{aligned}
	\alpha &= 180^\circ\\
	\beta &= 60^\circ\\\\
	\left<\text{red},\text{blue}\right>&\in\left\{-1,\frac{1}{2}\right\}
	\end{aligned}\]}
	\end{column}
\end{columns}
\end{frame}

\begin{frame}
\frametitle{Two simplices in $\mathbb{R}^3$}
\begin{columns}
	\begin{column}{0.7\textwidth}
		\[\begin{tikzpicture}[line join = round, line cap = round]
		
		\coordinate (a) at (0,0,0);
		\pgfmathsetmacro{\factor}{1/sqrt(2)};
		\coordinate (A) at (2,0,-2*\factor);
		\coordinate (B) at (-2,0,-2*\factor);
		\coordinate (C) at (0,2,2*\factor);
		\coordinate (D) at (0,-2,2*\factor);
		
		\draw[->] (0,0) -- (3,0,0) node[right] {$x$};
		\draw[->] (0,0) -- (0,3,0) node[above] {$y$};
		\draw[->] (0,0) -- (0,0,3) node[below left] {$z$};
		\foreach \i in {A,B,C,D}
		\draw[thick,blue,->] (0,0)--(\i);
		\only<1>{\draw[-,dashed, fill=blue!30, opacity=.2] (A)--(C)--(B)--cycle;
		\draw[-,dashed, fill=red!30, opacity=.2] (A) --(D)--(C)--cycle;
		\draw[-,dashed, fill=purple!30, opacity=.2] (B)--(D)--(C)--cycle;}\pause
		
		
		\coordinate (A1) at (-2,0,2*\factor);
		\coordinate (B1) at (2,0,2*\factor);
		\coordinate (C1) at (0,-2,-2*\factor);
		\coordinate (D1) at (0,2,-2*\factor);
		\foreach \i in {A1,B1,C1,D1}
		\draw[thick,red,->] (0,0)--(\i);
		\only<2>{\draw[-,dashed, fill=blue!30, opacity=.2] (A1)--(C1)--(B1)--cycle;
		\draw[-,dashed, fill=red!30, opacity=.2] (A1) --(D1)--(C1)--cycle;
		\draw[-,dashed, fill=purple!30, opacity=.2] (B1)--(D1)--(C1)--cycle;}\pause
		\draw pic["$\beta$",draw=orange,<->,angle eccentricity=1.2,angle radius=0.7cm] {angle=A--a--D1};
		\draw pic["$\alpha$",draw=purple,<->,angle eccentricity=1.2,angle radius=1cm] {angle=D1--a--D};
		\end{tikzpicture}\]
	\end{column}
	\begin{column}{0.3\textwidth}
		\only<3->{
			\[\begin{aligned}
			\alpha &= 180^\circ\\
			\beta &\approx 70.5^\circ\\\\
			\left<\text{red},\text{blue}\right>&\in\left\{-1,\frac{1}{3}\right\}
			\end{aligned}\]}
	\end{column}
\end{columns}
\end{frame}

\begin{frame}
\frametitle{Linked Simplices}
Regular simplex:
\begin{itemize}
	\item $v$ unit vectors spanning $\mathbb{R}^{v-1}$;
	\item pairwise inner products of $-\frac{1}{v-1}$.
\end{itemize}\pause
\vfill
Simplices $A$ and $B$ are ``linked" if there exists $\gamma,\delta\in\mathbb{R}$ with:
\[\forall (a,b)\in A\times B: \left<a,b\right> \in\left\{\gamma,\delta\right\}\]\pause
\vfill
Simplices $A_1,\dots,A_w$ are ``linked" if every pair is linked using the same constants.
\end{frame}

\begin{frame}
\frametitle{Two simplices in $\mathbb{R}^{v-1}$}
\begin{itemize}
	\item Let $A$ be a regular simplex in $\mathbb{R}^{v-1}$;
	\item let $A' = \left\{-a : a\in A\right\}$.
\end{itemize}
For each $a\in A$ and $a'\in A'$,
\[\left<a,a'\right> \in\left\{-1,\frac{1}{v}\right\}\]
\vfill\pause
Can you find 3 linked simplices?\\\pause
Can you find 2 without using -1 as an inner product?
\end{frame}


\begin{frame}
\frametitle{Pair of Linked Simplices}
\begin{thm}[K. `18]
	Let $\left\{a_i\right\}$ and $\left\{b_j\right\}$ be linked simplices with inner products $\gamma$ and $\delta$. Let $B_j = \left\{a_i: \left<a_i,b_j\right> = \gamma\right\}$. Then $(\left\{a_i\right\},\left\{B_j\right\})$ forms a $(v,k,\lambda)$ symmetric 2-design where $k$ and $\lambda$ depend only $\gamma$ and $\delta$.
\end{thm}\pause
\begin{thm}[K. `18]
	Let $A_1,\dots,A_w$ be $w$ linked simplices with inner products $\gamma$ and $\delta$. Define graph $\Gamma(X,E)$ on $X = \cup_i A_i$ and $E$ given by:\vspace{-.8em}
	\[\text{For }x\in A_i, y\in A_{i'}\quad (i\neq i'):\qquad x\sim y \iff \langle x,y\rangle = \gamma.\]
	Then $\Gamma(X,E)$ is a linked system of symmetric designs.
%\begin{enumerate}[(i)]
%\item For $x,y\in\mathbb{R}^{v-1}: %\sum_{i=1}^{v}\left<a_i,x\right>\left<a_i,y\right> = \frac{v}{v-1}\left<x,y\right>$;
%\end{enumerate}
\end{thm}
\end{frame}


\begin{frame}
\frametitle{Linked system of symmetric designs}
\begin{definition}[Cameron `72]
	A \textit{linked system of symmetric designs} ($LSSD(v,k,\lambda;w)$, $w\geq 2$) is a ($w$-partite) graph $\Gamma$ on $wv$ vertices with the following properties:\vspace{-.5em}
	\begin{enumerate}
		\item $X = \dot{\cup}_{i=1}^wX_i$ with $\vert X_i\vert = v$;
		\item no edge of $\Gamma$ has both ends in the same fiber;\pause
		\item the induced subgraph between distinct fibers is the incidence graph of a $(v,k,\lambda)$-design;\pause
		\item there exist constants $\mu$ and $\nu$ such that for distinct $h,i,j$:
		\vspace{-.5em}\[a\in X_i,b\in X_j\rightarrow\vert \Gamma(a)\cap \Gamma(b)\cap X_h\vert = \begin{cases}
		\mu & a\sim b\\
		\nu & a\not\sim b.\\
		\end{cases}\]
		\vfill
	\end{enumerate}
\end{definition}
\end{frame}

\begin{frame}
	\frametitle{(iv) visualized}
	\[\begin{tikzpicture}[shorten >=1pt,auto,node distance=2cm,
	thin,main node/.style = {circle,draw, inner sep = 2pt},main2 node/.style = {circle,draw, inner sep = 1pt},vertex/.style = {circle, draw, fill, inner sep = 0pt, minimum size=2pt}]
		
		
		\node at (-.3,1) {$X_1$};
		\node at (5.3,1) {$X_2$};
		\node at (2.5,-3.5) {$X_3$};
		\draw (2.5,-2.5) ellipse (1.5cm and .7cm);
		\draw[rotate=30] (0,0) ellipse (1.5cm and .7cm);
		\draw[rotate around={-30:(5,0)}]  (5,0) ellipse (1.5cm and .7cm);;
		\node[vertex,label = 90:x](x) at (0,0) {};
		\node[vertex,label = 90:y](y) at (5,0) {};
		\draw[dashed] (x) -- (y);
		
		\only<2->{\draw[semithick] (x) -- (1.6,-2.55);
		\draw[semithick] (x) -- (2.4,-2.15);
		\draw[semithick] (2.2,-2.5) ellipse (.6 and .4);}
		
		
		
		\only<2->{\draw[semithick] (2.8,-2.5) ellipse (.6 and .4);
		\draw[semithick] (y) -- (2.6,-2.15);
		\draw[semithick] (y) -- (3.38,-2.55);}
	
		\only<3->{\node at (2.5,-2.5) {S};}
	\end{tikzpicture}\]
	\only<4->{\[\vert S\vert = \begin{cases}
		\mu & x\sim y\\
		\nu & x\not\sim y\\
		\end{cases}\]}
\end{frame}

\begin{frame}
\frametitle{Three Linked Simplices}
Given simplices: $\left\{a_i\right\}$, $\left\{b_j\right\}$, $\left\{c_\ell\right\}$:
 \vspace{-.7em}\[\begin{aligned}x_j &= \left[\left<a_1,b_j\right>,\left<a_2,b_j\right>,\dots,\left<a_v,b_j\right>\right]\\
 y_\ell &= \left[\left<a_1,c_\ell\right>,\left<a_2,c_\ell\right>,\dots,\left<a_v,c_\ell\right>\right]
 \end{aligned}\]\vfill\pause
 Let $\eta_{j\ell} = \vert\left\{a_i : \langle a_i,b_j\rangle = \langle a_i,c_\ell\rangle = \gamma \right\}\vert$.
  \vspace{-.5em}
 \[\begin{aligned}
 \langle x_j,y_\ell\rangle &= (\eta_{j\ell})\gamma^2 + 2(k-\eta_{j\ell})\gamma\delta + (v-2k+\eta_{j\ell})\delta^2 \end{aligned}\]\vfill\pause
 A regular simplex is a tight frame, so:\vspace{-.5em}
\[\langle x_j,y_\ell\rangle = \sum_i\langle a_i,b_j\rangle\langle a_i,c_\ell\rangle = \frac{v}{v-1}\langle b_j,c_\ell\rangle
\]\vfill\pause
\end{frame}

\section{Association schemes/LSSDs}
\subsection*{test2}

\begin{frame}
\frametitle{Symmetric association scheme}
	$\mathcal{A}=(X,\mathcal{R})$ with finite set $X$ and symmetric relations $\mathcal{R} = \left\{R_0,R_1,\dots,R_d\right\}$ on $X$ satisfying:
	\begin{itemize}
		\item $R_0$ is the identity relation.
		\item $\mathcal{R}$ is a partition of $X\times X$.
		\item There exist integers $p_{ij}^k$ such that 
		\[\vert \left\{z\in X:(x,z)\in R_i, (z,y)\in R_j\right\}\vert = p^k_{ij}\]
		whenever $(x,y)\in R_k$.
	\end{itemize}
\end{frame}



\begin{frame}
\frametitle{3-class $Q$-antipodal association schemes}
\begin{theorem}[Van Dam `99]
Let $\Gamma$ be a $LSSD$ with adjacency matrix $A$. The algebra $\left<A\right>$ is the Bose-Mesner algebra of a 3-class Q-antipodal association scheme. Conversely, every 3-class Q-antipodal association scheme arises in this way.
\end{theorem}
\end{frame}


\begin{frame}
	\frametitle{First and Second Eigenmatrices}
	\[\begin{aligned}P &= \left[\begin{array}{cccc}
	1 & k(w-1) & v-1 & (v-k)(w-1)\\
	1 & \sqrt{k-\lambda}(w-1) & -1 & -\sqrt{k-\lambda}(w-1)\\
	1 & -\sqrt{k-\lambda} & -1 & \sqrt{k-\lambda}\\
	1 & -k & v-1 & k-v\\
	\end{array}\right]\\
	\only<1>{Q&=\left[\begin{array}{cccc}
	1 & v-1 & (w-1)(v-1) & w-1\\
	1 & \frac{v-k}{\sqrt{k-\lambda}} & -\frac{v-k}{\sqrt{k-\lambda}} & -1\\
	1& -1 & 1-w & w-1\\
	1 & -\frac{k}{\sqrt{k-\lambda}} & \frac{k}{\sqrt{k-\lambda}} & -1
	\end{array}\right]}
	\only<2>{Q&=\left[\begin{array}{c|c|cc}
		\cline{2-2}
	1 &  v-1  & (w-1)(v-1) & w-1\\
	1 &  \frac{v-k}{\sqrt{k-\lambda}}  & -\frac{v-k}{\sqrt{k-\lambda}} & -1\\
	1&  -1  & 1-w & w-1\\
	1 &  -\frac{k}{\sqrt{k-\lambda}}  & \frac{k}{\sqrt{k-\lambda}} & -1\\
	\cline{2-2}
	\end{array}\right]}
	\end{aligned}\]
\end{frame}

\begin{frame}
\frametitle{Gram matrix}
%Let $\gamma = \frac{v-k}{(v-1)\sqrt{k-\lambda}}$ and $\delta = -\frac{k}{(v-1)\sqrt{k-\lambda}}$.
\[cE_1 = \left[\begin{array}{c:c:c}
\begin{smallmatrix}
1 & &-\frac{1}{v-1}\\
& \ddots &\\
-\frac{1}{v-1} & & 1
\end{smallmatrix} & \gamma,\delta &\gamma,\delta\\\hdashline
\gamma,\delta& \begin{smallmatrix}
1 & &-\frac{1}{v-1}\\
& \ddots &\\
-\frac{1}{v-1} & & 1
\end{smallmatrix} & \gamma,\delta\\\hdashline
\gamma,\delta & \gamma,\delta  & \begin{smallmatrix}
1 & &-\frac{1}{v-1}\\
& \ddots &\\
-\frac{1}{v-1} & & 1
\end{smallmatrix} \\
\end{array}\right]\]
So linked simplices are equivalent to LSSDs!
\end{frame}

\section{New Examples/Theorems}
\subsection*{test}

\begin{frame}
	\frametitle{Other geometric objects}
	Given a $LSSD(v,k,\lambda;w)$, we can build:\vfill
	\begin{enumerate}
		\item $vw$ equiangular lines in $\mathbb{R}^{v+w-1}$.
		\begin{itemize}
		\item de Caen's construction (`00) gives $\frac{1}{2}v^2$ lines in dimension $\frac{3}{2}v-1$ when $v = 2^{2r}$.
		\item 11,664 lines in dimension $1304$.
		\end{itemize}\vfill\pause
		\item If $v = 4u^2$ for $u$ even, $w$ real MUBs in $\mathbb{R}^v$.
		\begin{itemize}
			\item $\frac{v}{2}$ MUBs in dimension $v = 2^{2r}$
			\item $9$ MUBs in dimension $v = 1296$.
		\end{itemize}
	\end{enumerate}
\end{frame}

\begin{frame}
\frametitle{Theorems}
\begin{enumerate}
	\item Given a regular Hadamard matrix of order $s$ and an orthogonal array of size $s^2\times N$, there exists a LSSD with $v=s^2$ and $w=N$.\pause
	\item For any $n\geq 1$ and $w>2$, there exists an odd $t$ permitting a $LSSD(16^nt,k,\lambda; w)$.\pause
	\item There exists an $LSSD(v,k,\lambda;w)$ with $v=36^{2n}$ and $w = 4^n+1$ for all $n\geq 1$.
\end{enumerate}
\end{frame}





\end{document}