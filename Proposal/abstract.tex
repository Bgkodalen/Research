\documentclass{article}


\usepackage{tikz}
\usepackage{units}
\usepackage{scalefnt}
\usepackage{bbm}
\usepackage{mathrsfs}
\usepackage{amsmath, amssymb, amsthm, url, graphicx,rotating,tkz-graph,relsize}
\usepackage{graphicx}
\usepackage{fullpage}
\usepackage{enumitem}

%\numberwithin{equation}{section}
\newcommand{\m}{{{24}}}  % Special for this paper, subscripts and superscripts {24}
%\newcommand{\ip}[1]{\langle {#1}\rangle}
\newcommand{\IP}{{\rm IP}}
\newcommand{\re}{{\mathbb R}}
\newcommand{\cx}{{\mathbb C}}
\newcommand{\ints}{{\mathbb Z}}
%\renewcommand{\i}{{\rm i}}
\newcommand{\BMA}{{\mathbb A}}
\newcommand{\A}{{\bf A}}
\newcommand{\B}{{\bf B}}
\newcommand{\cC}{{\mathcal C}} 
\newcommand{\E}{{\mathsf E}}
\newcommand{\cE}{{\mathcal E}}  % Euclidean space
\newcommand{\F}{{\mathcal F}}
\newcommand{\G}{{\mathcal G}}
\newcommand{\I}{{\mathsf I}}
\newcommand{\tI}{\tilde{I}}
\newcommand{\J}{{\mathsf J}}
\renewcommand{\L}{\boldsymbol{\wedge}}  % Leech lattice in \sf font
\newcommand{\Z}{{\mathsf Z}}
\newcommand{\cI}{{\mathcal I}}
\newcommand{\cJ}{{\mathcal J}}
\newcommand{\cR}{{\mathcal R}}
\renewcommand{\S}{{\mathsf S}}
\newcommand{\cS}{{\mathcal S}}
\newcommand{\cT}{{\mathcal T}}
\newcommand{\tU}{\tilde{U}}
\newcommand{\bv}{{\mathbf v}}
\newcommand{\bw}{{\mathbf w}}
\newcommand{\tW}{\tilde{W}}
\newcommand{\X}{{\mathcal X}}
\newcommand{\cZ}{{\mathcal Z}}
\newcommand{\QI}{{\mathcal Q}{\mathcal I}}
\newcommand{\Nm}{{\sf Nm}}
\newcommand{\cP}{{\mathcal P}}
\newcommand{\one}{{\mathbf 1}}
\newcommand{\dis}{\displaystyle}

\newtheorem{theorem}{Theorem}[section]
\newtheorem{corollary}{Corollary}[section]
\newtheorem{lemma}{Lemma}[section]
\newtheorem{remark}{Remark}
\newcommand*{\swap}[2]{#2#1}

\title{\textbf{Cometric Association Schemes}}
\author{Brian G. Kodalen}
\date{}
\begin{document}
	\maketitle
	\vspace{3cm}
	\begin{center} A thesis proposal presented for the degree of\\Doctor of Philosophy.\\
	\vspace{1cm}
	Mathematical Sciences Department\\Worcester Polytechnic Institute\\September 26, 2018\end{center}
	\vspace{3cm}
	\textbf{PhD Committee Members:}
	\begin{itemize}[label={}]
		\itemsep-.5em
		\item Dr. Peter J. Cameron, Queen Mary, University of London
		\item Dr. Padraig \'{O} Cath\'{a}in, WPI
		\item Dr. Peter R. Christopher, WPI
		\item Dr. William M. Kantor, University of Oregon
		\item Dr. William J. Martin, WPI (Advisor)
		\item Dr. G\'{a}bor N. S\'{a}rk\"{o}zy, WPI
	\end{itemize}
	
	\newpage
	The combinatorial objects known as association schemes arise in group theory, extremal graph theory, coding theory, the design of experiments, and even quantum information theory. One may think of a $d$-class association scheme as a $d+1$ dimensional matrix algebra closed under the entrywise product. These algebras admit two distinct bases, one of $01$-matrices and another of idempotents. We call an association scheme ``imprimitive" if there exists non-trivial subalgebras also closed under the entrywise product. These systems of imprimitivity provide us with ``quotient schemes" which are often helpful to understand the structure of the algebra as a whole. We say an association scheme is metric (cometric) whenever we may find polynomials $f_0(t), f_1(t)=t, \dots, f_d(t)$ with $\text{deg}(f_i(t)) = i$ so that the $01$ (idempotent) basis may be represented as $f_0(M), M, f_2(M),\dots, f_d(M)$ for a given basis matrix $M$. Metric association schemes have been studied extensively with fundamental results arising in the early 1970's. However, analogous results for cometric schemes have lagged by more than three decades with many conjectures still unproven today. Further, we observe that many imprimitive cometric schemes give rise to sets of unit vectors with restricted angles between them. Many of these $t$-distance sets have been studied for decades, driving much of the applications mentioned above.\par
	After introducing association schemes with relevant terminology and definitions, this talk focuses on a few recent results regarding cometric schemes, especially imprimitive schemes with small $d$. Our first result partially extends a theorem of Brouwer and Koolen to the general case, classifying all association schemes with connectivity 2. We then introduce sets of unit vectors known as \textit{linked simplices} and show their equivalence to certain 3-class cometric schemes. We use this connection to construct new infinite families of cometric schemes without parameter restrictions which have been required for every previous construction. This connection also allows us to construct other line systems known as equiangular lines and mutually unbiased bases, two line systems in particular which have been an interest for over half a century. We conclude this talk by mentioning a theorem of Sch\"{o}nberg concerning positive definite functions on the sphere and applying it to the idempotent basis matrices of cometric schemes. This theorem allows us to find new restrictions on the parameters of schemes, resolving the existence of many open cases. In particular, we will focus on certain imprimitive 4-class cometric schemes and show that we find new bounds on the valency of such a scheme based on the smallest eigenvalue of its quotient scheme.
	
\end{document}




