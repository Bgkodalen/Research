\documentclass{article}


\usepackage{tikz}
\usepackage{units}
\usepackage{scalefnt}
\usepackage{bbm}
\usepackage{mathrsfs}
\usepackage{amsmath, amssymb, amsthm, url, graphicx,rotating,tkz-graph,relsize}
\usepackage{graphicx}
\usepackage{fullpage}
\usepackage{enumitem}

%\numberwithin{equation}{section}
\newcommand{\m}{{{24}}}  % Special for this paper, subscripts and superscripts {24}
%\newcommand{\ip}[1]{\langle {#1}\rangle}
\newcommand{\IP}{{\rm IP}}
\newcommand{\re}{{\mathbb R}}
\newcommand{\cx}{{\mathbb C}}
\newcommand{\ints}{{\mathbb Z}}
%\renewcommand{\i}{{\rm i}}
\newcommand{\BMA}{{\mathbb A}}
\newcommand{\A}{{\bf A}}
\newcommand{\B}{{\bf B}}
\newcommand{\cC}{{\mathcal C}} 
\newcommand{\E}{{\mathsf E}}
\newcommand{\cE}{{\mathcal E}}  % Euclidean space
\newcommand{\F}{{\mathcal F}}
\newcommand{\G}{{\mathcal G}}
\newcommand{\I}{{\mathsf I}}
\newcommand{\tI}{\tilde{I}}
\newcommand{\J}{{\mathsf J}}
\renewcommand{\L}{\boldsymbol{\wedge}}  % Leech lattice in \sf font
\newcommand{\Z}{{\mathsf Z}}
\newcommand{\cI}{{\mathcal I}}
\newcommand{\cJ}{{\mathcal J}}
\newcommand{\cR}{{\mathcal R}}
\renewcommand{\S}{{\mathsf S}}
\newcommand{\cS}{{\mathcal S}}
\newcommand{\cT}{{\mathcal T}}
\newcommand{\tU}{\tilde{U}}
\newcommand{\bv}{{\mathbf v}}
\newcommand{\bw}{{\mathbf w}}
\newcommand{\tW}{\tilde{W}}
\newcommand{\X}{{\mathcal X}}
\newcommand{\cZ}{{\mathcal Z}}
\newcommand{\QI}{{\mathcal Q}{\mathcal I}}
\newcommand{\Nm}{{\sf Nm}}
\newcommand{\cP}{{\mathcal P}}
\newcommand{\one}{{\mathbf 1}}
\newcommand{\dis}{\displaystyle}

\newtheorem{theorem}{Theorem}[section]
\newtheorem{corollary}{Corollary}[section]
\newtheorem{lemma}{Lemma}[section]
\newtheorem{remark}{Remark}
\newcommand*{\swap}[2]{#2#1}

\title{\textbf{Cometric Association Schemes}}
\author{Brian G. Kodalen}
\date{}
\begin{document}
	\maketitle
	\vspace{3cm}
	\begin{center} A thesis proposal presented for the degree of\\Doctor of Philosophy.\\
	\vspace{1cm}
	Mathematical Sciences Department\\Worcester Polytechnic Institute\\September 26, 2018\end{center}
	\vspace{3cm}
	\textbf{PhD Committee Members:}
	\begin{itemize}[label={}]
		\itemsep-.5em
		\item Dr. Peter J. Cameron, Queen Mary, University of London
		\item Dr. Padraig \'{O} Cath\'{a}in, WPI
		\item Dr. Peter R. Christopher, WPI
		\item Dr. William M. Kantor, University of Oregon
		\item Dr. William J. Martin, WPI (Advisor)
		\item Dr. G\'{a}bor N. S\'{a}rk\"{o}zy, WPI
	\end{itemize}
	
	\newpage
	The combinatorial objects known as association schemes arise in group theory, extremal graph theory, coding theory, the design of experiments, and even quantum information theory. One may think of a $d$-class association scheme as a $d+1$ dimensional matrix algebra closed under the entrywise product. In particular, we consider polynomial schemes -- association schemes for which the basis matrices of the algebra may be generated by a single matrix using $d+1$ polynomials of distinct degrees increasing from $0$ to $d$.\par
	
	The research presented in this proposal focuses mainly on $Q$-polynomial association schemes with low class number. One may think of a $d$-class association scheme as a $d+1$ dimensional matrix algebra closed under the entrywise product. Within this matrix algebra, we find many large, low rank positive semidefinite matrices with very few distinct entries. These matrices equate to Gram matrices of vectors with few angles occurring among them. Two cases of particular interest are equiangular lines and mutually unbiased bases (MUBs) which have been sought after for decades with interest in the former set of lines reaching as far back as 1948. In this proposal, we first review the basics of association schemes and cite a few results concerning polynomial schemes which relate to the original research presented. With this context established, we then proceed to describe three distinct projects which encompass the main results of this dissertation, first mentioning known results in each section where applicable.\par
	First, we will examine connectivity of symmetric association schemes \cite{connectivity} where we show that deleting any vertex $x$ and its neighborhood disconnect a connected basis relation if and only if there is a second vertex $y$ such that the neighborhoods of $x$ and $y$ are exactly the same. This result is then used to partially extend Brouwer's connectivity result for distance-regular graphs to general symmetric association schemes by showing that a basis relation in a symmetric association scheme has connectivity $2$ if and only if that basis relation corresponds to a cycle. Further, we show that basis relations of diameter 2 have connectivity at least four unless they are either a cycle or a cubic graph.\par
	After this result, we move to $3$-class $Q$-antipodal schemes and discuss results recently published in \cite{lssd}. In this context, we introduce a third line system called ``linked simplices" and prove its equivalence to a linked system of symmetric designs (LSSD). These graphs were originally studied by Cameron in 1972 \cite{Cameron} in relation to groups with inequivalent doubly-transitive permutation representations which have the same permutation character. It was later noticed in 2009-2010 that some LSSDs gave rise to MUBs, though the only known examples at the time required the dimension of the space to be a power of 4. As part of the results of our paper, we construct infinite families of non-trivial LSSDs which produce MUBs without this dimensional requirement. We then prove the existence of LSSDs (also giving rise to MUBs) where the number of bases rises independent of the largest power of 4 dividing the dimension. As a final result, we examine the 21 classical families of symmetric designs and classify which families satisfy necessary conditions to appear as induced subgraphs of LSSDs.\par
	Finally, we shift to more general $Q$-polynomial schemes where we have shown that a theorem of Sch\"{o}nberg may be applied to rule out open parameter sets. We describe this technique in general for $d$-class schemes and give current results including a list of nine previously open 3-class primitive cometric schemes. We then discuss more detailed results when we may assume our scheme is 4-class $Q$-bipartite. In this setting, we show that the parameters of a $4$-class $Q$-bipartite scheme are determined completely by the quotient strongly regular graph. We then use Sch\"{o}nberg's theorem to rule out infinitely many open parameter sets, as well as restricting heavily the possible schemes where the smallest eigenvalue of the underlying strongly regular graph lies between $-1$ and $-12$. We finish this section by outlining further questions we wish to investigate within the next half-year.
	
\end{document}




